\documentclass[12pt, twoside]{report}

\usepackage{fontspec}
\usepackage{unicode-math}
\setmainfont{Tex Gyre Pagella}
\setmathfont{TeX Gyre Pagella Math}

\usepackage[a4paper]{geometry}
\geometry{verbose,tmargin=4cm,bmargin=3cm,lmargin=4cm,rmargin=3cm}

\usepackage{fancyhdr}
\pagestyle{fancy}

\setcounter{secnumdepth}{3}
\setcounter{tocdepth}{3}

\usepackage{polyglossia}
\setmainlanguage[variant=us]{english}
\setotherlanguage[variant=brazilian]{portuguese}

\usepackage{setspace}
\onehalfspacing

\usepackage[
unicode=true,
pdfusetitle,
bookmarks=true,
bookmarksnumbered=false,
bookmarksopen=false,
breaklinks=false,
pdfborder={0 0 1},
backref=false,
colorlinks=false
]{hyperref}

\makeatletter

%%%%%%%%%%%%%%%%%%%%%%%%%%%%%%%%%%%%
%%% USER PACKAGES

\usepackage{amsmath}
\usepackage{amsthm}
\usepackage{bm}
\usepackage{mathrsfs}
\usepackage{mathtools}

\usepackage{graphicx}
\usepackage{color}

\usepackage{verbatim}
\usepackage{booktabs}
\usepackage{cancel}
\usepackage{listings}

\usepackage[svgnames]{xcolor}
\usepackage{hyperref}
\hypersetup{colorlinks=true,urlcolor=NavyBlue,linkcolor=NavyBlue,citecolor=CornflowerBlue,naturalnames=true,hypertexnames=true}

\usepackage{doi}

\usepackage[comma]{natbib}
\setcitestyle{numbers,square}

\usepackage{svg}

\usepackage{caption}
\usepackage{subcaption}

\usepackage[none]{hyphenat}
\sloppy

\usepackage{multirow}

\usepackage[withpage]{acronym}

\usepackage{doi}

%%%%%%%%%%%%%%%%%%%%%%%%%%%%%%%%%%%%
%%% MATH MACROS

% Upright d
\newcommand{\ud}{\mathrm{d}}

% Integration element
\usepackage{xifthen}
\newcommand{\intelm}[2] {
  \ifthenelse{\isempty{#2}}
    {\ud #1 \,}
    {\ud^{#2} {#1} \,}
}

% Derivatives
\newcommand{\der}[2]{\frac{\ud #1}{\ud #2}}
\newcommand{\del}[2]{\frac{\partial #1}{\partial #2}}
\newcommand{\nder}[3]{\frac{\ud^{#3} #1}{\ud {#2}^{#3}}}
\newcommand{\ndel}[3]{\frac{\partial^{#3} #1}{\partial {#2}^{#3}}}

% Tensors
\newcommand{\tens}[3]{#1^{#2}_{\phantom{#2} #3}}
\newcommand{\itens}[3]{#1_{#3}^{\phantom{#3} #2}}

% Covariant derivative
\newcommand{\covd}[1]{\tens{\nabla}{}{#1}}
\newcommand{\ucovd}[1]{\tens{\nabla}{#1}{}}

% Metric tensor
\newcommand{\g}{\textsl{g}}
\newcommand{\mtrtens}[2]{\tens{\g}{}{#1#2}}
\newcommand{\umtrtens}[2]{\tens{\g}{#1#2}{}}

% Kronecker delta
\newcommand{\kdelta}[2]{\tens{\delta}{#1}{#2}}

% Connection symbols
\newcommand{\chris}[3]{\tens{\Gamma}{#1}{#2 #3}}

% Index partial derivative
\newcommand{\indexdel}[1]{\tens{\partial}{}{#1}}

% References
\newcommand{\myref}[1]{Eq. (\ref{#1})}

% Definitions, theorems
\newtheorem{theorem}{Theorem}[section]
\newtheorem{definition}{Definition}[section]

% "Fat" table cells
\newcommand{\tlt}[2]{\begin{tabular}[c]{@{}c@{}}#1\\ #2\end{tabular}}

%%%%%%%%%%%%%%%%%%%%%%%%%%%%%%%%%%%%
%%% SPACING MACROS

\usepackage{slashed}

\usepackage{nextpage}
\newcommand\myclearpage{\cleartooddpage[\thispagestyle{empty}]}

\usepackage[fit]{truncate}
\fancyhf{}
\fancyhead[LE]{\truncate{.95\headwidth}{\leftmark}}
\fancyhead[RO]{\truncate{.95\headwidth}{\rightmark}}
\fancyfoot[C]{\thepage}

\makeatother

%%%%%%%%%%%%%%%%%%%%%%%%%%%%%%%%%%%%
%%% DOCUMENT START

\begin{document}

\pagenumbering{gobble}

%%%%%%%%%%%%%%%%%%%%%%%%%%%%%%%%%%%%
%%% AUTHOR'S NAME

\begin{center}
  Lucas Timotheo Sanches
  \par\end{center}

\vspace{2.5cm}

%%%%%%%%%%%%%%%%%%%%%%%%%%%%%%%%%%%%
%%% THESIS TITLE

\begin{center}
  {\huge{}Energy Extraction and Quasinormal Modes of Black Hole Binaries: An analytical and numerical study}{\huge\par}
  \par\end{center}

\begin{center}
  \vspace{2.5cm}
  \par\end{center}

%%%%%%%%%%%%%%%%%%%%%%%%%%%%%%%%%%%%
%%% THESIS DESCRIPTION

\begin{center}
  \begin{minipage}[t]{0.6\columnwidth}
    \begin{center}
      Thesis presented to the Post-Graduation program in Physics of
      Federal University of ABC as a requirement for obtaining the title
      of Doctor in Physics.
      \par\end{center}
  \end{minipage}
  \par\end{center}

%%%%%%%%%%%%%%%%%%%%%%%%%%%%%%%%%%%%
%%% ADVISORS

\begin{center}
  \vspace{2cm}
  Advisor: Prof. Dr. Maurício Richartz
  \par\end{center}

%%%%%%%%%%%%%%%%%%%%%%%%%%%%%%%%%%%%
%%% CITY

\begin{center}
  \vspace{2cm}
  Santo André - SP
  \par\end{center}

%%%%%%%%%%%%%%%%%%%%%%%%%%%%%%%%%%%%
%%% YEAR

\begin{center}
  2023
  \par\end{center}

\begin{center}
  {\large{}\thispagestyle{empty}}{\large\par}
  \par\end{center}

\begin{center}
  \pagebreak{}
  \par\end{center}

\vfill{}

%%%%%%%%%%%%%%%%%%%%%%%%%%%%%%%%%%%%
%%% LIBRARY STUFF

\begin{center}
  \fbox{\begin{minipage}[t]{0.8\columnwidth}%
      TIMOTHEO SANCHES, Lucas

      \hspace{1cm}Energy Extraction and Quasinormal Modes of Black Hole Binaries: An analytical and numerical study / Lucas Timotheo Sanches - Santo André,
      Universidade Federal do ABC, 2023.

      \vspace{0.5cm}

      \hspace{1cm}XX fls. XX cm\vspace{0.5cm}

      \hspace{1cm}\textportuguese{Orientador: Maurício Richartz}\vspace{0.5cm}

      \hspace{1cm}\textportuguese{Tese (doutorado) — Universidade}
      \textportuguese{Federal do ABC, Programa de Pós-Graduação em Física, 2023}\vspace{0.5cm}

      \hspace{1cm}
      \begin{portuguese}
        1. Palavra-chave. 2. Palavra-chave.  3. Palavra-chave.
        I. TIMOTHEO SANCHES, Lucas. II. Programa de Pós-Graduação em Física,
        2021. III. Título: subtítulo%
      \end{portuguese}
    \end{minipage}}
  \par\end{center}

\begin{center}
  {\large{}\thispagestyle{empty}}{\large\par}
  \par\end{center}

\begin{center}
  \pagebreak{}
  \par\end{center}

%%%%%%%%%%%%%%%%%%%%%%%%%%%%%%%%%%%%
%%% DEDICATION

\begin{minipage}[t]{0.9\columnwidth}%
  \begin{center}
    \textbf{\large{}\rule[0.5ex]{1\columnwidth}{1pt}}{\large\par}
    \par\end{center}
  \begin{doublespace}
    \textportuguese{DEDICATÓRIA}
  \end{doublespace}

  \rule[0.5ex]{1\columnwidth}{1pt}%
\end{minipage}
\begin{center}
  {\large{}\thispagestyle{empty}}{\large\par}
  \par\end{center}
\begin{portuguese}
  Dedico esse trabalho à memória da minha amada avó, Ignez de Sousa Timotheo, que faleceu durante a elaboração do mesmo e que durante toda a sua vida representou um exemplo inabalável de força moral e superação das piores adversidades às quais um ser humano pode ser submetido. O seu exemplo e as suas palavras estarão eternamente gravados na minha memória e no meu coração.

  Maior que o seu exemplo, era talvez somente o amor que tivera pelos filhos, e por mim, o seu único neto. Tal amor era refletido no orgulho que ela sentia em ter um neto que seria o primeiro da nossa família a tornar-se doutor. Por conta do curso natural da nossa breve existência, ela não poderá presenciar fisicamente a conclusão desse trabalho que ela vira se iniciar. Entretanto, acredito e sinto que de alguma forma, ela estará comigo, não só na minha defesa, mas indubitavelmente no meu coração e acompanhando os meus passos durante toda a minha vida.
\end{portuguese}

\begin{center}
  \myclearpage
  \par\end{center}

%%%%%%%%%%%%%%%%%%%%%%%%%%%%%%%%%%%%
%%% THANKS

\begin{minipage}[t]{0.9\columnwidth}%
  \begin{center}
    \textbf{\large{}\rule[0.5ex]{1\columnwidth}{1pt}}{\large\par}
    \par\end{center}
  \begin{doublespace}
    \textportuguese{AGRADECIMENTOS}
  \end{doublespace}

  \rule[0.5ex]{1\columnwidth}{1pt}%
\end{minipage}
\begin{center}
  {\large{}\thispagestyle{empty}}{\large\par}
  \par\end{center}

\begin{portuguese}[variant=brazilian]
  Nenhum texto pode refletir adequadamente o sentimento de gratidão que tenho pelos maiores responsáveis pelos sucessos da minha trajetória até este ponto: os meus pais, Luis e Tania. Durante toda a minha vida, os meus pais incentivaram e deram apoio ao meu instinto investigativo, ao meu interesse pela ciência e à minha paixão pelos estudos. Deram-me apoio financeiro, e principalmente, emocional e todo o seu amor nos momentos mais difíceis da minha caminhada. Sinto-me honrado de ser filho de pessoas tão maravilhosas e inigualáveis, exemplos em todos os sentidos. Amorosos. Carinhosos. Cuidadores. Pais. Espero retornar-lhes com o orgulho de ter um filho titulado doutor uma parte de todo o amor incondicional que eles têm por mim desde o meu nascimento, e demonstrar através deste trabalho uma pequena parte de toda a admiração e amor que sinto por eles.

  Agradeço, de forma igualmente carinhosa, ao restante dos meus familiares, cujas características que atribuí aos meus pais de apoio, amor e a admiração que sinto por eles também se aplicam: a minha avó Marisa, o meu avô Luis e às minhas tias, Sandra e Márcia.

  Gostaria de expressar o meu sincero agradecimento a uma pessoa muito especial, que tive a sorte de conhecer durante a elaboração desse trabalho: a minha amada companheira, Thalía. Agradeço-lhe por todo o apoio e incentivo que você me deu durante o processo de elaboração da minha tese. Sem a sua ajuda, eu não teria conseguido concluir este trabalho. A sua presença foi, e ainda é, fundamental para o meu sucesso. Eu não poderia ter pedido por uma companheira mais amorosa, solidária e compreensiva nesta jornada. Este é um momento importante para mim e você contribuiu enormemente para eu chegar até aqui. Obrigado por ser minha namorada e a minha inspiração. Espero que possamos continuar a apoiar um ao outro nos nossos futuros desafios e conquistas, e que o nosso amor se fortaleça cada vez mais com os muitos anos que teremos pela frente.

  A vida, sem bons amigos, seria um erro. É por isso que agradeço profundamente àqueles que decidiram fazer parte da minha vida, minha luta e minha trajetória por livre e espontânea vontade e aqui permanecem ao meu lado mesmo dadas as dificuldades da convivência do dia a dia. São eles: Bruna, Caio, Gabriel, Karen, Rafaela, Thiago e por fim, mas não menos importante, Iara, a quem além dos anos de amizade devo agradecimentos pela troca de discussões, ajuda, opiniões, apoio e suporte, tanto em assuntos acadêmicos quanto em assuntos pessoais que foram essenciais à elaboração desse trabalho e durante a minha carreira. Que possamos continuar sempre nos apoiando em todos os aspectos das nossas vidas.

  Agradeço também a meu orientador, Maurício, por toda a orientação e apoio que você me proporcionou ao longo deste tempo. As suas orientações foram essenciais para eu poder alcançar os resultados desejados no meu trabalho. As suas críticas construtivas, sugestões e opiniões foram sempre muito valiosos para o meu crescimento pessoal e profissional. Agradeço por ter-me dedicado o seu tempo, compartilhado o seu conhecimento e ajudado a desenvolver as minhas habilidades de pesquisa. As suas orientações foram além do aspecto profissional, sempre se mostrando preocupado com o meu bem-estar e com a minha jornada acadêmica. Foi uma grande honra ter você como o meu orientador e com certeza levarei as lições aprendidas no meu futuro.
\end{portuguese}

I would also like to express my heartfelt gratitude to Dr. Erik Schnetter, my international collaborator and mentor. Your expertise in the field of numerical relativity is unparalleled, and your dedication to your research is truly inspiring. I feel incredibly fortunate to have had the opportunity to work with you and learn from you. You have not only taught me the intricacies of numerical relativity but also provided me with the necessary guidance and support to succeed in our research endeavors. Your enthusiasm for our projects has been infectious, and I have been constantly motivated by your passion for advancing scientific knowledge. I hope to continue working with you for many years to come, building on the foundation we have established together.

\begin{center}
  {\large{}\thispagestyle{empty}}{\large\par}
  \par\end{center}

\newpage

\begin{center}
  {\large{}\thispagestyle{empty}}{\large\par}
  \par\end{center}

\newpage

%%%%%%%%%%%%%%%%%%%%%%%%%%%%%%%%%%%%
%%% CAPES

\begin{center}
  This study was financed in part by the Coordenação de Aperfeiçoamento de Pessoal de Nível Superior – Brasil (CAPES) – Finance Code 001
\end{center}

\begin{center}
  {\large{}\thispagestyle{empty}}{\large\par}
  \par\end{center}

\newpage

\begin{center}
  {\large{}\thispagestyle{empty}}{\large\par}
  \par\end{center}

\newpage

%%%%%%%%%%%%%%%%%%%%%%%%%%%%%%%%%%%%
%%% ABSTRACT 1

\begin{center}
  \textbf{\large{}\rule[0.5ex]{1\columnwidth}{1pt}}{\large\par}
  \par\end{center}

\begin{center}
  \textbf{\Large{}Resumo}{\Large\par}
  \par\end{center}

\vspace{0.5cm}

\begin{portuguese}[variant=brazilian]
  Esta tese explora fenômenos clássicos em múltiplos modelos de binários de buracos negros utilizando técnicas numéricas, analíticas e semi-analíticas. Nosso estudo fornece percepções sobre a extração de energia, modos quasinormais e o espalhamento ondulatório ao redor de binários de buracos negros, e demonstra como a presença de um binário altera estes fenômenos quando comparados às suas contrapartidas na presença de um único buraco negro. O estudo desenvolve técnicas inovadoras e pacotes numéricos, incluindo uma técnica para o estudo da extração de energia no contexto de espaços-tempos gerais, um código baseado no Método da Iteração Assintótica para o cálculo de frequências quasinormais de espaços-tempos de buracos negros e um novo \texttt{Thorn} para o \texttt{EinsteinTookit} para a evolução numérica de um campo escalar superimposto em uma métrica de evolução dinâmica. Estes resultados demonstram aumento da eficiência da extração de energia na presença de um objeto secundário e o cálculo bem-sucedido de novas frequências para perturbações de spin $5/2$ para o buraco negro de Schwarzschild. Descobertas preliminares do espalhamento de um campo escalar sem massa no binário do evento de colisão GW150914 demonstram percepções valiosas, mas estão sujeitas a mais testes e desenvolvimentos. Direções futuras de pesquisa incluem a incorporação da proposta de estudo do Processo de Penrose em espaços-tempos dinâmicos em simulações completamente dinâmicas, maior exploração da simulação numérica do campo escalar no topo do evento GW150914 e a inclusão de \textit{backreaction} da métrica para o aumento do realismo das simulações. As contribuições dessa tese aumentam a confiança na precisão e no potencial de métodos numéricos para a resolução de novos problemas em modelos de binários de buracos negros.
\end{portuguese}

\vspace{0.5cm}

%%%%%%%%%%%%%%%%%%%%%%%%%%%%%%%%%%%%
%%% KEYWORDS 1

\textbf{\textportuguese{Palavras-chave: Buracos Negros, Processo de Penrose, Extração de Energia, Modos Quasinormais, \texttt{EinsteinTookit}, Relatividade Numérica}}

\begin{center}
  {\large{}\thispagestyle{empty}}{\large\par}
  \par\end{center}

\newpage

\begin{center}
  {\large{}\thispagestyle{empty}}{\large\par}
  \par\end{center}

\newpage

%%%%%%%%%%%%%%%%%%%%%%%%%%%%%%%%%%%%
%%% ABSTRACT 2

\begin{center}
  \textbf{\large{}\rule[0.5ex]{1\columnwidth}{1pt}}{\large\par}
  \par\end{center}

\begin{center}
  \textbf{\Large{}Abstract}{\Large\par}
  \par\end{center}

\vspace{0.5cm}

This thesis explores classical phenomena in multiple models of black hole binaries using numerical, analytical, and semi-analytical techniques. Our study provides insights into energy extraction, quasinormal modes, and wave scattering around black hole binaries, and demonstrates how the presence of a binary alters these phenomena compared to their single black hole counterparts. The study develops innovative techniques and numerical packages, including a technique for studying energy extraction in the context of general spacetime metrics, an Asymptotic Iteration Method-based code for computing quasinormal frequencies of black hole spacetimes, and a new \texttt{EinsteinToolkit Thorn} for the numerical evolution of a scalar field superimposed on a dynamically evolving metric. The results demonstrate enhanced efficiency of energy extraction in the presence of a secondary object, and the successful computation of new quasinormal frequencies for spin $5/2$ perturbations of a Schwarzschild black hole. Preliminary findings of the scattering of a massless field over the GW150914 binary collision demonstrate valuable insights but are subject to further testing and development. Future research directions include incorporating the proposed approach for studying the Penrose process in dynamic spacetimes into fully dynamical simulations, further exploring the numerical simulation of the scalar field on top of the GW150914 binary, and considering the backreaction of the spacetime metric to enhance the realism of the simulations. The thesis's contributions enhance confidence in the accuracy and potential of numerical methods for addressing novel problems in binary black hole spacetime models.

\vspace{0.5cm}

%%%%%%%%%%%%%%%%%%%%%%%%%%%%%%%%%%%%
%%% KEYWORDS 2

\textbf{Keywords: Black Holes, Penrose Process, Energy Extraction, Quasinormal Modes, \texttt{EinsteinTookit}, Numerical Relativity}

\begin{center}
  {\large{}\thispagestyle{empty}}{\large\par}
  \par\end{center}

\newpage

\begin{center}
  {\large{}\thispagestyle{empty}}{\large\par}
  \par\end{center}

\newpage

%%%%%%%%%%%%%%%%%%%%%%%%%%%%%%%%%%%%
%%% LIST OF FIGURES

\thispagestyle{empty}
\listoffigures

\begin{center}
  {\large{}\thispagestyle{empty}}{\large\par}
  \par\end{center}

\newpage

\begin{center}
  {\large{}\thispagestyle{empty}}{\large\par}
  \par\end{center}

\newpage

%%%%%%%%%%%%%%%%%%%%%%%%%%%%%%%%%%%%
%%% LIST OF TABLES

\listoftables

\begin{center}
  {\large{}\thispagestyle{empty}}{\large\par}
  \par\end{center}

\newpage

\begin{center}
  {\large{}\thispagestyle{empty}}{\large\par}
  \par\end{center}

\newpage

%%%%%%%%%%%%%%%%%%%%%%%%%%%%%%%%%%%%
%%% LIST OF ACRONYMS

\chapter*{List of Acronyms}

\begin{acronym}
  \acro{GR}{General Relativity}
  \acro{BH}{Black Hole}
  \acro{BBH}{Binary Black Hole}
  \acro{PP}{Penrose Process}
  \acro{SKS}{Super Imposed Kerr Metric in Kerr-Schild Coordinates}
  \acro{NR}{Numerical Relativity}
  \acro{QNM}{Quasinormal Mode}
  \acro{ADM}{Arnowit, Deser and Misner}
  \acro{BSSN}{Baumgarte, Shapiro, Shibata and Nakamura}
  \acro{AMR}{Adaptive Mesh Refinement}
  \acro{CAS}{Computer Algebra System}
  \acro{AIM}{Asymptotic Iteration Method}
  \acro{API}{Application Programming Interface}
  \acro{ODE}{Ordinary Differential Equation}
  \acro{MP}{Majumdar and Papapetrou}
  \acro{CMMR}{Cabrera-Munguia, Manko and Ruiz}
\end{acronym}

\begin{center}
  {\large{}\thispagestyle{empty}}{\large\par}
  \par\end{center}

\newpage

\begin{center}
  {\large{}\thispagestyle{empty}}{\large\par}
  \par\end{center}

\newpage

%%%%%%%%%%%%%%%%%%%%%%%%%%%%%%%%%%%%
%%% TABLE OF CONTENTS

\tableofcontents{}

\begin{center}
  {\large{}\thispagestyle{empty}}{\large\par}
  \par\end{center}

\newpage

\begin{center}
  {\large{}\thispagestyle{empty}}{\large\par}
  \par\end{center}

\newpage

%%%%%%%%%%%%%%%%%%%%%%%%%%%%%%%%%%%%
%%% CHAPTERS

\pagenumbering{arabic}
\setcounter{page}{35}
\chapter{Introduction}
\label{ch:introduction}
\section{The relevance of binary systems}

The astrophysical black hole (\ac{BH}) is a vacuum solution of Einstein's field equations, and its mathematical description depends on two parameters: its mass and angular momentum~\cite{1986bhmp.book.....T}. Hence, \acp{BH} are one of the simplest objects in astrophysics from a mathematical standpoint.

Despite their mathematical simplicity, \acp{BH} exhibit rich physical interactions with their surroundings. These interactions can be classical, such as the formation of accretion disks around them~\cite{Abramowicz2013} or their interaction with other matter fields~\cite{Ficarra2023}, which generate effects like superradiance~\cite{PhysRevD.87.043513}. Alternatively, they can be semiclassical, such as Hawking radiation~\cite{Wald2001}. Recently, some of these phenomena, specifically superradiance and Hawking radiation, have been studied in the laboratory context using analogue models of gravity~\cite{Barcel2011} as reported in Refs.~\cite{Torres2017} and \cite{Kolobov2021}, respectively.

One of the most well-known predictions of the Theory of General Relativity (\ac{GR}) is the existence of spacetime oscillations that propagate throughout the universe, known as gravitational waves. While this may seem to be an orthogonal topic to that of astrophysical black holes, the two are closely related. In 1975, the binary system PSR B1913$+$16 provided the first (indirect) evidence of the existence of gravitational waves \cite{1975ApJ...195L..51H}. The observational data was consistent with theoretical analysis for neutron star binaries that emit gravitational waves.

Direct observations of gravitational waves, which had been attempted since the 1960s, only became a reality in 2015 with measurements from the Advanced Laser Interferometer Gravitational-wave Observatory (Advanced LIGO) \cite{grav1,grav2}, ushering in a new era in astronomy and cosmology. These observations also involve binary systems, in which two astronomical objects are close enough for their mutual gravitational attraction to cause them to orbit around a common center of mass. As they get closer and closer, a catastrophic collision event occurs, releasing huge amounts of energy and angular momentum in the form of gravitational radiation. Therefore, we can view gravitational waves as an interesting effect arising from the interaction of black holes with each other.

So far, no analytic description of a binary system in \ac{GR} is known. This is due to the extreme complexity of Einstein's equations. Even if such an exact solution were to be found, it would likely be too large, complicated and impractical to use. Given that in a collision event the nonlinear character of the equations becomes important the problem must be treated numerically, with the techniques of the field of Numerical Relativity (\ac{NR}).

One of the objectives of this thesis is to investigate interaction effects in General Relativity that are well understood in spacetimes containing a single astrophysical object (such as a star or a black hole) and extend them to binary systems.

\section{Modeling binary systems}

Our work employs different types of models to describe binary systems. Each of these models has different characteristics and nomenclature.  We will now define precisely the terms that will be used throughout the text to refer to different binary black hole models.

\begin{definition}[Static/Dynamic \ac{BBH} model]
  A \textbf{static} \ac{BBH} model represents two black holes that \textbf{do not move} with respect to the observers that will participate in the \ac{PP}. This is in contrast to a \textbf{dynamic} \ac{BBH} model.
\end{definition}

\begin{definition}[Exact/Approximate \ac{BBH} model]
  An \textbf{exact} \ac{BBH} model is one that is an exact vacuum solution of Einstein's field equations. An \textbf{approximate} \ac{BBH} model is one that is not an exact vacuum solution of said equations, and the deviation from vacuum is not considered to be an exotic type of matter but a measure of its ``non-exactness''.
\end{definition}

\begin{definition}[Analitic/Numeric \ac{BBH} model]
  An \textbf{analytic} \ac{BBH} model is one whose entire spacetime metric is analytically known at all points and times. In contrast, a \textbf{numeric} \ac{BBH} model is obtained only at a certain coordinate time hypersurface (typically $t=0$) by solving the Arnowit-Deser-Misner (\ac{ADM}) constraint equations for a vacuum configuration numerically. Note that even tough such numerical solutions contain errors, and thus do not strictly solve the vacuum field equations (giving rise to constraint violations), we do not consider these to be approximate, since it is possible (at least in theory but not in practice due to physical limitation of machines and computational resources) to obtain these solutions with arbitrarily small constraint violations given that they are convergent. For more information on solving the \ac{ADM} constraint equations and obtaining initial data in Numerical Relativity, see for instance Chapter 3 of Ref~\cite{Alcubierre2012-xp}
\end{definition}

\section{The Penrose process}

One instance of a black hole \ac{BH} interaction with its surrounding that will be extensively discussed is the Penrose Process (\ac{PP}). The first incarnation of the \ac{PP} arose as a consequence of the Kerr metric. The Kerr metric is the best-known mathematical description of rotating black holes given by the theory of General Relativity~\cite{Visser:2007fj,Bambi:2011mj,Teukolsky:2014vca,berti}.

Unlike static black holes, a Kerr black hole is characterized by the existence of a very peculiar region around its event horizon, known as the \emph{ergosphere} or \emph{ergoregion}. Particles that reach the ergosphere can still avoid the event horizon and, hence, are not doomed to end at the spacetime singularity. Nevertheless, any observer lying inside the ergosphere is unavoidably dragged along by the rotational motion of the black hole.

Particles inside the ergosphere may have negative energies (according to a static observer at infinity). Relying on this property, Penrose and Floyd devised a mechanism that allows one to extract energy from a rotating black hole~\cite{PENROSE1971}. The idea consists of sending a particle from infinity towards the black hole and assuming that, once inside the ergosphere, it decays into two other particles. If one of the fragments is counter-rotating with the black hole and has negative energy (which implies that the split happens inside the ergosphere), it will be captured by the black hole, meaning that the other fragment will escape to infinity. Due to the conservation of the four-momentum, the escaping fragment will have more energy and more angular momentum than the incident particle. Rotational energy and angular momentum are, thus, effectively extracted from the black hole.

The main motivating factor for this investigation was that despite being a well-known process, the \ac{PP} had not been studied in the context of black hole binaries. Moreover, many recent research endeavors attempt to establish a relation of the \ac{PP} with observable astrophysical phenomena.

The collisional version of the process, for instance, considers multiple particles that collide and scatter in the ergoregion, allowing for arbitrarily high center-of-mass energies. This process can potentially act as a mechanism to eliminate dark matter particles near a supermassive black hole~\cite{Schnittman:2018ccg}.

The magnetic \ac{PP}~\cite{Wagh1985,Tursunov:2019oiq}, on the other hand, considers charged particles and black holes surrounded by magnetic fields (originated, for instance, by plasma accretion disks around the black hole). The electromagnetic interaction allows for highly efficient energy extraction schemes, such as the one introduced in Ref.\cite{Tursunov:2020juz} to model the emission of ultra-high energy cosmic rays from supermassive black holes. Furthermore, recent numerical simulations of plasma and jets around Kerr black holes indicate the important role that negative energy particles and the \ac{PP} play on the total energy flux coming from the black hole's jets\cite{Parfrey:2018dnc}.

Recognizing these already discussed difficulties of modeling astrophysical binary systems, we have utilized \ac{BBH} models that are static, exact and analytic in this investigation.

This discussion can be found in Chapter \ref{ch:penrose_binaries}, where we discuss energy extraction via the Penrose mechanism of different binary black hole models. The contents of Sec.~\ref{ch:penrose_binaries:sec:mp_penrose} and \ref{ch:penrose_binaries:sec:cmmr_penrose}, regarding non-coalescing binaries, have been published in Ref.~\cite{PhysRevD.104.124025}. The contents of Sec.~\ref{ch:penrose_binaries:sec:arbitrary_penrose} extend the discussion to non-static binaries and present novel preliminary results that we intend to publish soon.

\section{Black Hole Perturbations and Quasinormal Modes - Wave Scattering}

When a closed physical system (like a guitar string) is perturbed, it relaxes by emitting certain natural frequencies know as \emph{normal modes}. If, however, the system is open (and therefore energy is being somehow dissipated away), its emitted natural frequencies will decay with time (like for instance sounding a bell in a church). These decaying modes are called \emph{quasinormal modes} (\acp{QNM}).

Quasinormal frequencies can be used to obtain information about the system that produces them and black holes, like church bells, are also subject to these phenomena: Perturbed black holes relax by emitting waves in characteristic frequencies that decay with time, thanks to the dissipative nature of the event horizon. See Refs.~\cite{review1, review2, review3, review4} for an in-depth review of the quasinormal mode problem in the context of general relativity and black holes.

Determining these characteristic frequencies quickly and accurately for a large range of models is important for many practical reasons. It has been shown that the gravitational wave signal emitted at the final stage of the coalescence of two compact objects is well described by quasinormal modes~\cite{buonanno,seidel} (note that recent developments during the elaboration of this thesis have called this fact into question. See~\ref{PhysRevLett.130.081401} for details). This means that if one has access to a database of quasinormal modes and of gravitational wave signals from astrophysical collision events, it is possible to characterize the remnant object using its quasinormal frequencies. Since there are many models that aim to describe remnants, being able to compute the quasinormal frequencies for such models reliably is paramount for confirming or discarding them.

In the context of binary systems, in Chapter~\ref{ch:wave_scattering} we will focus on the numerical simulation of a scalar wave scattering off a black hole binary that mirrors the first event detected by LIGO, know as GW150914. In our simulations, the spacetime metric will not ``back-react'' to the presence of the field. That means that the scalar wave shall propagate on top of the evolving black hole binary without interfering in the stress-energy tensor of the spacetime. An analysis of the scattered signal will be made in an attempt to identify quasinormal ringing of the binary system.

In recent years, significant progress has been made in the numerical simulation of black hole binaries thanks to state-of-the-art strongly hyperbolic reformulations of the field equations, such as the \ac{BSS} formulation~\cite{PhysRevD.52.5428,PhysRevD.59.024007} and advancements in the understanding of stability criteria in partial differential equation discretization schemes employing techniques such as summation by parts~\cite{Diener2007}. These techniques have enabled us to simulate the dynamics of black hole binaries and their gravitational wave emission with unprecedented accuracy, stability and reliability.

Numerical evolution through self-coding can be a daunting task. This is where the employment of pre-existing tools, such as the \texttt{EinsteinToolkit}~\cite{EinsteinToolkit:2022_11}, becomes beneficial. The \texttt{EinsteinToolkit} is, according to its home page, ``a community-driven software platform of core computational tools to advance and support research in relativistic astrophysics and gravitational physics.''. The \texttt{EinsteinToolkit} is composed of a variety of software modules, referred to as \texttt{Thorns}, that are integrated into the \texttt{Cactus}~\cite{Goodale:2002a} computational framework. Each \texttt{Thorn} has the ability to perform a range of tasks, including infrastructure tasks such as file input/output, adaptive mesh refinement (\ac{AMR}), as well as physics-related tasks such as the evolution of binary systems, the extraction of gravitational waves, and the multipole decomposition of a signal.

The outcomes reported in this chapter are preliminary and exploratory in nature. They do not possess the necessary level of maturity for publication. However, we have deemed it appropriate to incorporate them in this thesis as they demonstrate crucial skills that were acquired during the course of this research. Additionally, they can serve as a foundation for more advanced analyses in the future.

It should be noted that while we were drafting this chapter, another PhD thesis authored by G. Ficarra was made publicly available and can be found in Ref.~\cite{Ficarra2023}. In his thesis, Ficarra has carried out similar work with significantly more detail and precision, as his findings are not preliminary and his thesis is wholly dedicated to that topic. It is not our intention to compete with his work. Rather, we aim to highlight the contributions we have made and the skills gained in doing so. If one is interested in sophisticated analysis of binary systems interacting with scalar fields, including instances where the gravitational potential is influenced by the scalar cloud, we recommend reading his thesis.

\section{Black Hole Perturbations and Quasinormal Modes - \texttt{QuasinormalModes.jl}}

Frequently, the task of computing quasinormal modes of a given system can be reduced to determining the discrete set of eigenvalues of a second-order differential equation subject to appropriate boundary conditions and asymptotic behavior~\cite{review3}. Several techniques can be employed to address such problems, but Leaver's continued fraction method~\cite{Leaver1985} stands out as the most widely used and successful method for a broad range of spacetimes.

Nevertheless, Leaver's method may not always yield satisfactory results, such as in the case of determining the quasinormal modes of an extreme Reisnner-N"ordstrom black hole, where additional treatments are required~\cite{PhysRevD.93.064062}. Therefore, Leaver's method cannot be regarded as universally applicable and may require manual adjustments for specific cases.

Similarly, the pseudospectral method, another popular approach to solving eigenvalue problems in ordinary differential equations, may encounter difficulties. This method constructs matrices from the original \ac{ODE}, and if the sought eigenvalue is not a polynomial function, manual adaptations must be performed, rendering the method non-general and specific to certain problems. While this is not prohibitively restrictive to most astrophysics problems, it can be an important limitation in other areas

There is also a lack of generalists, free (both in the financial and license-wise sense) open source tools that compute quasinormal modes generally. More precisely, there are tools which are free and open source, but run on top of a proprietary paid and expensive software framework such as the ones developed in Refs.~\cite{qnmspectral,spectralbp}, which are both excellent packages that can be obtained and modified freely but, unfortunately, require the user to own a license of the proprietary \texttt{Wolfram Mathematica} \ac{CAS}. There are also packages that are free and run on top of \texttt{Mathematica} but are not aimed at being general eigenvalue solvers at all, such as the one in Ref.~\cite{bhpt_quasinormalmodes}, that can only compute modes of Schwarzschild and Kerr black holes. Finally, the \texttt{Python} package in Ref.~\cite{bhpt_qnm} is open source and free but can only compute Kerr quasinormal modes.

Fortunately, a new numerical technique was recently developed for tackling these problems, known as the \emph{Asymptotic Iteration Method} (\ac{AIM}). The groundwork of the technique was laid out in Ref.~\cite{aim_original} and in Ref.~\cite{aim_improved} the method was refined and adapted to \ac{GR}. Given its generality, the method is directly applicable to cases where other methods require greater care.

We have implemented and published \texttt{QuasinormalModes.jl} (see the accompanying paper in Ref.~\cite{Sanches2022}), a \texttt{Julia}~\cite{Bezanson2017} software package for finding quasinormal modes using the \ac{AIM}. Not only that, the package can be used to compute the discrete eigenvalues of \emph{any} second order homogeneous \ac{ODE} (such as the energy eigenstates of the time independent Schrödinger equation or spin weighted spheroidal harmonics, defined in Ref.~\cite{PhysRevD.73.024013}) provided that these eigenvalues actually exist.

\texttt{QuasinormalModes.jl} fills the existing gap for free, open source tools that are able to compute discrete eigenvalues (and in particular, quasinormal modes) efficiently for a broad class of models and problems. The package features a flexible and user-friendly \ac{API} where the user simply needs to provide the coefficients of the problem \ac{ODE} after incorporating boundary and asymptotic conditions on it.

The user can also choose to use machine or arbitrary precision arithmetic for the underlying floating point operations involved and whether to do computations sequentially or in parallel using threads. The \ac{API} also tries not to force any particular workflow on the users so that they can incorporate and adapt the existing functionality on their research pipelines without unwanted intrusions.

Often user-friendliness, flexibility and performance are treated as mutually exclusive, particularly in scientific applications. By using \texttt{Julia} as an implementation language, the package can have all these features simultaneously.

The package was used in Ref.~\cite{Mamani2022} where perturbations of various spins of the Schwarzschild metric were considered. Novel frequencies were obtained and results that were previously available were compared against literature values, while also cross-checking results for the same models obtained via the more traditional pseudo-spectral method.

In chapter~\ref{ch:qnm_aim}, we describe the \ac{AIM} in detail, it's implementation in \texttt{QuasinormalModes.jl} and the perturbation equations and results obtained in Ref.~\cite{Mamani2022} using the package.

\myclearpage
\par

\chapter{Energy Extraction From Black Hole Binaries}
\label{ch:penrose_binaries}
In this chapter, we will explore energy extraction from black hole binaries via the Penrose Process (PP) in black hole binaries.

Our first task was to investigate the possibility of energy extraction from the Majumdar-Papapetrou (MP) metric~\cite{MAJUMDAR1947,PAPAPETROU1947}, which is an exact solution of Einstein's equations that describes a static binary of extremally charged black holes. Despite its mathematical simplicity, the MP solution has recently been used as a surrogate model for black hole binaries in order to understand how single black hole phenomena transpose to a binary system. For instance, Ref.~\cite{ASSUMPCAO2018} has employed the MP metric to understand the connection between quasinormal modes and light rings in the context of black hole binaries. Refs.~\cite{Shipley:2016omi,Shipley:2019kfq}, on the other hand, have computed the shadows cast by an MP binary to better understand chaotic scattering in a binary black hole system. The resulting shadow shares many similarities and qualitative features with the shadows computed in Ref.~\cite{Bohn:2014xxa} using a numerically simulated binary black hole background. Ref.~\cite{BINI2019} has also applied the MP metric to analyze particle scattering around a black hole binary and asserted its effectiveness in approximating a head on collision in the limit of large separations and small approach speeds. We followed an analog approach in order to gain physical intuition and qualitative insights about energy extraction from black hole binaries by the Penrose mechanism. In particular, we extended the concept of a particle dependent \emph{generalized ergosphere}~\cite{RUFFINI1971}, which enables the extraction of electromagnetic energy from Reissner-N\"ordstrom (RN) black holes, to the MP solution and study how the energy extraction efficiency is affected by the presence of a companion black hole.

Taking into account the fact that, in astrophysical contexts, any excess of electric charge in a black hole tends to be quickly neutralized~\cite{gibbons1975}, we also considered rotating systems in our work. More specifically, in order to illustrate how the main results for the MP spacetime can be extrapolated to a binary system composed of Kerr black holes, we employ a static exact and analytic solution of Einstein's field equations, discovered independently by Cabrera-Munguia, Manko and Ruiz (hereby referred to as the CMMR metric)~\cite{cabrera_metric,manko_ruiz_metric, manko_ruiz_thermo}. The CMMR metric describes two generic Kerr black holes that do not coalesce thanks to the presence of a ``strut'' that holds them apart at a fixed distance. In particular, we sketched the ergosphere of the CMMR spacetime for a selected set of parameters and gave an example of a PP around a binary of rotating black holes.

The combined results presented in Sections \ref{ch:penrose_binaries:sec:mp_penrose} and \ref{ch:penrose_binaries:sec:cmmr_penrose} have been compiled in a paper and were published in Ref.~\cite{PhysRevD.104.124025}. As an extension to this work, we have begun the development of a framework that should allow one to observe the PP for an arbitrary spacetime and shall present these preliminary results in Sec. \ref{ch:penrose_binaries:sec:arbitrary_penrose}. We illustrate the viability of our proposition by utilizing a non-exact solution to Einstein's field equations obtained by superimposing two Kerr metric solutions in Kerr-Schild coordinates (thus called the SKS - Superimposed Kerr-Schild Kerr solution), that models an orbiting binary of rotating black holes close to merger.

\section{A brief review of the Penrose process in the Kerr spacetime}
\label{ch:penrose_binaries:sec:penrose_review}
Our review will follow closely Ref.~\cite{carroll}. We start by reminding that the Kerr metric, in Boyer-Lindquist coordinates $(t, r, \theta, \phi)$ is given by

\begin{multline}
  \ud s^2 = -\left(1 - \frac{2 M r}{\rho^2}\right)\ud t^2 - \frac{2 M a r \sin^2\theta}{\rho^2}\left( \ud t \ud \phi + \ud \phi \ud t \right)\\
  + \frac{\rho^2}{\Delta} \ud r^2 + \rho^2 \ud\theta^2 + \frac{\sin^2\theta}{\rho^2}\left[ (r^2 + a^2)^2 - a^2\Delta\sin^2\theta \right]\ud\phi^2,
  \label{eq:kerr_penrose_review_kerr_metric}
\end{multline}
%
where
%
\begin{equation}
  \Delta(r) = r^2 - 2Mr + a^2
  \label{eq:kerr_penrose_review_kerr_delta}
\end{equation}
%
and
%
\begin{equation}
  \rho^2(r, \theta) = r^2 + a^2\cos^2\theta.
  \label{eq:kerr_penrose_review_kerr_rho}
\end{equation}

The constants $M$ and $a$ represent, respectively the black hole's mass and spin parameter (angular momentum per unit mass). The metric possesses two event horizons, located at
%
\begin{equation}
  r_{H\pm} = M \pm \sqrt{M^2 - a^2}
  \label{eq:kerr_penrose_review_kerr_horizons}
\end{equation}
%
and since its components are independent of both the coordinate time $t$ and the axial angular variable $\phi$, there are global Killing vector fields $K = \partial_t$ and $R = \partial_\phi$ that generate these symmetries. Since the metric is stationary, the region where the time-like global Killing vector field changes its sign and thus static observers become prohibited does not coincide with the event horizons. In fact, one can easily see that
%
\begin{equation}
  K^\mu K_\mu = -\frac{1}{\rho^2}\left({\Delta - a^2 \sin^2\theta}\right) = -\frac{a^2 + r(r-2M)-a^2\sin^2\theta}{(r^2+a^2\cos^2\theta)^2}.
  \label{eq:kerr_penrose_review_kerr_killing_horizon_equation}
\end{equation}
%
and thus $K^\mu K_\mu = 0$ implies that the killing horizons are located at
%
\begin{equation}
  r_{K\pm} = M \pm \sqrt{M^2 - a^2\cos^2\theta}.
  \label{eq:kerr_penrose_review_kerr_killing_horizon_solution}
\end{equation}
%
which means that $r_{K+}$ is outside $r_{H+}$, coinciding with it only at the poles ($\theta=0$ or $\theta=\pi$). Notice also that using $r_{H+}$ in Eq.~\eqref{eq:kerr_penrose_review_kerr_killing_horizon_equation} yields $\Delta=0$ and thus
%
\begin{equation}
  K^\mu K_\mu = \frac{a^2}{\rho^2}\sin^2\theta \geq 0.
  \label{eq:kerr_penrose_review_kerr_killing_horizon_solution}
\end{equation}
%
The region that lies in-between $r_{K+}$ and $r_{H+}$ is called the \emph{ergosphere} or \emph{ergoregion}. The fact that this region is outside the event horizons and that $K^\mu K_\mu>0$ in it is paramount to the Penrose mechanism, as we shall see further on. A schematic representation of important structures in the Kerr spacetime can be found in Fig. \ref{fig:kerr_penrose_review_kerr_surfaces}, where the ergosphere is shaded in gray.

\begin{figure}[!htbp]
  \centering
  \includesvg[scale = 1.0]{img/penrose_binaries/kerr_horizons_and_ergosphere.svg}
  \caption{Schematic representation of important boundaries and regions in a Kerr black hole. The gray region represents the BH's ergosphere. The black hole's rotational axis goes across the figure, from bottom to top.}
  \label{fig:kerr_penrose_review_kerr_surfaces}
\end{figure}

Let us now consider a particle of mass $m$ and 4-momentum $p^\mu$ moving along a time-like geodesic in the Kerr spacetime parametrized by its proper time $\tau$. The energy per unit mass of the particle along its trajectory, as measured by a static observer infinitely far away from the black hole, is given by
%
\begin{equation}
  E = -K_\mu p^\mu/m.
  \label{eq:kerr_penrose_review_kerr_enrgy_per_unit_mass_full}
\end{equation}
%
Outside the ergosphere, $K^\mu$ is time-like and $K_\mu p^\mu/<0$. Since we would like the energy to be positive infinitely far away from the BH, we must introduce a leading minus sign in Eq.~\eqref{eq:kerr_penrose_review_kerr_enrgy_per_unit_mass_full}. On the other hand, inside the ergosphere $K^\mu$ is space-like and $K_\mu p^\mu/ >0$ which implies that in this region $E < 0$.

Let us now imagine that a particle labeled as $(0)$ traveling in the Kerr spacetime comes from infinity with 4-momentum $p^{(0)\mu}$ and decays \emph{inside} the ergosphere in a break-up point $b$ into two other particles, the first of which labeled $(1)$ has 4-momentum $p^{(1)\mu}$ and gets absorbed by the black hole and the second of which labeled $(2)$, 4-momentum $p^{(2)\mu}$ and escapes the gravitational pull of the system and returns to infinity. The law of conservation of 4-momentum applied at point $b$ implies that
%
\begin{equation}
  p^{(0)\mu} = p^{(1)\mu} + p^{(2)\mu}.
  \label{eq:kerr_penrose_review_kerr_4_momentum_conservation}
\end{equation}
%
Contracting Eq.~\eqref{eq:kerr_penrose_review_kerr_4_momentum_conservation} with $K^\mu$ we get
\begin{equation}
  E^{(0)} = E^{(1)} + E^{(2)}.
  \label{eq:kerr_penrose_review_kerr_energy_conservation}
\end{equation}
%
If we engineer the trajectory of particle $(1)$ such that $E^{(1)} < 0$, we get
\begin{equation}
  E^{(2)} = E^{(0)} + E^{(1)} > E^{(0)},
  \label{eq:kerr_penrose_review_kerr_energy_increase}
\end{equation}
%
which means that the energy of the returning particle is greater than the energy of the original. This mechanism is precisely the one proposed by Penrose and Floyd, and has since been known simple as the \emph{Penrose Process} or \emph{Penrose Mechanism}. This break-up process is represented in Fig.~\ref{fig:kerr_penrose_review_kerr_breakup}.

\begin{figure}[!htbp]
  \centering
  \includesvg[scale = 1.0]{img/penrose_binaries/kerr_breakup.svg}
  \caption{Schematic representation of a Penrose process. A particle of energy $E^{(0)}$ (red) comes in from infinity and decays at point $b$ inside the ergosphere in a negative energy $E^{(1)}$ trajectory (black) and a positive energy $E^{(2)}>E^{(0)}$ trajectory (blue) that returns to infinity. The black hole's rotational axis is pointing outside the page, towards the reader.}
  \label{fig:kerr_penrose_review_kerr_breakup}
\end{figure}

To do:
\begin{itemize}
  \item Discuss where the energy comes from: Carroll + Reversib1e and Irreversible Transformations in Black-Hole physics
  \item Show irreducible mass and max energy that can be extracted
  \item Discuss that the negative energy particle must be confined in the ergosphere
  \item Discuss that the local energy is still positive
\end{itemize}

\section{Majumdar-Papapetrou Spacetime}
\label{ch:penrose_binaries:sec:mp_penrose}
In this section we will show that energy can be extracted from the MP spacetime using the Penrose process. Since the MP spacetime consists of a binary of Reissner-N\"ordstrom constituents, we will use an electrically charged particle to extract electromagnetic energy from the system.

\subsection{Spacetime metric}

For two black holes of masses $M_1$ and $M_2$ and electric charges $Q_1 = M_1$ and $Q_2 = M_2$, in equilibrium and separated by a distance $2a$ along the $z$-axis, the MP line element, in Weyl's cylindrical coordinates, is given by~\cite{SMERAK2016}

\begin{equation}
    \ud s^2 = -U(\rho,z)^{-2} \ud t^2 + U(\rho,z)^2\left[\ud\rho^2 + \rho^2\ud\phi^2 + \ud z^2\right],
    \label{ch:penrose_binaries/ch:penrose_binaries/eq:majumdar_papapetrou_line_element}
\end{equation}
%
where
\begin{equation}
    U(\rho,z) = 1 + \frac{M_1}{\sqrt{\rho^2 + (z+a)^2}} + \frac{M_2}{\sqrt{\rho^2 + (z-a)^2}}.
    \label{ch:penrose_binaries/eq:mp_metric_potential_cylindric}
\end{equation}

The components of the electromagnetic potential $\tens{A}{}{\mu}$ associated with the MP solution are

\begin{equation}
    \tens{A}{}{\mu} = \left(1 - \frac{1}{U}\right)\tens{\delta}{}{\mu t},
    \label{ch:penrose_binaries/eq:electromagnetic_potential_mp}
\end{equation}
%
where $\tens{\delta}{}{\mu\nu}$ is the Kronecker delta.

We remark that Weyl's coordinates describe only the exterior of the black holes, in such a way that their event horizons and their interiors are collapsed into the points $\rho=0, \, z=\pm a$. In particular, due to its staticity, the MP spacetime does not possess an ergosphere (in the usual sense of a spacetime region outside the event horizon where it is impossible for freely-falling particles to ``stand still" -- i.e remain static -- while still being able to escape to infinity). In view of that, and motivated by the possibility of Penrose processes for single charged black holes, instead of analysing geodesic motion we will start by studying the trajectories of charged particles in the MP spacetime.

\subsection{Motion of charged particles}

The motion of a massive charged test particle (with charge-to-mass ratio $\mu$) in any given spacetime with metric $g_{\mu \nu}$ and electromagnetic potential $\tens{A}{}{\mu}$ is governed by the Lagrangian
\begin{equation}
    \mathcal{L} = \frac{1}{2}\mtrtens{\mu}{\nu}\tens{\dot{x}}{\mu}{}\tens{\dot{x}}{\nu}{} - \mu \tens{A}{}{\alpha}\tens{\dot{x}}{\alpha}{},
    \label{ch:penrose_binaries/eq:lagrangian_for_charged_particle}
\end{equation}
%
where $\tens{x}{\mu}{} = \tens{x}{\mu}{}(\lambda) =  \left(t(\lambda), \rho(\lambda), \phi(\lambda), z(\lambda)\right)$ denotes the particle's position (parametrized by its proper time $\lambda$) and dots represent derivatives with respect to $\lambda$. Taking into account the explicit form of the MP metric and of the MP electromagnetic potential, the Lagrangian \eqref{ch:penrose_binaries/eq:lagrangian_for_charged_particle} can be recast as~\cite{RYZNER2015}

\begin{equation}
    \mathcal{L} = \frac{1}{2}\left[-\frac{\dot{t}^2}{U^2} + U^2\left( \dot{\rho}^2 + \rho^2\dot{\phi}^2 + \dot{z}^2 \right) \right] - \mu\left(1-\frac{1}{U}\right)\dot{t}.
    \label{ch:penrose_binaries/eq:explicit_lagrangian_for_charged_particle}
\end{equation}
%
Since the expression above does not depend explicitly on the coordinates $t$ and $\phi$, two constants of motion can be identified, namely the energy divided by the mass of the particle (as measured by a freely-falling observer at infinity),

\begin{equation}
    E \equiv -\del{\mathcal{L}}{\dot{t}} = \frac{\dot{t}}{U^2} + \mu\left(1-\frac{1}{U}\right),
    \label{ch:penrose_binaries/eq:conserved_energy}
\end{equation}
%
and the angular momentum about the $z$ axis divided by the mass of the particle (as measured by a freely-falling observer at infinity),

\begin{equation}
    L \equiv \del{\mathcal{L}}{\dot{\phi}} = U^2 \rho^2 \dot{\phi}.
    \label{ch:penrose_binaries/eq:conserved_momentum}
\end{equation}

Plugging these constants of motion into the 4-velocity normalization condition, $\tens{\dot{x}}{\mu}{}\tens{\dot{x}}{}{\mu} = -1$, and solving for the energy $E$, yields

\begin{equation}
    E = \mu\left(1-\frac{1}{U}\right) + \sqrt{\frac{L^2}{\rho^2U^4} + \frac{1}{U^2} + \dot{\rho}^2 + \dot{z}^2}
    \label{ch:penrose_binaries/eq:alternative_expression_for_energy},
\end{equation}
where the negative root has been neglected due to the fact that $E$ must be positive at infinity when $\mu = 0$ (for a detailed discussion about positive and negative root states for $E$ see Ref. \cite{RUFFINI1971}). We note that Eq.~\eqref{ch:penrose_binaries/eq:alternative_expression_for_energy} reduces to the expression found in Ref.~\cite{DENARDO1973} for a Reissner-Norstr{\"o}m black hole if the mass of one of the black holes is taken to be zero and an appropriate coordinate system, centered around the other black hole, is used.

Finally, the equations of motion are obtained directly from the Euler-Lagrange equations. After using \myref{ch:penrose_binaries/eq:conserved_energy} and \myref{ch:penrose_binaries/eq:conserved_momentum} to eliminate $\dot{t}$ and $\dot{\phi}$ from the resulting expressions, one is left with

\begin{multline}
    \ddot{\rho}-\frac{L^2(U+\rho\indexdel{\rho}U)}{\rho^3 U^5} + \frac{2\dot{\rho}\dot{z}\indexdel{z}U - (E^2 + \dot{z}^2 - \dot{\rho}^2)\indexdel{\rho}U}{U} \\- \frac{\mu}{U}\left(\mu-2E+\frac{E-\mu}{U}\right)\indexdel{\rho}U = 0,
    \label{ch:penrose_binaries/eq:ode_for_rho_motion}
\end{multline}
%
and
\begin{multline}
    \ddot{z}-\frac{L^2\indexdel{z}U}{\rho^2 U^5} + \frac{2\dot{\rho}\dot{z}\indexdel{\rho}U - (E^2 - \dot{z}^2 + \dot{\rho}^2)\indexdel{z}U}{U} \\- \frac{\mu}{U}\left(\mu-2E+\frac{E-\mu}{U}\right)\indexdel{z}U = 0,
    \label{ch:penrose_binaries/eq:ode_for_z_motion}
\end{multline}
which reduce to the equations of motion found in Ref.~\cite{ASSUMPCAO2018} for neutral particles. Together with Eqs.~\eqref{ch:penrose_binaries/eq:conserved_energy} and \eqref{ch:penrose_binaries/eq:conserved_momentum}, the two equations above fully determine the trajectory of a massive charged particle in the MP spacetime once appropriate initial conditions have been specified.

\subsection{Generalized ergosphere}

The possibility of extracting energy from a black hole through the Penrose mechanism is intimately linked to the existence of negative-energy trajectories (as measured by observers far away from the black hole). As explained before, in the MP spacetime (and in other static spacetimes like Reissner-Nordstrom) there are no ergospheres in the usual sense, meaning that geodesic motion is always associated with positive energy trajectories. Any charged particle moving through the MP spacetime, however, will be affected by Lorentz forces from its electromagnetic interaction with the black holes and, consequently, will satisfy the equations of motion derived in the previous section. Even though there are no ergospheres in MP, since the trajectories associated with Eqs.~\eqref{ch:penrose_binaries/eq:conserved_energy}, \eqref{ch:penrose_binaries/eq:conserved_momentum}, \eqref{ch:penrose_binaries/eq:ode_for_rho_motion} and \eqref{ch:penrose_binaries/eq:ode_for_z_motion} are not geodesics, negative values for the constant of motion $E$ are not forbidden \emph{a priori}. If negative energy trajectories exist in the MP spacetime, then one can define generalized ergospheres as the regions of spacetime where they are allowed to exist. The concept of generalized ergospheres is not new, however, and was first introduced by Denardo and Ruffini for the Reissner - Nordstr\"om solution in Ref.~\cite{DENARDO1973}, where they have additionally shown that the Penrose process is viable in this spacetime by presenting concrete examples. That being said, defining a generalized ergosphere for the MP spacetime is a natural step forward from D. and R. since it consists of two extremely charged RN black holes.

From Eq.~\eqref{ch:penrose_binaries/eq:alternative_expression_for_energy} it is evident that, for fixed $\mu$, the energy is completely determined by the angular momentum $L$, and the values of $\rho$, $z$, $\dot{\rho}$ and $\dot{z}$  at any given instant of time. Therefore, at a fixed position in space, the minimum possible energy is associated with particles at rest. Letting $\dot{\rho}=\dot{z}=0$ and $L=0$ (i.e.~$\dot{\phi}=0$), we conclude that the minimum energy $E_{\mathrm{min}}$ for a particle, as a function of its position, is
\begin{equation} \label{ch:penrose_binaries/eq:minimum_energy}
    E_{\mathrm{min}} = \mu\left(1-\frac{1}{U}\right) + \frac{1}{U}.
\end{equation}

By requiring the minimum energy to be negative, we find that $\mu$ must necessarily be negative (which means that the particle's charge is opposite to the charge of the black holes) and that the particle (at rest) must be inside the spatial region determined by
\begin{equation} \label{ch:penrose_binaries/eq:negative_energy_condition}
    \frac{M_1}{\sqrt{\rho^2 + (z+a)^2}} + \frac{M_2}{\sqrt{\rho^2 + (z-a)^2}} > - \frac{1}{\mu}.
\end{equation}

Since $\mu$ is a dimensionless quantity, it makes sense to recast the left side of \eqref{ch:penrose_binaries/eq:negative_energy_condition} as a dimensionless quantity as well so that we are comparing two pure numbers. To do so, one must simply divide both numerators and denominators of the left side of the inequality by, say, $M_1$ and define $\overline{\rho} \equiv \rho/M1$, $\overline{z} \equiv z/M1$, $M \equiv M_1/M_2$, $\alpha = a/M1$ to obtain

\begin{equation} \label{ch:penrose_binaries/eq:rescaled_negative_energy_condition}
    \frac{1}{\sqrt{\overline{\rho}^2 + (\overline{z}+\alpha)^2}} + \frac{M^{-1}}{\sqrt{\overline{\rho}^2 + (\overline{z}-\alpha)^2}} > - \frac{1}{\mu}.
\end{equation}

This last inequality determines the generalized ergosphere of the MP spacetime as a region depending only on three dimensionless parameters: $M$, $\alpha$ and $\mu$. Particles with charge-to-mass ratio $\mu$ inside the region determined by \eqref{ch:penrose_binaries/eq:rescaled_negative_energy_condition} can have negative energies, while particle outside will never have negative energies. Note that this notion of ergosphere depends not only on the geometry of the spacetime, but also on properties of the particle.

In order to have a graphical representation of the generalized ergosphere, let us use cartesian-like coordinates, also reescaled by $M_1$, $(\overline{x},\overline{y},\overline{z})$ related to Weyl's coordinates by $\overline{x}=\overline{\rho} \cos \phi$ and $\overline{y}=\overline{\rho} \sin \phi$. In Figs.~\ref{ch:penrose_binaries/fig:splitting_1}-\ref{ch:penrose_binaries/fig:splitting_4}, we plot the generalized ergosphere for a MP black hole binary of unitary mass-ratio $M$, whose horizons (represented by red dots) have also unitary separation parameter $\alpha$. As the value of $\mu$ increases, one can observe that the generalized ergosphere splits into two disconnected spherical regions with the same (albeit smaller than the original) radius. Thus we see that the bigger the value of $\mu$, the closer to the horizon an orbit must be in order to realize energy extraction. Extrapolating this result makes it easy to see that the maximum possible amount of energy can only be extracted if a particle is orbiting exactly \emph{at} the event horizon. We will explore this fact further in another section of this work but this result is in agreement with~\cite{RUFFINI1971}.

In Figs.~\ref{ch:penrose_binaries/fig:mass_increase_1}-\ref{ch:penrose_binaries/fig:mass_increase_2}, we turn to the influence of the mass-ratio in the generalized ergosphere's shape while keeping fixed $\mu = -1$ and $\alpha = 1$. The ergosphere starts of with the same shape as in Fig.~\ref{ch:penrose_binaries/fig:splitting_1}. As the mass-ratio increases, the ergosphere around one of the black holes is enlarged while the other's shrunken until a split-up occurs, and again, one ends up with two disconnected regions but this time of different sizes, the larger of which is centered around the heavier black hole. Again, this is to be expected. Because the MP solution represents extremally charged black holes, as the mas of the objects increase (or decrease) so does their electric charge thus allowing for more energy to be extracted and consequently a larger generalized ergosphere.

Finally, in Figs.~\ref{ch:penrose_binaries/fig:distance_1}-\ref{ch:penrose_binaries/fig:distance_2} we show the effects on the generalized ergosphere resulting from the increase in the distance parameter $\alpha$ with $M=1$ and $\mu=-1$ fixed. Initially the ergosphere is shaped as in Fig.~\ref{ch:penrose_binaries/fig:splitting_1}. Yet again, as the value of $\alpha$ increases, the ergosphere becomes elongated along the $z$ direction, assuming a `peanut'-like shape until it is split into two disconnected regions. From \eqref{ch:penrose_binaries/eq:rescaled_negative_energy_condition}, we can verify that the splitting occurs when $\alpha = -(M^{-1}+1)\mu$. If the black holes are moved further apart, the system will eventually behave as two isolated extremal black holes of mass $M$, each one surrounded by its own spherical generalized ergosphere.

It's also very important to point out that for small separations the generalized ergosphere of the binary system becomes equal to the one around a single RN black hole of mass $M_1 + M_2$. One can easily see that by taking the limit $a \rightarrow 0$ in \eqref{ch:penrose_binaries/eq:negative_energy_condition} and going to spherical coordinates, related to Weyl's coordinates trough $\rho = \sin\theta$ and $z=r\cos\theta$, which leads to the conclusion that that the generalized ergosphere is the spherical region centered around the origin which satisfies $r < -\mu (M_1 + M_2).$ This is precisely the same region as the one presented in Eq. (12) of Ref.~\cite{DENARDO1973} if one considers an extremal black hole in coordinates such that the event horizon is a point at the origin (which amounts the transformation $r \rightarrow r + M$) with the RN solution's mass parameter $M$ replaced by the total mass of the binary system, $M_1 + M_2$.

\begin{figure}[!htbp]
    \centering
    \subfloat[$\mu=-1$]{
        \includegraphics[width = 0.43\linewidth]{img/penrose_binaries/mp/ergo_1.pdf}
        \label{ch:penrose_binaries/fig:splitting_1}
    }
    \subfloat[$\mu=-0.6$]{
        \includegraphics[width = 0.43\linewidth]{img/penrose_binaries/mp/ergo_2.pdf}
        \label{ch:penrose_binaries/fig:splitting_2}
    }
    \newline
    \subfloat[$\mu=-0.5$]{
        \includegraphics[width = 0.43\linewidth]{img/penrose_binaries/mp/ergo_3.pdf}
        \label{ch:penrose_binaries/fig:splitting_3}
    }
    \subfloat[$\mu=-0.35$]{
        \includegraphics[width = 0.43\linewidth]{img/penrose_binaries/mp/ergo_4.pdf}
        \label{ch:penrose_binaries/fig:splitting_4}
    }
    \caption{Splitting and shrinking of the generalized ergosphere caused by an increase in the charge-to-mass ratio $\mu$ of a test particle. The separation parameter and mass-ratio are kept fixed at $\alpha = 1$ and $M = 1$, respectively. Note that the splitting produces two symmetric regions that are closer to the horizons.}
    \label{ch:penrose_binaries/fig:splitting_ergosphere_mass_increase}
\end{figure}

\begin{figure}[!htbp]
    \centering
    \subfloat[$M=2$]{
        \includegraphics[width = 0.43\linewidth]{img/penrose_binaries/mp/ergo_5.pdf}
        \label{ch:penrose_binaries/fig:mass_increase_1}
    }
    \subfloat[$M=20/3$]{
        \includegraphics[width = 0.43\linewidth]{img/penrose_binaries/mp/ergo_6.pdf}
        \label{ch:penrose_binaries/fig:mass_increase_2}
    }
    \caption{Splitting of the generalized ergosphere caused by an increase in the mass ratio $M$. The separation parameter and charge-to-mass ratio are kept fixed at $\alpha = 1$ and $\mu = -1$, respectively. The final region is much larger around the heavier black hole.}
    \label{ch:penrose_binaries/fig:splitting_ergosphere_mass_increase}
\end{figure}

\begin{figure}[!htbp]
    \centering
    \subfloat[$\alpha=2$]{
        \includegraphics[width = 0.43\linewidth]{img/penrose_binaries/mp/ergo_7.pdf}
        \label{ch:penrose_binaries/fig:distance_1}
    }
    \subfloat[$\alpha=3$]{
        \includegraphics[width = 0.43\linewidth]{img/penrose_binaries/mp/ergo_8.pdf}
        \label{ch:penrose_binaries/fig:distance_2}
    }
    \caption{Splitting of the generalized ergosphere caused by an increase in the distance parameter $\alpha$ The mass ratio and charge-to-mass ratio are kept fixed at $M = 1$ and $\mu = -1$, respectively. The split region is symmetric and it is not closer farther than it previously was to either horizon.}
    \label{ch:penrose_binaries/fig:splitting_ergosphere_distance_increase}
\end{figure}

\subsection{Negative energy orbits and the Penrose process}

Having understood the notion of generalized ergospheres around binary black holes described by the MP metric, we are now in position to analyze the possibility of energy extraction in such spacetimes. The idea follows Penrose's original proposal~\cite{PENROSE1971} and its extension to RN black holes~\cite{RUFFINI1971,DENARDO1973}. The mechanism we shall explore consists in sending a negatively charged particle from sufficiently far away towards the binary black hole.  Once the particle enters the generalized ergosphere, we need it to break up into two pieces. Depending on how the splitting process is set up, one of the pieces will escape back to infinity with more energy than the incident particle had.

In order to accomplish this one needs three trajectories. The first one, with positive energy $E^{(0)} > 0$ and labeled $T^{(0)}$, must start outside the ergosphere and end inside it, at the point where the other two trajectories, labeled $T^{(1)}$ and $T^{(2)}$ start. One of them, let us say $T^{(1)}$, must have negative energy, i.e.~$E^{(1)}<0$, and, therefore, will remain confined inside the ergosphere. The other one, $T^{(2)}$, on the other hand, will have positive energy $E^{(2)} > E^{(0)}$, which allows it to escape the ergospohere and reach back infinity.

Suppose that the break-up of the particle occurs at $(\rho_0,\phi_0,z_0)$. Let $m^{(i)}$ and $\tens{P}{(i)}{\mu}$ denote, respectively, the mass and the the 4-momentum of the particle on trajectory $T^{(i)}$. The conservation of the four-momentum, applied at the break-up point, reads
\begin{equation}
    \tens{P}{(0)}{\mu} = \tens{P}{(1)}{\mu} + \tens{P}{(2)}{\mu}.
    \label{ch:penrose_binaries/eq:four_momentum_conservation}
\end{equation}
%
Each component of the vector equation above yields a different conservation equation with straightforward physical interpretation. The zero-component, for a start, is just the conservation of total energy, i.e.
\begin{equation}
    m^{(0)}E^{(0)} = m^{(1)}E^{(1)} + m^{(2)}E^{(2)},
    \label{ch:penrose_binaries/eq:conservation_of_charge_energy}
\end{equation}
while the angular component is simply the conservation of angular momentum around the $z$ axis, namely
\begin{equation}
    m^{(0)}L^{(0)} = m^{(1)}L^{(1)} + m^{(2)}L^{(2)}.
    \label{ch:penrose_binaries/eq:conservation_ang_mom}
\end{equation}
The other two components represent the conservation of linear momentum along the radial direction and along the $z$ axis, respectively
\begin{equation}
    \tens{m}{(0)}{}\tens{\dot{\rho}}{(0)}{} = \tens{m}{(1)}{}\tens{\dot{\rho}}{(1)}{} + \tens{m}{(2)}{}\tens{\dot{\rho}}{(2)}{}
    \label{ch:penrose_binaries/eq:conservation_rho}
\end{equation}
%
and
%
\begin{equation}
    \tens{m}{(0)}{}\tens{\dot{z}}{(0)}{} = \tens{m}{(1)}{}\tens{\dot{z}}{(1)}{} + \tens{m}{(2)}{}\tens{\dot{z}}{(2)}{},
    \label{ch:penrose_binaries/eq:conservation_z}
\end{equation}

where all the derivatives are to be evaluated at the break-up point $(\rho_0,\phi_0,z_0)$.

Because we've assumed that $E^{(1)} < 0$, it's easy to see from \eqref{ch:penrose_binaries/eq:conservation_of_charge_energy} that the particle on trajectory $T^{(2)}$ will be more energetic than the particle on trajectory $T^{(0)}$ and thus it has \emph{gained} energy during the split-up process. Obviously this extra energy must come from somewhere. In Penrose's original proposal, the extra energy came from a small loss of angular momentum in the background Kerr black hole. This idea can be readily extended to charged black holes: now instead of angular momentum, the black hole donates part of it's electromagnetic energy to an outgoing particle. One might naturally ask how much energy can be extracted using this process. This was not answered by Penrose but by Christodoulou in Ref.~\cite{CHRISTODOULOU1970}. It turns out that the mass of a black hole cannot be smaller than a certain \emph{irreducible mass}, and about 50\% of a charged black hole's electromagnetic energy and 29\% of a rotating black hole's angular momentum can be extracted through this process. An important variation of the Penrose process is to consider particle collisions, instead of break-ups. This is befittingly known as the collisional Penrose process and it is capable of producing much more energetic ejecta that could be a potential source of gamma rays and high energy cosmic rays (see e.g.~\cite{PhysRevLett.114.251103}), however, in our work we will only focus on the non-colisional version the Penrose process.

The existence of a generalized ergosphere as we've shown, indicates that negative energy orbits can exist and thus electromagnetic energy can be extracted from a MP binary through the Penrose process. However we are not only interested in where these orbits are located but also on some of their features, e.g., weather they fall into one of the black holes or keep orbiting the system in a closed orbit. In the following sections we will describe briefly the characteristics of negative energy orbits constrained to two planes of motion around the system. This analysis will come into play later when we construct two explicit examples of the penrose process in action. We start by introducing two quantities, the \emph{effective energy} $E_{\text{eff}}(\rho,z)$ and the \emph{effective potential} $V_{\text{eff}}(\rho,z)$, given by

\begin{equation}
    E_{\text{eff}}(\rho,z) \equiv E - \mu\left(1 - \frac{1}{U(\rho,z)}\right)
    \label{ch:penrose_binaries/eq:effective_energy_definition}
\end{equation}
%
and

\begin{equation}
    V_{\text{eff}}(\rho,z) \equiv \frac{L^2}{\rho^2 U(\rho,z)^4} + \frac{1}{U(\rho,z)^2}.
    \label{ch:penrose_binaries/eq:effective_potential_definition}
\end{equation}

With that, we can recast Eq.~\eqref{ch:penrose_binaries/eq:alternative_expression_for_energy} as

\begin{equation}
    \dot{\rho}^2 + \dot{z}^2 = E_{\text{eff}}(\rho,z)^2 - V_{\text{eff}}(\rho,z).
    \label{ch:penrose_binaries/eq:first_order_ode_for_orbits}
\end{equation}

\subsubsection{Orbits in the \texorpdfstring{$\rho$}{$\symit{rho}$}-$z$ plane}

One can easily see that by setting $L$ (and thus $\dot{\phi}$) to zero, the particle's orbit will be constrained to a plane that contains both of the black holes, which we call the $\rho$-$z$ plane. Because of the system's cylindrical symmetry one can set $\phi = 0$ without any loss of generality. To be able to integrate the equations of motion \eqref{ch:penrose_binaries/eq:ode_for_rho_motion} and \eqref{ch:penrose_binaries/eq:ode_for_z_motion} one must specify the remaining 4 initial conditions for $\rho$, $z$ and their time derivatives. Note however that $E$, $\dot{\rho}$ and $\dot{z}$ are related trough \eqref{ch:penrose_binaries/eq:first_order_ode_for_orbits} and thus it suffices to choose $\rho(0)$, $z(0)$, $\dot{z}(0)$ and $E$ while using \eqref{ch:penrose_binaries/eq:first_order_ode_for_orbits} to determine $\dot{\rho}(0)$.

In order to visualize the orbits we employ once again Cartesian-like coordinates $(\overline{x},\overline{y},\overline{z})$ reescaled by $M_1$, $M$ as the system's mass-ratio and $\alpha$ as it's separation parameter. In Figs.~\ref{ch:penrose_binaries/fig:orbit_1} and \ref{ch:penrose_binaries/fig:orbit_2} the event horizons are represented by black dots and the spacetime parameters were fixed at $M = \alpha = 1$. Additionally, because our process for finding $\dot{\rho}(0)$ involves square roots we use solid black lines for orbits resulting from the positive solution and dashed black lines for the negative solution. By extensively testing for many different initial conditions, we found that most orbits end up in either of the two black holes as in Fig.~\ref{ch:penrose_binaries/fig:orbit_1}. This is not surprising, as particles that follow negative energy orbits must always be charged oppositely to the black holes and thus they will always be electrically attracted towards the horizons. On the other hand, by choosing parameters very carefully, one is be able to find highly unstable but closed orbits as in Fig.~\ref{ch:penrose_binaries/fig:orbit_2}.

At a first glance, closed negative energy orbits might cause some concern as they never fall into one of the horizons and thus never effectively interact with the black holes to cause them to loose a small amount of electric charge but one must keep in mind that in the MP spacetime such orbits are always unstable and are completely within the generalized ergosphere. Additionally, the existence of closed negative energy orbits is not an exclusive feature of the MP metric. Usually these orbits occur only \emph{inside} event horizons or in naked singularities. Stuchlik has shown in Ref.~\cite{STUCHLIK1980} that there can exist stable circular orbits of negative energy around a Kerr naked singularity and that the Penrose process can effectively take place using said orbits. Patil and collaborators in Ref.~ \cite{PATIL2016} have even used this fact to construct a model to explain the origin of ultra high energy particles detected on Earth as a result of the colisional Penrose process in a slightly over spinning Kerr black hole. Similarly, Mukherjee and Nayak  in Ref.~\cite{MUKHERJEE2018} have used the colisional Penrose process in a Kerr naked singularity with orbits outside the equatorial plane to construct a model for energetic outgoing particle jets. In some situations though, closed negative energy orbits can exist outside of the event horizon as Prasana and Dadhich have shown in Ref.~\cite{PRASANA1982} in the spacetime of a Kerr black hole immersed in an external magnetic field that is regular up to the event horizon. More recently Felice and collaborators in Ref.~\cite{FELICE2004} also have shown that magnetized particles in a Kerr black hole can give rise to stable circular orbits of negative energy inside the ergosphere. Therefore, these unstable orbits that we've found are not violating any underlying fundamental principle.

\begin{figure}[!htbp]
    \centering
    \subfloat[$\overline{\rho}(0) = 0$, $\overline{z}(0) = 4$, $\dot{\overline{z}} = 0$, $E = -0.1$, $\mu = -4$]{
        \includegraphics[width = 0.51\linewidth]{img/penrose_binaries/mp/orbit_1.pdf}
        \label{ch:penrose_binaries/fig:orbit_1}
    }
    \subfloat[$\overline{\rho}(0) = 0$, $\overline{z}(0) = 4.06$, $\dot{\overline{z}} = 0$, $E = -1$, $\mu = -9.63222$]{
        \includegraphics[width = 0.51\linewidth]{img/penrose_binaries/mp/orbit_2.pdf}
        \label{ch:penrose_binaries/fig:orbit_2}
    }
    \caption{Two different types of orbits in the $\rho$-$z$ plane: terminating (left) and closed (right). Initial conditions used for each orbits are presented in their respective captions. The dashed trajectory represents that of negative $\dot{\overline{\rho}}(0)$.}
    \label{ch:penrose_binaries/fig:rho_z_orbits}
\end{figure}

\subsubsection{Orbits in the $z=0$ plane}

If we set $z(0) = \dot{z}(0) = 0$ we confine the particle's motion to the plane exactly in between the two black holes and once the background spacetime parameters have been set the trajectory can now be readily determined using only Eq.~\eqref{ch:penrose_binaries/eq:first_order_ode_for_orbits} with appropriate values for $\rho(0)$, $\phi(0)$, $L$, $E$ and $\mu$. Thanks to the constrained motion we can now take full advantage of the effective potential formulation that we've introduced earlier. Our search revealed that particles with negative energy in this plane stay in a perpetual oscillatory motion either in straight lines or more complicated precessing orbits, depending on the value of $L$. These oscilating orbits are similar in nature to the closed orbits found in the $\rho$-$z$ plane as they also don't fall towards either of the black holes but they are stable in the sense that they do not require any fine tuning of the orbital parameters to be found, although they are globally unstable as any perturbation along the $z$ direction will cause the to fall towards one of the black holes. Proceeding with the same rescaling by $M_1$ scheme we've adopted so far, we plot in Figs.~\ref{ch:penrose_binaries/fig:orbit_z_1} and \ref{ch:penrose_binaries/fig:orbit_z_2} two oscillating orbits for a binary system with $M = \alpha = 1$. Note that if an infinite amount of time s allowed to pass, a particle following these trajectories will have swept the whole finite area between the minimum and maximum values attained by $\overline{\rho}$ during motion.

It's very hard (or even impossible) to obtain the orbital turning points of any oscilating orbit analytically by solving $E_{\text{eff}}(\rho,0)^2 - V_{\text{eff}}(\rho,0) = 0$ for $\rho$ as one can show that the problem reduces to finding the roots of a sixth-order polynomial equation. Despite that, its quite easy to find the roots of the polynomial numerically after the orbital parameters have been chosen and thus this is the approach we've adopted as a complete and rigorous analysis of the orbital turning points go beyond the scope of this work. In Figs.~\ref{ch:penrose_binaries/fig:veff_for_3} and \ref{ch:penrose_binaries/fig:veff_for_4} we plot the squared effective energy and the effective potential to the orbits described by \ref{ch:penrose_binaries/fig:orbit_z_1} and \ref{ch:penrose_binaries/fig:orbit_z_2} respectively and give their turning points. The effective potential is represented by a solid black line while the squared effective energy is represented by a dashed red line.

\begin{figure}[!htbp]
    \centering
    \subfloat[$\overline{\rho}(0) = 2$, $\phi(0) = 0$, $L = 2$, $E = -1$ and $\mu = -4$.]{
        \includegraphics[width = 0.5\linewidth]{img/penrose_binaries/mp/orbit_3.pdf}
        \label{ch:penrose_binaries/fig:orbit_z_1}
    }
    \subfloat[$\overline{\rho}(0) = 2$, $\phi(0) = 0$, $L = 10$, $E = -1$ and $\mu = -6$.]{
        \includegraphics[width = 0.5\linewidth]{img/penrose_binaries/mp/orbit_4.pdf}
        \label{ch:penrose_binaries/fig:orbit_z_2}
    }
    \caption{Two oscilating orbits in the $z = 0$ symmetry plane. Note that the area swept by a particle in either orbit after an infinite amount of time is finite and proportional to the difference between the and outer orbital radii.}
    \label{ch:penrose_binaries/fig:z_orbits}
\end{figure}

\begin{figure}[!htbp]
    \centering
    \subfloat[Turning points: $\overline{\rho}_{\text{min}} = 0.1386$ and $\overline{\rho}_{\text{max}} = 2.6883$.]{
    \includegraphics[width = 0.5\linewidth]{img/penrose_binaries/mp/veff_for_3.pdf}
    \label{ch:penrose_binaries/fig:veff_for_3}
    }
    \subfloat[Turning points: $\overline{\rho}_{\text{min}} = 0.4352$ and $\overline{\rho}_{\text{max}} = 2.8091$.]{
    \includegraphics[width = 0.5\linewidth]{img/penrose_binaries/mp/veff_for_4.pdf}
    \label{ch:penrose_binaries/fig:veff_for_4}
    }
    \caption{The effective potential (solid black line) and squared effective energy (dashed red line) for the orbits depicted in \ref{ch:penrose_binaries/fig:orbit_z_1} (left) and \ref{ch:penrose_binaries/fig:orbit_z_2} (right).}
    \label{ch:penrose_binaries/fig:veff_graphs}
\end{figure}

\subsection{Examples of Penrose processes}

Let us now give concrete examples of energy extraction in a binary black hole spacetime. As we've already seen, under a few different circumstances such as when the black holes are sufficiently apart, the charge-to-mass ratios of the fragments are sufficiently small or when the mass ratio is large, the ergosphere is composed by two disconnected regions. When this happens, the only way to extract energy from the binary is by having the negative energy trajectory reaching the event horizon of one of the black holes. Consequently, the negative energy particle will eventually reach one of the singularities. If, on the other hand, the black holes are sufficiently close to each other, an interesting possibility opens up. Instead of having the negative energy orbit reach one of the event horizons, we can engineer the break up so that it will exist forever outside the black holes, confined to the ergosphere of the binary using trajectories like those in Figs.~\ref{ch:penrose_binaries/fig:orbit_2}, \ref{ch:penrose_binaries/fig:orbit_z_1} or \ref{ch:penrose_binaries/fig:orbit_z_2}. Naturally, $T^{(1)}$ (the negative energy trajectory) will only make sense physically if $\dot{\rho}^2 \ge 0$ everywhere. Thus, once $E^{(1)}$ and $L^{(1)}$ have been specified, the requirement that\label{ch:penrose_binaries/eq:effective_potential_orbit_equation} be non-negative constrains the possible value for the break-up coordinate $\rho_0$. Note that the angular break-up coordinate $\phi_0$ is irrelevant due to the system being symmetric with respect to rotations around the $z$ axis.

To construct an example, we start of by selecting a negative energy trajectory $T^{(1)}$ and a break-up point inside the ergosphere. After that, we construct an arbitrary orbit of positive energy that reaches infinity to serve as $T^{(2)}$. By solving the system formed by Eqs.~\eqref{ch:penrose_binaries/eq:four_momentum_conservation}-\eqref{ch:penrose_binaries/eq:conservation_rho} together with the conservation of electric charge, one is able to determine all of the orbital parameters for $T^{(0)}$. In Table~\ref{ch:penrose_binaries/tab:exemple_orbital_parameters_1} we summarize the orbital parameters for the process depicted in Fig. \ref{ch:penrose_binaries/fig:penrose_example_1}. The particle comes in orbit $T^{(1)}$ represented by the solid black curve and then splits at the grey dot into a negative energy orbit $T^{(1)}$ represented by a blue dotted curve and a positive energy orbit $T^{(2)}$ represented by a red dashed curve. In this example, we've set the spacetime parameters at $M = \alpha = 1$ and the two black dots represent the black hole's event horizons. Note that the entire process is confined to the $\rho$-$z$ plane.

\begin{table}[!htbp]
    \centering
    \caption{Orbital parameters for an example Penrose process in the $\rho$-$z$ plane.}
    \begin{tabular}{c|c|c|c|c|c|c|c|}
        \cline{1-8}
        \multicolumn{1}{|c|}{$T^{(i)}$} & $\mu^{(i)}$ & $E^{(i)}$ & $L^{(i)}$ & $\rho_0^{(i)}$ & $z_0^{(i)}$ & $\dot{\rho}_0^{(i)}$ & $\dot{z}_0^{(i)}$ \\ \hline
        \multicolumn{1}{|c|}{$T^{(0)}$} & $0.16076$   & $3.93401$ & $0$       & $0$            & $4.06$      & $3.82283$            & $0$               \\ \hline
        \multicolumn{1}{|c|}{$T^{(1)}$} & $-9.63222$  & $-1$      & $0$       & $0$            & $4.06$      & $-2.21869$           & $0$               \\ \hline
        \multicolumn{1}{|c|}{$T^{(2)}$} & $10$        & $10$      & $0$       & $0$            & $4.06$      & $6.52697$            & $0$               \\ \hline
    \end{tabular}
    \label{ch:penrose_binaries/tab:exemple_orbital_parameters_1}
\end{table}

\begin{figure}[!htbp]
    \centering
    \includegraphics[scale = 0.4]{img/penrose_binaries/mp/penrose_xz.pdf}
    \caption{Example of a penrose process in the $\rho$-$z$ plane. The incoming trajectory (solid black) splits at the grey point into an negative energy (dotted blue) and increased energy (dashed red) orbit.}
    \label{ch:penrose_binaries/fig:penrose_example_1}
\end{figure}

If instead we choose to confine our orbits to the $z=0$ plane, by employing the same algorithm with the same background spacetime parameters as before, we obtain the orbital parameters described in Tab.~\ref{ch:penrose_binaries/tab:exemple_orbital_parameters_2} and depicted in Fig. \ref{ch:penrose_binaries/fig:penrose_example_2}. Here we've chosen to draw only a few periods of the oscilating trajectory $T^{(1)}$ to improve the figure's clarity.

\begin{table}[!htbp]
    \centering
    \caption{Orbital parameters for an example Penrose proces in the $z=0$ plane.}
    \begin{tabular}{c|c|c|c|c|c|c|c|}
        \cline{1-8}
        \multicolumn{1}{|c|}{$T^{(i)}$} & $\mu^{(i)}$ & $E^{(i)}$ & $L^{(i)}$ & $\rho_0^{(i)}$ & $z_0^{(i)}$ & $\dot{\rho}_0^{(i)}$ & $\dot{z}_0^{(i)}$ \\ \hline
        \multicolumn{1}{|c|}{$T^{(0)}$} & $-0.85050$  & $2.18701$ & $1.70101$ & $2$            & $0$         & $2.52301$            & $0$               \\ \hline
        \multicolumn{1}{|c|}{$T^{(1)}$} & $-4$        & $-1$      & $2$       & $2$            & $0$         & $-0.65820$           & $0$               \\ \hline
        \multicolumn{1}{|c|}{$T^{(2)}$} & $0.5$       & $10$      & $4$       & $2$            & $0$         & $-9.72474$           & $0$               \\ \hline
    \end{tabular}
    \label{ch:penrose_binaries/tab:exemple_orbital_parameters_2}
\end{table}

\begin{figure}[!htbp]
    \centering
    \includegraphics[scale = 0.43]{img/penrose_binaries/mp/penrose_xy.pdf}
    \caption{Example of a Penrose process that happens entirely on the $z=0$ plane of a MP binary black hole. Once again, the incoming trajectory (solid black) splits at the grey point into an negative energy (dotted blue) and increased energy (dashed red) orbit.}
    \label{ch:penrose_binaries/fig:penrose_example_2}
\end{figure}

\subsection{Efficiency of energy extraction}

The efficiency $\eta$ of the Penrose process can be trivially defined as the ratio between the energy input and output. Using Eq.~\eqref{ch:penrose_binaries/eq:conservation_of_charge_energy}, we can see that

\begin{equation}
    \eta = - \frac{m^{(1)}}{m^{(0)}} \frac{E^{(1)}}{E^{(0)}}.
    \label{ch:penrose_binaries/eq:penrose_efficiency_general}
\end{equation}

One can choose $E^{(1)} = 1$ and $m^{(1)}/m^{(0)} = 1$ without any loss of generality. Furthermore it's clear that in order to obtain maximum efficiency, one must choose $E^{(1)}$ to be as negative as possible, and thus of the form of Eq.~\eqref{ch:penrose_binaries/eq:minimum_energy}. With these choices, the efficiency becomes

\begin{equation}
    \eta = \mu^{(1)}\left(\frac{1}{U} - 1\right) - \frac{1}{U}.
    \label{ch:penrose_binaries/eq:penrose_efficiency_max}
\end{equation}

Notice that because \eqref{ch:penrose_binaries/eq:penrose_efficiency_max} depends on $U$, the efficiency depends on the particle's position. Using spherical coordinates centered around the origin, one can see that as $a\rightarrow 0$,

\begin{equation}
    \eta = -\frac{\mu^{(1)}(M_1 + M_2) + r}{M_1 + M_2 + r},
    \label{ch:penrose_binaries/eq:a_zero_efficiency}
\end{equation}
%
which is precisely the efficiency one would obtain for and extremal RN black hole of mass $M_1 + M_2$ in a coordinate system where the event horizon is a point. If the particle is locate precisely at the event horizon, and thus $r = 0$, the efficiency attains it's maximum value of $-\mu^{(1)}$. The direct consequence of this fact is that one can have Penrose processes in which the efficiency of energy extraction goes above 100\% (as $\mu^{(1)}$ can be arbitrarily large) if the participating particles are electrically charged, a well know fact pointed out in Refs.~\cite{bhat1985energetics} and \cite{parthasarathy1986high}.

Additionally, because $0 \leq 1/U \leq 1$, one can easily see from Eq.~\eqref{ch:penrose_binaries/eq:penrose_efficiency_max} that $0 \leq \eta \leq -\mu^{(1)}$ and thus the presence of the second black hole does not allow for more efficient processes than the single RN black hole. It's also noteworthy that $\eta$ attains it's maximum value when $1/U = 0$ which occurs only when the particle is exactly over one of the event horizons. In Fig.~\ref{ch:penrose_binaries/fig:efficiency}, we plot the extraction efficiency for a MP binary with mass ratio $M = 1$ and varying separation parameter $\alpha$ for a particle of charge-to-mass ratio $\mu = -1$ located at $\overline{r} = 1$ and $\overline{\theta} = 0$ (solid black line), $\overline{\theta} = \pi/8$ (dashed red line) and $\overline{\theta} = \pi/2$ (dotted blue line). The maximum value of $\alpha$ is caped by the generalized ergosphere condition of Eq.\eqref{ch:penrose_binaries/eq:rescaled_negative_energy_condition}. Note that the maximum efficiency of $-\mu = 2$ is attainable only when the particle is located at one of the event horizons, namely when $\alpha = 1$ and $\overline{\theta} = 0,\pi/2$.

\begin{figure}[!htbp]
    \centering
    \includegraphics[scale = 0.4]{img/penrose_binaries/mp/efficiency.pdf}
    \caption{Extraction efficiency for a particle $\overline{r} = 1$ and $\overline{\theta} = 0$ (solid black line), $\overline{\theta} = \pi/8$ (dashed red line) and $\overline{\theta} = \pi/2$ (dotted blue line). Maxmimum efficiency is attained at $\alpha = 1$ and $\overline{\theta} = 0,\pi/2$.}
    \label{ch:penrose_binaries/fig:efficiency}
\end{figure}

\section{\ac{CMMR} Spacetime}
\label{ch:penrose_binaries:sec:cmmr_penrose}
We now consider the extension of the results presented in Sec.~\ref{ch:penrose_binaries:sec:mp_penrose} to a binary system of rotating black holes described by the CMMR metric. In Weyl's cylindrical coordinates, the CMMR line element reads
%
\begin{multline}
  \ud s^2 = -f(\rho,z)\left[\ud t - \omega(\rho,z)\ud\phi\right]^2 \\
  + f(\rho,z)^{-1}\left[e^{2\gamma(\rho,z)}\left(\ud \rho^2 + \ud z^2\right) + \rho^2\ud\phi^2\right],
  \label{eq:cmmr_line_element}
\end{multline}
%
where the real valued functions $f(\rho,z)$, $\omega(\rho,z)$ and $\exp[2\gamma(\rho,z)]$ are defined as in Sec.~IV of Ref.~\cite{manko_ruiz_thermo}. As in the case of the MP metric, only the exterior of the black holes is described by these coordinates. In particular, the outer event horizons of the constituent black holes are straight lines in these coordinates (see Fig.~\ref{fig:cmmr_ergospheres}).

The CMMR solution is fully characterized by five independent parameters, namely the masses $M_{1,2}$, the angular momenta per unit mass $a_{1,2}$ and the coordinate distance $R$ between the black hole centers. From these, we define three additional parameters, namely $M_T = M_1 + M_2$, which represents the total mass of the system, $J_T = M_1 a_1 + M_2 a_2$, which represents the total angular momentum of the system, and $a_*$, which is a root of the cubic equation
%
\begin{equation}
  \left(R^2 - M_T^2 + a_*^2\right) \left(a_1 + a_2 - a_*\right) + 2 \left(R + M_T \right)\left(J_T - M_T a_* \right) = 0.
  \label{eq:cubic_eq_for_a}
\end{equation}

We note that, depending on the parameters, the CMMR metric can represent a black hole-black hole binary, a naked singularity-naked singularity binary, or a black hole-naked singularity binary~\cite{cabrera_metric,manko_ruiz_metric, manko_ruiz_thermo}. In our analysis, the chosen parameters always correspond to a binary black hole solution. In practice, this means that the chosen parameters must:
%
\begin{enumerate}
  \item Produce real valued and positive horizon lengths. The horizon half-lengths are given by the expressions $\sigma_1$ and $\sigma_2$ in Sec.~IV of Ref.~\cite{manko_ruiz_thermo}.
  \item Produce horizons that do not touch or overlap.
  \item Produce a single real root for $a_*$ in Eq.~\eqref{eq:cubic_eq_for_a}.
\end{enumerate}

\subsection{Geodesics}

Once again, we will make use of the Lagrangian formalism to determine the geodesic equations and the conserved quantities corresponding to the symmetries of the system. The Lagrangian associated with the geodesic motion of a massive and neutral test particle in the CMMR metric is~\cite{Dubeibe2016}
%
\begin{equation}
  2\mathcal{L} = -f(\dot{t} - \omega\dot{\phi})^2 +f^{-1}\left[e^{2\gamma}\left( \dot{\rho}^2 + \dot{z}^2 \right) + \rho^2\dot{\phi}^2 \right],
  \label{eq:cmmr_lagrangian}
\end{equation}
where, once again, dots represent derivatives with respect to the proper time $\lambda$.

%
Due to the stationarity and the axisymmetry of the system, we can identify two constants of the motion analogous to the quantities defined in Eqs.~\eqref{eq:conserved_energy}  and \eqref{eq:conserved_momentum}. The energy per unit mass, as measured by a static observer at infinity, is given by
%
\begin{equation}
  E = f\dot{t} - \omega f\dot{\phi},
  \label{eq:cmmr_energy_t_phi}
\end{equation}
%
and the angular momentum (with respect to the $z$ axis) per unit mass, as measured by a static observer at infinity, is given by
%
\begin{equation}
  L = \omega f \dot{t} + \left( \frac{\rho^2}{f} - \omega^2 f \right)\dot{\phi}.
  \label{eq:cmmr_ang_mom_t_phi}
\end{equation}
%

Using Eqs.~\eqref{eq:cmmr_energy_t_phi} and \eqref{eq:cmmr_ang_mom_t_phi} to eliminate $\dot{t}$ and $\dot{\phi}$ from the normalization of the four velocity, i.e. $\dot{x}^\mu\dot{x}_\mu = -1$, we obtain an expression for the energy $E$ in  terms of $\dot{\rho}$, $\dot{z}$ and the angular momentum $L$:
%
\begin{align}
  E  = & \frac{-f^2\omega L}{\rho^2-\omega^2 f^2}  + \left[ \frac{\rho^2e^{2\gamma}(\dot{\rho}^2 + \dot{z}^2)}{\rho^2 - \omega^2f^2}  \right. \nonumber \\
       & \left. + \left( \frac{\rho fL}{\rho^2 - \omega^2f^2} \right)^2 + \frac{\rho^2f}{\rho^2 - \omega^2f^2} \right]^{1/2},
  \label{eq:cmmr_energy_rho_z}
\end{align}
%
where the positive sign is once again chosen for the square root in order to guarantee that a static particle at infinity has positive energy. We rewrite the equation above as Eq.~\eqref{eq:effective1}, where the effective energy and the effective potential are now given by
%
\begin{equation}
  V_{\text{eff}} = \frac{\rho^2 - \omega^2 f^2}{\rho^2 e^{2\gamma}}\left[ \left( \frac{\rho fL}{\rho^2 - \omega^2f^2} \right)^2 + \frac{\rho^2f}{\rho^2 - \omega^2f^2}\right]
  \label{eq:cmmr_effective_potential}
\end{equation}
%
and
%
\begin{equation}
  E_{\text{eff}} = \frac{\rho^2 - \omega^2 f^2}{\rho^2 e^{2\gamma}}\left(E + \frac{f^2\omega L}{\rho^2-\omega^2 f^2}\right)^2.
  \label{eq:cmmr_effective_energy}
\end{equation}
%
%
Note that the constraints given in Eq.~\eqref{eq:effective_constraints} also apply to Eqs.~\eqref{eq:cmmr_effective_potential} and \eqref{eq:cmmr_effective_energy}.

The geodesic equations, analogous to Eqs.~\eqref{eq:ode_for_rho_motion} and \eqref{eq:ode_for_z_motion}, can be derived from the Euler-Lagrange equations for the Lagrangian \eqref{eq:cmmr_lagrangian}. Their explicit forms, in terms of $E$, $L$ and the metric functions $f$, $\omega$ and $e^{2\gamma}$, are given in Eqs.~(16) and (17) of Ref.~\cite{Dubeibe2016}. Once $\rho(\lambda)$ and $z(\lambda)$ are known, $t(\lambda)$ and $\phi(\lambda)$ are determined by direct integration of Eqs.~\eqref{eq:cmmr_energy_t_phi} and \eqref{eq:cmmr_ang_mom_t_phi}.


\subsection{Ergosphere}

Since the CMMR metric is stationary, and we are considering neutral particles in geodesic motion, we use the standard definition of an ergosphere to study the possibility of negative energy orbits and energy extraction. In other words, the ergosphere is the region where the time translation Killing vector field becomes space-like, i.e.~$(\partial_t)^\mu (\partial_t)_\mu > 0$. Taking into account the line element \eqref{eq:cmmr_line_element}, it is straightforward to show that the ergosphere of the CMMR spacetime is the locus of points that satisfy
%
\begin{equation}
  f(\rho,z) < 0.
  \label{eq:cmmr_ergo_ineq}
\end{equation}

We sketch this ergosphere in Fig.~\ref{fig:cmmr_ergospheres}, where each panel is labeled by a letter (\textbf{A}-\textbf{I}) and corresponds to a different set of parameters (which are specified in Table \ref{tab:cmmr_ergo_tab}). In each panel, the blue shaded region represents the $\phi = 0$ section of the ergosphere, while the red lines represent the event horizons of the black holes. The top row of the figure (panels \textbf{A}-\textbf{C}) exhibits the effect of changing the mass ratio of the system while keeping the total mass, both spins and the separation parameter fixed. It shows that, analogously to the MP case, initially disjoint ergospheres may merge into a single connected ergosphere when the mass ratio increases. The middle row of Fig.~\ref{fig:cmmr_ergospheres} (panels \textbf{D}-\textbf{E}), on the other hand, shows the effect of changing the spin parameter of the top black hole while keeping all other parameters fixed. We observe that when initially aligned spins become anti-aligned, the ergosphere becomes thinner and elongated along the symmetry axis. Finally, the bottom row (panels \textbf{G}-\textbf{H}) illustrates the effect of increasing the separation parameter when all other parameters are kept fixed. Similarly to what happens in the MP case, if the distance between the black holes is sufficiently large, there will be two disconnected ergospheres, one for each black hole.

\begin{figure}[h]
  \centering
  \includegraphics[scale=0.70]{img/penrose_binaries/fig12.pdf}
  \caption{The $\phi=0$ section of the ergosphere of the CMMR metric for different set of parameters labeled \textbf{A}-\textbf{I} (see Table  \ref{tab:cmmr_ergo_tab}). In each plot the horizontal and vertical axes are $\rho/M_T$ and $z/M_T$, respectively. The red lines indicate the location of the black hole's horizons.}
  \label{fig:cmmr_ergospheres}
\end{figure}

\begin{table}[h]
  \centering
  \begin{tabular}{ccccc}
    \hline\hline
    Panel      & $M_1/M_2$ & $a_1/M_T$ & $a_2/M_T$ & $R/M_T$ \\
    \textbf{A} & $0.16$    & $0.65$    & $0.65$    & $1.00$  \\
    \textbf{B} & $0.58$    & $0.65$    & $0.65$    & $1.00$  \\
    \textbf{C} & $1.00$    & $0.65$    & $0.65$    & $1.00$  \\
    \textbf{D} & $1.00$    & $0.50$    & $0.65$    & $1.00$  \\
    \textbf{E} & $1.00$    & $0.30$    & $0.65$    & $1.00$  \\
    \textbf{F} & $1.00$    & $-0.10$   & $0.65$    & $1.00$  \\
    \textbf{G} & $1.00$    & $0.65$    & $0.65$    & $1.11$  \\
    \textbf{H} & $1.00$    & $0.65$    & $0.65$    & $1.30$  \\
    \textbf{I} & $1.00$    & $0.65$    & $0.65$    & $2.00$  \\
    \hline\hline
  \end{tabular}
  \caption{Parameters that define the CMMR metrics associated with the ergospheres \textbf{A}-\textbf{I} shown in Fig.~\ref{fig:cmmr_ergospheres}.}
  \label{tab:cmmr_ergo_tab}
\end{table}


\subsection{Bound negative energy orbits and the Penrose Process}

To demonstrate the existence of bound negative energy orbits and the possibility of using them to extract energy from non-coalescing Kerr binaries, we shall restrict our attention to systems of equal mass and spin. This symmetry allows for the existence of stable orbits (in the sense already discussed for the MP metric) in the $z=0$ plane. To find a negative energy trajectory that is confined outside the black holes, we choose the energy $E$ and the angular momentum $L$ such that there are two orbital turning points of Eq.~\eqref{eq:effective1} that lie inside the ergosphere of the system. Similarly to what was done in the MP case, once the initial radius $\rho(0)$, the energy, and the angular momentum are fixed, we solve Eq.~\eqref{eq:cmmr_energy_rho_z} to determine $\dot{\rho}(0)$ and integrate the geodesic equations. Using the parameters that produce the ergosphere \textbf{C} of Fig.~\ref{fig:cmmr_ergospheres} and Table \ref{tab:cmmr_ergo_tab}, we show an example of such a negative energy orbit in Fig.~\ref{fig:cmmr_veff} (right panel). The corresponding effective potential and effective energy are also shown in Fig.~\ref{fig:cmmr_veff} (left panel).

\begin{figure}[h]
  \centering
  \includegraphics[scale=0.70]{img/penrose_binaries/fig13.pdf}
  \caption{Left panel: effective energy (black curve) and effective potential (purple dashed curve) for $L=-2.5$ and $E=-0.054$, when the CMMR metric is characterized by $a_1=a_2=0.65$, $M_1 = M_2 = 0.5$, and $R = 1$ (corresponding to the label \textbf{C} in Table \ref{tab:cmmr_ergo_tab}). The turning points are located at $\rho_{-}/M_T = 0.112531$ and at $\rho_{+}/M_T = 0.306081$. Right panel: the associated trajectory in the $z = 0$ plane when $\rho(0)/M_T=0.25$ and $\phi(0)=0$. The blue region is the $z=0$ section of the ergosphere of the spacetime.}
  \label{fig:cmmr_veff}
\end{figure}

Adopting the same notation introduced Sec.~\ref{ch:penrose_binaries:sec:mp_penrose} for the trajectories in a Penrose process around the MP black hole, and taking advantage of the negative energy orbit depicted in Fig.~\ref{fig:cmmr_veff}, we now consider the possibility of energy extraction in the CMMR spacetime. By employing the conservation of 4-momentum (as in Sec.~III), we construct an explicit example of a Penrose process. The obtained trajectories are shown in Fig.~\ref{fig:cmmr_penrose} and the corresponding parameters are given in Table \ref{tab:cmmr_penrose_example}. The efficiency of the process, calculated through Eq.~\eqref{eq:penrose_efficiency_general}, is $\eta \approx 0.08 \%$.

\begin{figure}[h]
  \centering
  \includegraphics[scale=0.70]{img/penrose_binaries/fig14.pdf}
  \caption{Penrose's process in the $z=0$ plane of a CMMR spacetime with $M_1=M_2=0.5$, $a_1=a_2=0.65$, and $R=1$. The incoming trajectory $T_{(0)}$ (black curve) splits at the black point ($\rho/M_T=0.25$, $\phi=0$) into the negative energy orbit $T_{(1)}$ (red curve) and the trajectory of the escaping fragment $T_{(2)}$ (purple curve). The parameters that generate these trajectories are shown in Table \ref{tab:cmmr_penrose_example}. The blue region is the $z=0$ section of the ergosphere (for particle 1).}
  \label{fig:cmmr_penrose}
\end{figure}

\renewcommand{\arraystretch}{1.2}
\begin{table}[h]
  \centering
  \begin{tabular}{ccccccc}
    \hline\hline
    $i$ & $m_{(i)}/m_0$ & $E_{(i)}$ & $L_{(i)}$         & $\dot{\rho}_{(i)}$ & $\dot{z}_{(i)}$ \\ \vspace{-0.3cm} \\
    0   & 1.0000000     & 2.00000   & 0         .000000 & 4.343904           & 0               \\
    1   & 0.0289697     & -0.05400  & -2.500000         & 0.313887           & 0               \\
    2   & 0.3148980     & 6.35623   & 0.229993          & 13.765800          & 0               \\
    \hline\hline
  \end{tabular}
  \caption{Parameters that generate the trajectories $T_{(0)}$, $T_{(1)}$, and $T_{(2)}$ of the Penrose process shown in Fig.~\ref{fig:cmmr_penrose}. The derivatives $\dot{\rho}_{(i)}$ and $\dot{z}_{(i)}$ are evaluated at the break-up point.}
  \label{tab:cmmr_penrose_example}
\end{table}

\section{Non-Stationary Spacetimes}
\label{ch:penrose_binaries:sec:arbitrary_penrose}
In the previous sections, all of our considerations depended upon the existence of a conserved negative and ``global'' (as seen by a static observer at infinity) energy. This occurs if the spacetime under consideration is stationary. In this section we will demonstrate how one can still observe the Penrose Process even when there is no global and conserved energy available, looking only at locally defined quantities. This technique allows one to study the Penrose mechanism even when the spacetime metric is defined numerically, such as is the case in numerical simulations of binary black hole collisions.

\subsection{3+1 split of the geodesic equation}

To understand our proposed technique, it is first fundamental to understand how General Relativity can be reformulated by explicitly separating its spatial and temporal components. This decomposition know as a 3+1 split is very commonly used in Numerical Relativity and was motivated by the first attempts of posing GR as a Cauchy problem and numeric spacetime metrics are often given in terms of it's 3+1 components. In this section, we will assume familiarity of the reader with this concept which can be readily reviewed in Refs.~\cite{Alcubierre2012-xp, 9780521514071, 9781108928250}. Given that the Penrose process requires us to investigate the trajectories of particles in a background spacetime, we must now solve the geodesic equation taking into account that the spacetime metric (and it's derivatives) will be provided via 3+1 split components. The need to solve the geodesic equation in this context overlaps with works that are interested in simulating an image of a black hole, that is, what a camera would capture if a picture of a BH was to be taken. In this type of simulation a technique called \emph{backwards ray tracing} is employed, which consists in choosing a position and orientation of a model camera and integrating the trajectory of the photons that hit the camera's ``film'' backwards in time. If the photons fall in the black hole, that pixel of the image will be black or colored otherwise.

Although our purposes differ, the mathematical tools used in backwards ray tracing are fundamental in our work. The complete and detailed derivation of the 3+1 split of the geodesic equation can be found in Ref.~\cite{Vincent_2012}. We shall recover only the nomenclatures and concepts necessary for the further development of our proposal. To begin with, we consider the background spacetime of interest to be described by a metric tensor $\mtrtens{\mu}{\nu}$ and to be globally hyperbolic, and thus admits a one parameter space-like foliation of constant coordinate time $t$ hypersurfaces that we shall denote by $\Sigma_t$. We will also assume that the space-time has coordinates\footnote{We use the convention that Greek indices run over all 4 coordinates while Latin indices run over only the spatial coordinates.} that are compatible with the foliation, that is, $x^0=t$ and $x^i$ span $\Sigma_t$. Let us denote the unit time-like (future directed) normal vector of $\Sigma_t$ by $n^\mu$. This vector coincides with the 4-velocity of an observer whose worldlines are orthogonal to $\Sigma_t$, which we call the \emph{Eulerian Observer} $\mathcal{O}_E$. We denote by $\gamma_{\mu\nu}$ the spatial metric induced in $\Sigma_t$, $D_i$ it's associated covariant derivative, $K_{ij}$ the extrinsic curvature tensor, $N$ the lapse function and $\beta^\mu$ the shift vector. The 3+1 split metric is thus
%
\begin{equation}
  \ud s^2 = -N^2 \ud t^2 + \gamma_{ij}(\ud x^i + \beta^i \ud t)(\ud x^j + \beta^j \ud t).
  \label{eq:arbitrary_penrose_decomposed_metric}
\end{equation}

Let us now consider a particle $\mathcal{P}$ of 4-momentum $p^\mu$. Let us assume that the particle moves in either a time-like or null geodesic (and thus without the influence of any force but gravity), which implies that
%
\begin{equation}
  p_\mu p^\mu = \left\{
  \begin{array}{lr}
    -m^2, & \text{if the particle is massive}  \\
    0,    & \text{if the particle is a photon}
  \end{array}
  \right. .
  \label{eq:arbitrary_penrose_p_norm}
\end{equation}
%
The 4-momentum can be decomposed as
%
\begin{equation}
  p^\mu = E(n^\mu + V^\mu)
  \label{arbitrary_penrose_p_decomp}
\end{equation}
%
in which $E$ represents the particle's energy as measured $\mathcal{O}_E$ (which means that $E = - n_\mu p^\mu$) and $V^\mu$ represents the 3-velocity of the particle, also as measured by $\mathcal{O}_E$. The 3-momentum $P^\mu$ of $\mathcal{P}$ as observed by $\mathcal{O}_E$ is thus
%
\begin{equation}
  P^\mu \equiv \tens{\gamma}{\mu}{\nu} p^\nu = E V^\mu
  \label{arbitrary_penrose_3_momentum}
\end{equation}
%
and the normalization of $p^\mu$, together with $n_\mu n^\mu = -1$ and Eq.~\eqref{arbitrary_penrose_p_decomp} imposes
%
\begin{equation}
  V_\mu V^\mu = V_i V^i \left\{
  \begin{array}{lr}
    = 1, & \text{if the particle is massive}  \\
    < 1, & \text{if the particle is a photon}
  \end{array}
  \right. .
  \label{eq:arbitrary_penrose_V_norm}
\end{equation}

Parametrizing the particle's position vector $X^i$ by the coordinate time (that is $x^i = X^i(t)$) the geodesic equation of $\mathcal{P}$ is decomposed in a set of 7 equations, namely
%
\begin{align}
  \der{X^i}{t} & = N V^i - \beta^i \label{eq:arbitrary_penrose_geodesic_eq_X}                                                                                                                                                  \\
  \der{V^i}{t} & = N V^j\left[ V^i \left( \partial_j \ln N - K_{jk} V^k \right) + 2 \tens{K}{i}{j} - {}^3\Gamma^{i}_{jk}V^k\right] - \gamma^{ij}\partial_j N - V^j\partial_j\beta^i \label{eq:arbitrary_penrose_geodesic_eq_V} \\
  \der{E}{t}   & = E (N K_{jk} V^j V^k - V^j \partial_j N) \label{eq:arbitrary_penrose_geodesic_eq_E}
\end{align}
%
that can be solved once initial positions, velocities and energy are supplied.

\subsection{Global energy}

Eq.~\eqref{eq:arbitrary_penrose_geodesic_eq_E} gives us the energy as measured by the Eulerian Observer. From here on, we will refer to this quantity as the \emph{local energy}, since it is measured locally by the observer that is orthogonal to the foliation. Up until now, we've been dealing with what we shall now call the \emph{global energy}, that is, the energy as measured by a static observer at infinity. To study the Penrose process in this context, we must bridge the gap between these two energy quantities. Let us proceed by writing down the Lagrangian of a particle moving about a 3+1 decomposed spacetime. By virtue of Eq.~\eqref{eq:arbitrary_penrose_decomposed_metric} we get
%
\begin{equation}
  \mathcal{L} = \frac{1}{2} \left\{ \left[ -N^2 + \gamma_{ij}\beta^i\beta^j \right]\dot{t}^2 + 2 \gamma_{ij}\beta^i\dot{x}^j \dot{t} + \gamma_{ij}\dot{x}^i\dot{x}^j\right\}.
  \label{eq:arbitrary_penrose_lagrangian}
\end{equation}
%
where dots denote derivatives with respect to the proper time. If we define the global energy $\epsilon$ as
%
\begin{equation}
  \epsilon \equiv -\frac{\partial \mathcal{L}}{\partial \dot{t}}
  \label{eq:arbitrary_penrose_global_energy_def}
\end{equation}
%
we get from Eq.~\eqref{eq:arbitrary_penrose_lagrangian} that
%
\begin{equation}
  \epsilon = \left[ N^2 - \gamma_{ij}\beta^i\beta^j \right]\dot{t} - \gamma_{ij}\beta^i\dot{x}^j.
  \label{eq:arbitrary_penrose_global_energy_def_2}
\end{equation}
%
By noting that~\cite{Vincent_2012}
%
\begin{equation}
  \dot{t} \equiv \der{t}{\lambda} = \frac{E}{N},
  \label{eq:arbitrary_penrose_t_dot_relation}
\end{equation}
%
we can rewrite Eq.~\eqref{eq:arbitrary_penrose_global_energy_def_2} as
%
\begin{equation}
  \epsilon = \left[ N^2 - \gamma_{ij}\beta^i\beta^j \right]\frac{E}{N} - \gamma_{ij}\beta^i\dot{x}^j.
  \label{eq:arbitrary_penrose_global_energy_def_3}
\end{equation}
%
If we remember, from Ref.~\cite{Vincent_2012} that
%
\begin{equation}
  V^i = \frac{1}{N}\left( \der{x^i}{t} + \beta^i \right) \Rightarrow \der{x^i}{t} = N V^i - \beta^i,
  \label{eq:arbitrary_penrose_vi_def}
\end{equation}
%
we can write
%
\begin{equation}
  \dot{x}^i \equiv \der{x^i}{\lambda} = \der{x^i}{t} \der{t}{\lambda} = (N V^i - \beta^i ) \frac{E}{N}.
  \label{eq:arbitrary_penrose_dx_dlambda_def}
\end{equation}
%
Substituting these results back into Eq.~\eqref{eq:arbitrary_penrose_global_energy_def_3} and after some algebra, we finally get
%
\begin{equation}
  \epsilon = \left( N - \gamma_{ij}\beta^i V^j \right) E.
  \label{eq:arbitrary_penrose_global_energy_def_final}
\end{equation}

\subsection{Penrose process}

Previously, we were able to determine if energy was extracted by comparing energies the energy observed by a static observer at infinity. Since this quantity was conserved, it did not matter the exact place in which this comparison took place. In a more general scenario, the global energy may no longer be conserved, and local energy must be always positive. How can we then determine if energy extraction took place? There is an important consequence of Eq.~\eqref{eq:arbitrary_penrose_global_energy_def_final} to be observed if the spacetime metric is asymptotically flat, that is, if infinitely far away from the black hole (or holes) the spacetime becomes the Minkowski solution. In this case we get that $\gamma_{ij} \rightarrow \eta_{ij}$, $\beta^i \rightarrow 0$ and $N \rightarrow 1$ which implies that $\epsilon = E$.

Let us now suppose that a particle moves about a spacetime containing a coalescing BBH. Let us suppose that such particle comes from infinity and falls towards the BHs. Before setting the particle free, we record it's global and local energies, and assert that they are the same since the spacetime in question is asymptotically flat. As the particle falls, it's local energy increases as it gets closer to the event horizons. At a certain point outside either event horizon, let the particle break up into two other particles, like in the traditional Penrose mechanism. This time, however, we cannot assume that one of the fragments has negative energy, since we can only measure its local energy (which must always be positive). Nevertheless, let us assume that one of the fragments is absorbed by the BHs and the other escapes back to infinity. Once the returning particle reaches infinity we record it's global and local energies and assert that they are the same. If we compare the global energies at infinity of both the ingoing and outgoing particle, we must be able to ascertain weather or not energy was extracted via the Penrose mechanism.

In practice, what one does is to choose a background spacetime metric and solve Eqs.~\eqref{eq:arbitrary_penrose_geodesic_eq_X}-\eqref{eq:arbitrary_penrose_geodesic_eq_E} numerically with a set of initial conditions $X^i(0), V^i(0), E(0)$. We evolve the system until the particle gets absorbed by one of the black holes or reaches a sphere of predetermined radius that we denominate the \emph{background sphere}. Since it is not possible to integrate the trajectory to spatial infinity, the radius of this background sphere must be large enough that the difference becomes smaller than a certain threshold $T_E$, that is, $|E(t_f)- \epsilon(t_f)| < T_E$, where $t_f$ represents the final integration coordinate time. Furthermore, we consider a particle to be absorbed by the system at a given time of swallowing $t_S$, if $|E(t_S) - E(0)|/E(0) = T_S$ where $T_S$ is an arbitrary swallowing threshold. These criteria are consistent with the ones introduced in Ref.~\cite{Bohn:2014xxa}.

The scheme described above was implemented in a public \texttt{C++} code available in Ref.~\cite{GRLensingRepo}. Compilation and usage instructions are provided within the repository. The code was originally created with the intent of producing Gravitational Lensing images (hence the name) but was later adapted to investigate the Penrose process. The code consists of a \emph{kernel} using the ARKODE~\cite{ARKODE} infrastructure responsible for integrating the geodesic equation and verifying stop criteria for an arbitrary spacetime metric. The kernel expects only to receive a set of functions that compute the ADM quantities of the spacetime metric (lapse, shift, extrinsic curvature and a few derivatives of these objects). Each spacetime metric is then a \emph{plugin} (a dynamic library) provides such functions at runtime. This allows the integration problem be separated of the problem of determining the ADM quantities of a given spacetime metric. Users can focus on the latter, since the kernel solves the former for an arbitrary input metric. In addition to spacetime metrics being plugins, file writers are also plugins in the same way. This allows the user to implement different output data formats for the trajectories according to their needs. Currently, only a simple \texttt{ASCII} file writer is available, but that is enough for our  in this section. The behavior of the program is driven at runtime by command line arguments, \texttt{YAML} configuration files and the available dynamic libraries implementing spacetime metrics and writers. A basic configuration file named \texttt{grlensing\_config.yaml} is expected to be present, detailing various ARKODE internal settings, such as error tolerances, max. number of iterations, etc. It also details the metric and file writer plugins to be loaded and made available and settings of the \texttt{dump-metric} command. Additional configuration files are required in certain modes (for instance, describing a certain spacetime parameters and initial conditions of a particle). Through the command line, the user selects the operation mode of the code. Currently, the available modes of operation are (these can be viewed by invoking the program with \texttt{--help})
%
\begin{itemize}
  \item \texttt{list-plugins}: Lists all the plugins that are set to be loaded (and were found)
  \item \texttt{dump-metric}: Writes a spacetime metric in a cube of arbitrary size and arbitrary number of internal points. This option is used mostly for debugging the implementation of a spacetime metric plugin.
  \item \texttt{integrate-trajectory}: Integrates a single particle trajectory with the specified configurations.
  \item \texttt{penrose-breakupu}: Integrates a particle breakup process by using two particle configurations and obtaining a third from conservation of 4-momentum.
\end{itemize}
%
The general usage pipeline, involves finding interesting single particle trajectories and feeding them into the \texttt{penrose-breakup} mode. Several utility scripts are provided in the program repository under the \texttt{resources} folder. These scripts serve multiple purposes, from plotting trajectories, energies, the ADM quantities of dumped spacetime metrics or even generating a skeleton of a metric plugin that can be filled by users according to their interests. It also includes an assortment of papers and notes required in the development of the code.

\subsubsection{Kerr spacetime (calibration)}

To illustrate the ideas discussed thus far, we shall analyze a particle breakup and Penrose mechanism in the Kerr spacetime. This shall serve as a test for the code and is also useful to illustrate the main physical points made.

%\begin{equation}
%  \label{eq:arbitrary_penrose_}
%\end{equation}
%


\myclearpage
\par

\chapter{Numerical Scalar Wave Scattering in GW150914}
\label{ch:wave_scattering}
In this chapter, we will focus on the numerical simulation of a scalar wave scattering off a black hole binary that mirrors the first event detected by LIGO, know as GW150914. Due to the already discussed challenges imposed by the strong gravity and the highly dynamic nature of astrophysical binary systems, to accurately simulate wave scattering around black hole binaries, it is necessary to solve Einstein's field equations numerically. In recent years, significant progress has been made in the numerical simulation of black hole binaries thanks to state-of-the-art strongly hyperbolic reformulations of the field equations, such as the BSSN formulation~\cite{PhysRevD.52.5428,PhysRevD.59.024007} and advancements in the understanding of stability criteria in partial differential equation discretization schemes employing techniques such as summation by parts~\cite{Diener2007}. These techniques have enabled us to simulate the dynamics of black hole binaries and their gravitational wave emission with unprecedented accuracy, stability and reliability.

Numerical evolution through self-coding can be a daunting task. This is where the employment of pre-existing tools, such as the \texttt{EinsteinToolkit}~\cite{EinsteinToolkit:2022_11}, becomes beneficial. The \texttt{EinsteinToolkit} is, according to its home page, ``a community-driven software platform of core computational tools to advance and support research in relativistic astrophysics and gravitational physics.''. The \texttt{EinsteinToolkit} is composed of a variety of software modules, referred to as \texttt{Thorns}, that are integrated into the \texttt{Cactus}~\cite{Cactuscode:web,Cactusprize:web,Goodale:2002a} computational framework. Each \texttt{Thorn} has the ability to perform a range of tasks, including infrastructure tasks such as file input/output, adaptive mesh refinement (AMR), as well as physics-related tasks such as the evolution of binary systems, the extraction of gravitational waves, and the multipole decomposition of a signal.

In the following sections, I will introduce the equations utilized for the evolution of a scalar wave atop a numerically metric without back-reaction, where the spacetime evolution disregards the scalar field's contribution to its stress-energy-momentum tensor. Subsequently, I will provide a detailed description of the \texttt{EinsteinToolkit Thorn KleinGordon} responsible for this evolution, and present the results that have been obtained thus far. The scalar field evolution code was developed by the author and its available, together with useful resources, in the \texttt{GitHub} repository~\cite{FieldPerturbationsRepo}. It is hoped that these results will offer valuable insights into the dynamics of the relaxation of astrophysical binary systems when perturbed.

\section{Evolution System}
\label{ch:wave_scattering:sec:system}
The formulation employed in this chapter was initially developed and published in Ref.~\cite{PhysRevD.96.104040}. Although the primary focus of that publication differs from the subject matter of this chapter, Appendix A of the paper contains the 3+1 decomposed Klein-Gordon equation for a massive scalar field, as presented in equations A3c and A3d. For the sake of completeness, these equations are
%
\begin{align}
  \partial_t \Phi =   & -2 \alpha K_\phi + \mathcal{L}_\beta \Phi \label{eq:wave_scattering_a3c}                                                                                                                                                     \\
  \partial_t K_\Phi = & \alpha \left( K K_\Phi - \frac{1}{2} \gamma^{ij} D_i \partial_j \Phi + \frac{1}{2} \mu^2 \Phi \right) - \frac{1}{2} \gamma^{ij} \partial_i \alpha \partial_j \Phi + \mathcal{L}_\beta K_\Phi, \label{eq:wave_scattering_a3d}
\end{align}
%
where $\Phi$ is the scalar field being evolved, $K_\Phi$ its ``canonical momentum'', given by
%
\begin{equation}
  K_\Phi = -\frac{1}{2\alpha} \left( \partial_t - \mathcal{L}_\beta \right)\Phi,
  \label{eq:wave_scattering_a2}
\end{equation}
%
$\alpha$ is the spacetime lapse, $\beta^i$ its shift vector, $\gamma^{ij}$ its induced 3-metric, $K$ the trace of its extrinsic curvature tensor, $D_i$ the covariant derivative associated with $\gamma^{ij}$, $\mu$ the spacetime mass parameter and $\mathcal{L}_\beta$ is the Lie derivative along the shift vector $\beta^i$, given explicitly by
%
\begin{equation}
  \mathcal{L}_\beta\Phi = \beta^i \partial_i \Phi.
  \label{eq:wave_scattering_lie_derivative}
\end{equation}

To expand each component of these equations, we utilized the mathematical software package \texttt{Wolfram Mathematica}. The resulting expansion was subsequently implemented into the \texttt{Thorn} by automatically generating corresponding \texttt{C} code from within \texttt{Mathematica}. The code used in this process can be found in \texttt{Notebooks/equations.nb} of the \texttt{Thorn}'s repository.

\section{Multipatch coordinates}
\label{ch:wave_scattering:sec:multipatch}
In order to simulate wave scattering phenomena in binary black hole systems accurately, it is imperative to extract the wave components that are far away from the sources. This is necessary to ensure precise modeling of real-world detectors and to minimize the impact of numerical reflections that occur when multiple grid resolutions are used in adaptive mesh refinement (as is the case in our simulation) and near the boundaries of the computational domain.

To achieve this, it is essential to locate a radius $r_d$ that is sufficiently far from the computational boundary radius $r_b$, such that incoming data from the boundaries does not interfere with the simulated ``measurement'' of the signal. Figure \ref{fig:wave_scattering_multipatch_signal} is a schematic representation of these desired distance relationships, adapted from Ref.~\cite{Reisswig2010}.

\begin{figure}[h]
  \centering
  \includesvg[width=\linewidth]{img/wave_scattering/multipatch_signal.svg}
  \caption{Schematic representation of distance relationships and data propagation in a \ac{BBH} simulation. Adapted from Ref.~\cite{Reisswig2010}.}
  \label{fig:wave_scattering_multipatch_signal}
\end{figure}

The requirement to position the computational boundary at a significant distance from the measurement location exposes a limitation of utilizing a solitary cubical Cartesian domain for simulation. To explicitly demonstrate this limitation, let us examine a simulation performed within a cubical domain that is mapped with Cartesian coordinates, and extends from $-L$ to $L$ in all three spatial dimensions, while containing $N_A$ grid points and uniform grid spacing $h=2L/N$. The total number of points $N_{PA}$ encompassed within this domain, which must be stored in the computer's memory for the simulation to proceed, is determined by
%
\begin{equation}
  N_{PA} = \left( \frac{2L}{h} \right)^3 = N_A^3.
  \label{eq:wave_scattering_npa}
\end{equation}

Let us consider the scenario where the size of the domain is increased by a factor of $\delta$ in each dimension, resulting in each coordinate falling within the range of $[-L-\delta, L+\delta]$. In order to preserve a constant grid spacing $h$, the number of grid points required in this configuration, denoted as $N_B$, must be determined by
%
\begin{equation}
  N_{B} = \frac{N_A(L + \delta)}{L}
  \label{eq:wave_scattering_nb}
\end{equation}
%
and thus the total number of points $N_{PB}$ encompassed within this domain is given by
%
\begin{equation}
  N_{PB} = \left(\frac{N_A(L + \delta)}{L}\right)^3
  \label{eq:wave_scattering_npb}
\end{equation}

As an illustrative example, let us consider an initial grid that spans $[-1,1]$ with $N_A=20$, yielding a grid spacing of $h=0.1$. We will now increase the size of this grid by a factor of $\delta=9$, resulting in an expanded grid spanning $[-10,10]$. To maintain the grid spacing at $h=0.1$, we must select $N_B=200$ grid points. Considering that each numerical value stored in the simulation grid is represented by a 64-bit precision floating-point value requiring 8 B of storage, we can conclude that the memory requirements for each grid configuration are $N_{PA}=64$ KB and $N_{PB}=64$ MB of storage per grid function, respectively.

To provide a sense of scale, the entire evolved system for the first grid configuration (comprising two grid functions) could fit into a single 8-inch Memorex 650 floppy disk, the first commercial model made in the year 1972, whereas the second configuration would require approximately $370$ units of the same model to store a single grid function.

The phenomenon occurs due to the coupling of coordinates in Cartesian space, making it impossible to solely increase one dimension and achieve a radius expansion. In contrast, if one employs spherical coordinates, the situation is altered, as the angular and radial resolutions become uncoupled. As a result, only one dimension requires modification to attain a delta radius expansion, which is memory-efficient. Additionally, the utilization of spherical coordinates exploits the spherical symmetry present in \ac{BBH} merger processes.

Within the context of the \texttt{Einstein Toolkit}, the simulation domain's coordinates may be altered by utilizing the \texttt{Llama} infrastructure~\cite{Reisswig2010,PhysRevD.83.044045}. \texttt{Llama} is recognized as a \textit{multipatch system} which envelops the simulation domain with distinct coordinate patches. These patches are subsequently combined to describe the global simulation domain. The \texttt{Thornburg04} coordinate system is the chosen system for our simulations within \texttt{Llama}. It comprises six spherical wedge patches and an additional central Cartesian patch. It should be noted that the usage of the six spherical patches is to ensure coverage of the sphere without the introduction of coordinate singularities.

Fig.~\ref{fig:wave_scattering_multipatch_coords} depicts a schematic representation of a 2D slice of the \texttt{Thornburg04} used in our simulations. It is worth noting that in \texttt{Llama}, the coordinate patches overlap, allowing for a certain region of the domain to be covered by more than one patch at a time. This technical design choice facilitates data communication between patches, but is not essential to create a multipatch system. However, in Fig.~\ref{fig:wave_scattering_multipatch_coords}, the overlap between the central Cartesian patch and the surrounding spherical wedge patches is represented by a light blue region. Within the central Cartesian patch, additional mesh refinement may occur, as indicated in the figure by the two squares inside the central patch.

\begin{figure}[h]
  \centering
  \includegraphics[width=0.60\linewidth]{img/wave_scattering/multipatch_coords}
  \caption{Schematic representation of the \texttt{Thornburg04} coordinate system employed in the simulations. Adapted from Ref.~\cite{Reisswig2010}.}
  \label{fig:wave_scattering_multipatch_coords}
\end{figure}

To preserve the structure of the evolution equations, \texttt{Llama} employs both \textit{local} and \textit{global} coordinate systems. Equations and derivatives are consistently expressed in terms of their respective patch-local coordinates, enabling the use of standard forms of equations and approximations for derivative operators. However, \texttt{Llama} internally stores quantities in a Cartesian global coordinate system in which each patch is embedded. To convert local derivatives (and other vector or tensor quantities, if they exist) to this global coordinate system, it is necessary to project each derivative operator using the Jacobians of the coordinate transformations.

Assuming $x^k$ represents the global Cartesian coordinate system and $a^k$ denotes the local coordinates, the construction of the Jacobian $\tens{J}{i}{k} = \partial a^i / x^k$ permits the expression of first and second order derivatives in the global coordinate system as
%
\begin{equation}
  \hat{\partial}_k = \tens{J}{i}{k} \partial_i
  \label{eq:wave_scattering_projection_1}
\end{equation}
%
and
%
\begin{equation}
  \hat{\partial}_i \hat{\partial}_j = \tens{J}{k}{i} \partial_k \left( \tens{J}{l}{j} \right) \partial_l + \tens{J}{k}{i}\tens{J}{l}{j}\partial_k\partial_l,
  \label{eq:wave_scattering_projection_2}
\end{equation}
%
respectively, where $\partial_i$ denotes derivatives with respect to $a^i$ and $\hat{\partial}_i$ denotes derivatives with respect to $x^i$.

It should be noted that, in order to perform the required calculations, not only the Jacobian of local coordinates is necessary, but also the Jacobian derivatives. \texttt{Llama} provides both the components of the Jacobian and its derivatives at each point on the grid during the evolution process so that the user can project the results according to Eqs~\eqref{eq:wave_scattering_projection_1} and \eqref{eq:wave_scattering_projection_2} manually.

Alternatively, \texttt{Llama} provides a convenience \texttt{Thorn} called \texttt{GlobalDerivatives} that automatically performs the projections for the user. However, in our \texttt{Thorn}, we have opted to perform these projections manually, as \texttt{GlobalDerivatives} does not offer a way to directly compute second-order spatial derivatives. Once again, we have employed Wolfram Mathematica for generating the \texttt{C} code responsible for projecting derivatives into global coordinates.

\section{Initial Data}
\label{ch:wave_scattering:sec:id}
The implementation supports three distinct initial conditions for the scalar field and its canonical momentum. These include a plane wave, an exact Gaussian, and a multipolar Gaussian function. The plane wave and exact Gaussian initial data were developed for the purpose of testing multipatch derivatives in flat spacetime, which will be elaborated on further in subsequent sections. The exact Gaussian initial data is named as such to reflect the fact that it is an exact solution of the Klein-Gordon equation in flat spacetime.

The multipolar Gaussian function is a basic Gaussian function that is multiplied by a linear combination of real spherical harmonics. This particular function was the one chosen during our ``production'' runs as it has the ability to excite specific modes in the system based on the chosen parameters for the linear combination. The development of this initial data is based on the methodology presented in Ref.~\cite{PhysRevD.87.043513}. In order to construct it, let us first introduce its components. We start with the Gaussian function $G(r, \sigma)$, given by
%
\begin{equation}
  G(r, \sigma) = \exp\left( -\frac{1}{2} \left( \frac{r}{\sigma} \right)^2  \right).
  \label{eq:wave_scattering_gaussian}
\end{equation}
%
where $\sigma$ is the Gaussian width parameter. Next, we introduce the real spherical harmonics $Y_{lm}(x,y,z)$, given by
%
\begin{equation}
  Y_{lm}(x, y, z) =
  \begin{cases}
    A_{lm} P_{lm}\left( \frac{z}{\sqrt{x^2 + y^2 + z^2}} \right) \cos(m \arctan(y,x)) \text{ if } m \geq 0  \\
    A_{lm} P_{l|m|}\left( \frac{z}{\sqrt{x^2 + y^2 + z^2}} \right) \sin(|m| \arctan(y,x)) \text{ if } m < 0 \\
  \end{cases}
  ,
  \label{eq:wave_scattering_real_spherical_harmonics}
\end{equation}
%
where $P_{lm}(x)$ are the associated Legendre polynomials and the $A_{lm}$ coefficients are given by
%
\begin{equation}
  A_{lm} = (-1)^m \sqrt{\frac{2 l + 1}{4\pi} \frac{(l-m)!}{(l+m)!}}.
  \label{eq:wave_scattering_real_spherical_harmonics_coeffs}
\end{equation}
%
By introducing ``shifted coordinates''
%
\begin{align}
  X & \equiv x - x_0 \label{eq:wave_scattering_real_spherical_harmonics_shifted_x}                \\
  Y & \equiv y - y_0 \label{eq:wave_scattering_real_spherical_harmonics_shifted_y}                \\
  Z & \equiv z - z_0 \label{eq:wave_scattering_real_spherical_harmonics_shifted_z}                \\
  R & \equiv \sqrt{X^2 + Y^2 + Z^2} \label{eq:wave_scattering_real_spherical_harmonics_shifted_r} \\
\end{align}
%
it is possible to shift the location of the center of the function to $(x_0,y_0,z_0)$. Combining these pieces, the multipolar Gaussian function $M_G(x,y,z)$ is written as
%
\begin{equation}
  M_G(x, y, z) = \sum_{l=0}^{N}\sum_{m = -l}^{l} c_{l m} Y_{l m}(X,Y,Z) G(R-R_0,\sigma)
  \label{eq:wave_scattering_multipolar_gaussian}
\end{equation}
%
where $R_0$ is the radius of the Gaussian function. Once again, code generation routines for the initial data can be found in \texttt{Notebooks/equation.nb}. It is important to note that in the implemented code, the series is expanded up to $N = 2$, and higher multipole orders are not supported.

Figure~\ref{fig:multipolar_gaussian_id_demo} showcases three distinct configurations of the multipolar Gaussian function. For all panels, the parameters $x_0$, $y_0$, and $z_0$ were assigned a value of zero, while $R_0$ was set to 5. Panel \textbf{A} displays a plot where $c_{00}$ is assigned a value of $1/A_{00}$, and all other coefficients are set to zero. In Panel \textbf{B}, $c_{11}$ is set to $1/A_{11}$, and all other coefficients are set to zero. Finally, in Panel \textbf{C}, $c_{22}$ is assigned a value of $1/(3A_{22})$, and all other coefficients are set to zero.

\begin{figure}[!ht]
  \centering
  \includegraphics[width=\linewidth]{img/wave_scattering/multipolar_gaussian_id_examples.png}
  \caption{Demonstration of the multipolar Gaussian function with different parameters. For all panels, $x_0 = y_0 = z_0 = 0$ and $R_0 = 5$. In Panel \textbf{A} displays the only nonzero coefficient is $c_{00} = 1/A_{00}$, in Panel \textbf{B}, $c_{11} = 1/A_{11}$, and in Panel \textbf{C}, $c_{22} = 1/(3A_{22})$.}
  \label{fig:multipolar_gaussian_id_demo}
\end{figure}

\section{Evolution}
\label{ch:wave_scattering:sec:code}
To configure the evolution of Eqs.~\eqref{eq:wave_scattering_a3c} and \eqref{eq:wave_scattering_a3d} on top of the GW150914 event simulation within the \texttt{EinsteinToolkit}, the demonstration parameter file that reproduces this event, which can be found in Ref.~\cite{GW150914Demo}, was used as a basis and modified in order to accommodate the \texttt{KleinGordon} thorn. To set up the gravitational initial data, the \texttt{TwoPunctures Thorn}~\cite{Ansorg:2004ds} was utilized and spacetime evolution was performed using \texttt{ML\_BSSN}~\cite{Brown:2008sb,Kranc:web,McLachlan:web}.

As explained in Sec.~\ref{ch:wave_scattering:sec:multipatch}m the coordinate system used in our grid setup was the \texttt{Thornburg04} coordinates, implemented through the \texttt{Llama} infrastructure covering the region between $r_\text{min}$ and $r_\text{max}$, with smooth inner and outer boundaries, with an added a cubical patch enclosing $r < r_\text{min}$ where standard \texttt{Carpet}~\cite{Schnetter:2003rb} box-in-box mesh refinement can be applied.

We employed 7 levels of mesh refinement boxes via \texttt{Carpet}, centered around each of the BH centers. The motion of the BHs during the evolution was tracked via the \texttt{PunctureTracker Thorn}. Once the BHs were moved by a certain configurable threshold, \texttt{Carpet} ``regrided'' the simulation domain, that is, it updated the positions of the refinement boxes so that they keep following the BHs.

We have utilized the newly developed \texttt{KleinGordon Thorn}, to evolve a massless scalar field with initial data provided by a multipolar Gaussian function, centered around the origin with $R_0 = 15.0$, which is large enough to engulf both BHs in their initial configuration. Additionally, we set the Gaussian width $\sigma$ to $1.0$, and the only non-zero multipolar coefficient being $c_{00} = 1 / A_{00}$. This spherically symmetric initial data allowed us to activate reflection symmetry around the $+z$ plane in the code, which saved computational time, memory, and storage space.

As was explained in Sec.~\ref{ch:wave_scattering:sec:multipatch}, \texttt{KleinGordon} handcrafted 8th order central finite difference derivative operators were utilized and projected onto global coordinates utilizing the Jacobian and Jacobian derivative components provided by \texttt{Llama}. Artificial Kreiss-Oliger dissipation was added to the evolved Klein-Gordon fields via the \texttt{Dissipation Thorn}. Radiating boundary conditions were employed in the outer boundaries via the \texttt{NewRad Thorn}.

%In the gravitational evolution code, spatial derivatives were provided by the \texttt{SummationByParts Thorn}~\cite{Diener:2005tn}, which provides finite difference stencils satisfying the summation-by-parts property with embedded Kreiss-Oliger type artificial dissipation. In \texttt{KleinGordon}, we utilized handcrafted central finite difference stencils and added artificial Kreiss-Oliger dissipation to the evolved Klein-Gordon fields via the \texttt{Dissipation Thorn}.

The time integration scheme utilized in this was method of lines, provided by the \texttt{MoL Thorn}. A 4th order Runge-Kutta method with fixed time step was employed for the time integration process. It is important to note that \texttt{Carpet} employs different time steps for various refinement levels, thus the time step specified in the parameter files pertains to the coarsest level. The final integration time was selected in conjunction with the outer boundary radius to minimize interference from reflections at the outer boundaries, inter-patch boundaries, and inter-refinement level boundaries on the scalar field signal being measured.

At regular time intervals, data was extracted at $7$ different radii using \texttt{Multipole}, as well as 2D and 3D field data evolved by \texttt{KleinGordon}. Additionally, statistics such as puncture locations were recorded, which enabled the construction of visualizations of the simulated data. These results will be presented on the next section. The parameter file used for this evolution can be located in the specified GitHub repository under \texttt{KleinGordon/par/GW150914/GW150914\_Scalar\_field.par} in Ref~\cite{FieldPerturbationsRepo}.

\section{Results}
\label{ch:wave_scattering:sec:results}
The results depicted in Figure~\ref{fig:wave_scattering_results} illustrate the scattering of the scalar field $\Phi$ by the GW150914 binary black hole merger system at three distinct moments in time. Panel \textbf{A} corresponds to $t = 0$, panel \textbf{B} represents $t = 8.811$, and panel \textbf{C} corresponds to $t = 55.068$. Following the initial interaction, the black holes generate a wavefront in a pattern resembling a cardioid, where the cusp coincides with the hole location and propagates in alternating directions. This shape is identical to that observed in wave scattering by a single black hole, however, in the case of a merging binary, we can observe clear effects arising from the rotation of the holes and interference from the waves produced by each. An animation of the scattering process can be viewed \href{https://github.com/lucass-carneiro/phd-thesis/tree/main/img/wave_scattering/scattering.gif}{here}.

%Frames used: 0 7 19
\begin{figure}[h]
  \centering
  \includegraphics[width=\linewidth]{img/wave_scattering/scattering_frames}
  \caption{Scalar field $\Phi$ scattering of the GW150914 binary merger in three moments in time. Panel \textbf{A} corresponds to $t = 0$, panel \textbf{B} represents $t = 8.811$, and panel \textbf{C} corresponds to $t = 55.068$.}
  \label{fig:wave_scattering_results}
\end{figure}

In addition to pure field values, we have, as was previously mentioned, extracted the multipolar components of the field at various radii during the simulation. For clarity, we will focus on the $l=2,m=0$ component of the field at a radius of $40$ in this discussion. The plot in the top left panel of Fig.~\ref{fig:multipolar_plots} shows the field value as a function of time. The signal has a ``blip'' between $t=0$ and $t=50$ due to the fact that our multipolar Gaussian initial data splits in half at $t=0$, resulting in an outgoing component of the initial data the gets detected first by the decomposition. After $t=50$, we observe a nonlinear signal resulting from the field's interaction with the binary, which settles down after a period of oscillations.

It is also useful to analyze the logarithm of the absolute value of the field in order to identify regions where quasinormal ringdown might be taking place. This is what can be seen in the top right plot of Fig.~\ref{fig:multipolar_plots} between approximately $t=60$ to $140$. Additionally, in order to better visualize this decaying oscillation, we plot on the bottom panels of Fig.~\ref{fig:multipolar_plots} the same quantities as described on the top panels but in between $t=67.2$ and $t=136.5$. Even tough the signal is clearly oscillating in a damped manner, it is difficult to identify the shape of the plot of the bottom right panel to the classical shape of the quasinormal ringdown. There are very prominent deformations on the shapes of the oscillation ``crests'' and at times their periods seem to be of variable size.

\begin{figure}[h]
  \centering
  \includegraphics[width=\linewidth]{img/wave_scattering/multipolar_plots}
  \caption{Top panels: Plot of the multipolar field component $l=2,m=0$ at $r=40$ for the whole simulation time. Bottom panels: Same quantities as in the top panels but restricted to the $t=[67.2, 136.5]$ time period. The right panels plot the logarithm of the absolute value of the field.}
  \label{fig:multipolar_plots}
\end{figure}

The departure from the quasi-normal ringdown is more noticeable when attempting to fit the classical quasi-normal model, $A \exp(-\omega_i t) \cos(\omega_r t + \phi)$, to the data between $t=67.2$ and $t=136.5$. The left plot of Fig.~\ref{fig:multipolar_plots_fit} shows the residual of the fitting, i.e., the absolute value of the difference between the simulated data and the data obtained by substituting the fit parameters back into the model. Despite the quasi-normal ringdown model not agreeing well with the simulated data, the fitted frequency parameters are presented on the left plot for completeness.

\begin{figure}[h]
  \centering
  \includegraphics[width=\linewidth]{img/wave_scattering/fit_plot}
  \caption{Left panel: The original data is plotted against a fit to a quasinormal mode model. Fitted frequencies are displayed in the plot. Right panel: Plot of the absolute value of the difference between original and fitted data (fit residual). }
  \label{fig:multipolar_plots_fit}
\end{figure}

In conclusion, a spectral analysis of the signal is performed using the periodogram in Fig.~\ref{fig:multipolar_plots_psd}, which displays the Power Spectral Density (PSD) of each frequency component of the entire simulated signal. The plot also highlights the peak frequency at approximately $f\approx0.13$.

\begin{figure}[h]
  \centering
  \includegraphics[width=\linewidth]{img/wave_scattering/psd_plot}
  \caption{Periodogram of the simulated signal with peak frequency indicated in the plot.}
  \label{fig:multipolar_plots_psd}
\end{figure}

\myclearpage
\par

\chapter{Quasinormal Modes and the Asymptotic Iteration Method}
\label{ch:qnm_aim}
Moving forward with the discussion of quasinormal modes, this section will describe the Asymptotic Iteration Method (\ac{AIM}) and the published implementation of it, the \texttt{QuasinormalModes.jl} Julia package.

The chapter will commence with a portrayal of the equations regulating perturbations of diverse spins on a Schwarzschild background. Subsequently, details of the \ac{AIM} and the construction of \texttt{QuasinormalModes.jl} will be provided. Finally, the results obtained using the package to determine the quasinormal modes of the various spin perturbations established at the start of the chapter will be demonstrated. Each value will be computed utilizing literature values, when possible, and results from an independent implementation of the pseudospectral method for the same problems. Moreover, newly determined frequencies will be reported using the package.

\section{The asymptotic iteration method and \texttt{QuasinormalModes.jl}}
\label{ch:qnm_aim:sec:aim}
Here we shall briefly review the mathematical foundations of the \ac{AIM} following closely Ref.~\cite{aim_original}. Let us suppose that exists a variable $x \in [a,b]$ where $a,b \in \mathbb{R}$ and functions $\lambda_i = \lambda_i(x) \in \mathbb{R}$ and $s_j = s_i(x) \in \mathbb{R}$ with integer indexes $i$ and $j$ that are $C_\infty(a,b)$. Let us also suppose that there is a function $y=y(x)\in\mathbb{R}$ that satisfies
%
\begin{equation}
  y^{(2)}(x) - \lambda_0(x) y^{(1)}(x) - s_0(x)y(x) = 0
  \label{eq:aim_general_ode}
\end{equation}
%
where the parenthesized superscript denotes $n$ derivatives with respect to the variable $x$. These equations can be found in many areas of physics, such as the time-independent Schr\"odinger equation in Quantum Mechanics, or the differential equations governing the perturbations of a Schwarzschild black hole. The \ac{AIM} is based upon the following theorem:

\begin{theorem}
  The differential equation~\eqref{eq:aim_general_ode} has a general solution of the form
  %
  \begin{equation}
    y(x) = \exp\left( -\int^x\alpha\ud t \right) \left\{ C_2 + C_1 \int^{x} \exp \left[ \int^{t} ( \lambda_0(\tau) + 2\alpha(\tau) )\ud \tau \right] \ud t \right\}
    \label{eq:aim_general_solution}
  \end{equation}
  %
  if for some $n>0$ the condition
  %
  \begin{equation}
    \alpha(x) \equiv \frac{s_n(x)}{\lambda_n(x)} = \frac{s_{n-1}(x)}{\lambda_{n-1}(x)}
    \label{eq:aim_alpha_definition}
  \end{equation}
  %
  or equivalently
  %
  \begin{equation}
    \delta(x) \equiv s_n(x)\lambda_{n-1}(x) - \lambda_{n}(x)s_{n-1}(x) = 0
    \label{eq:aim_delta_definition}
  \end{equation}
  %
  is satisfied, where
  %
  \begin{align}
    \lambda_k(x) & \equiv \lambda^{(1)}_{k-1}(x) + s_{k-1}(x) + \lambda_0(x)\lambda_{k-1}(x) \label{eq:aim_lambda_k} \\
    s_k(x)       & \equiv       s^{(1)}_{k-1}(x) + s_0(x)\lambda_{k-1}(x) \label{eq:aim_sk}
  \end{align}
  %
  with $k \in [1, n]$
  \label{theo:aim_theorem}
\end{theorem}
%
From now on, we shall refer to the condition expressed by Eq.~\eqref{eq:aim_delta_definition} as the \emph{\ac{AIM} quantization condition}. Provided that Theo.~\ref{theo:aim_theorem} is satisfied we can find both the eigenvalues and eigenvectors of Eq.~\eqref{eq:aim_general_ode} using, respectively, Eq.~\eqref{eq:aim_delta_definition} and Eq.~\eqref{eq:aim_general_solution}. More specifically, the quasinormal modes of a perturbed black hole will be the complex frequency values $\omega$ that satisfy Eq.~\eqref{eq:aim_delta_definition} for any value of $x$. Recently, it was shown in Ref.~\cite{Ismail2020} that for the method to converge, one must have
%
\begin{equation}
  \lim_{n \rightarrow \infty} \frac{\delta_n(x)}{\lambda_{n-1}^2(x)} = 0
  \label{eq:aim_convergence_criteria}
\end{equation}

Despite being quite general, the method presents a computational difficulty hidden in Eq.~\eqref{eq:aim_lambda_k} and Eq.~\eqref{eq:aim_sk}. The definitions of the $n$-th coefficients are coupled, recursive and involve the derivatives of previous entries. This means that to compute the quantization condition, Eq.~\eqref{eq:aim_delta_definition}, using $n$ iterations we end up computing the $n$-th derivatives of $\lambda_0$ and $s_0$ multiple times. Depending on the size of the original functions, the size and complexity of each coefficient can quickly spiral out of control as $n$ is increased. To address these issues, Cho et al. have proposed in Ref.~\cite{aim_improved} to instead of computing these coefficients directly, use a Taylor expansion of both $\lambda_n(x)$ and $s_n(x)$ around an arbitrary point $\xi$ where the \ac{AIM} is to be performed, thus introducing a new free parameter to the method. We, however, remind the reader that the results must be independent of the choice of $\xi$. Mathematically, we have
%
\begin{align}
  \lambda_n(\xi) = & \sum_{i=0}^{\infty}c^{i}_n(x - \xi)^i, \label{eq:taylor_lambda0} \\
  s_n(\xi) =       & \sum_{i=0}^{\infty}d^{i}_n(x - \xi)^i, \label{eq:taylor_s0}
\end{align}
%
where $c^i_n$ and $d^i_n$ are the Taylor coefficients of the expansions of $\lambda_n$ and $s_n$ around $\xi$, respectively. By plugging Eqs.~\eqref{eq:taylor_lambda0} and \eqref{eq:taylor_s0} into Eqs.~\eqref{eq:aim_lambda_k} and Eq.~\eqref{eq:aim_sk} one gets

\begin{align}
  c^i_n = & (i+1)c^{i+1}_{n-1} + d^i_{n-1} + \sum_{k=0}^{i}c^k_0c^{i-k}_{n-1}, \label{eq:cin_def} \\
  d^i_n = & (i+1)d^{i+1}_{n-1} + \sum_{k=0}^{i}d^k_0c^{i-k}_{n-1}. \label{eq:din_def}
\end{align}
%
Finally, using Eqs.~\eqref{eq:cin_def} and \eqref{eq:din_def} the quantization condition, Eq.~\eqref{eq:aim_delta_definition}, becomes
%
\begin{equation}
  \delta \equiv d^0_n c^0_{n-1} - d^0_{n-1}c^0_n = 0.
  \label{eq:improved_delta}
\end{equation}

In order to better visualize and understand the improved algorithm, it is useful to arrange the $c^i_n$ (or $d^i_n$) coefficients as elements of a matrix $C$ (or $D$), where the index $i$ indicates the matrix row and the index $n$ represents the matrix column. To aid in our visualization, let us also assume, without loss of generality, that we have chosen to perform the \ac{AIM} with $n=2$. According to Eq.~\eqref{eq:improved_delta}, the largest $n$ coefficients that need to be computed  are $d^0_2$ and $c^0_2$. These coefficients need, in turn, to be computed recursively via Eqs.~\eqref{eq:cin_def} and \eqref{eq:din_def}. This process was represented in Fig.~\ref{fig:aim_coeffs_c} for $c^0_2$. Each row in the figure represents a step in the algorithm. A red circle marks the coefficient that is being calculated at the given step and a blue circle with arrows mark the coefficients that are necessary for the calculation. We remind that the first column of the matrices, that is, $c^i_0$ and $d^i_0$, are computed directly from $\lambda_0(x)$ and $s_0(x)$ from their Taylor expansions. Note that the lower right coefficients of the $c^i_n$ matrix, that is, $c^1_1$, $c^2_2$ and $c^2_1$ are never used in any step. Since these coefficients are not required, they need not be computed, saving time in the algorithm.

\begin{figure}[!ht]
  \centering
  \includesvg[width=\linewidth]{img/aim_qnm/aim_coeffs.svg}
  \caption{Schematic representation of the \ac{AIM} for computing the $c^i_n$ matrix. Each row of matrices represent an \ac{AIM} step. Coefficients circled in red are currently being computed, while coefficients marked in blue withe arrows are the coefficients required for that computation. Notice how the lower-right triangle of coefficients is never used.}
  \label{fig:aim_coeffs_c}
\end{figure}

In  Fig.~\ref{fig:aim_coeffs_d}, we see a similar representation, but now for the $d^i_n$ coefficients. The first two rows of the image represent the steps required for computing $d^0_2$. Notice however, that $d^0_1$ (see the third row in Fig.~\ref{fig:aim_coeffs_d}) is not explicitly required for the computation of the target coefficients, but it is required for the computation of $c^0_2$ and can be readily calculated since it depends only on the initial Taylor expansion of the \ac{ODE} coefficients. Similarly, coefficients $d^1_2$, $d^2_2$ and $d^2_1$ are never used and thus do not need to be computed.

\begin{figure}[!ht]
  \centering
  \includesvg[width=\linewidth]{img/aim_qnm/aim_coeffs_d.svg}
  \caption{Schematic representation of the \ac{AIM} for computing the $d^i_n$ matrix. The first two row of matrices represent \ac{AIM} steps and the third row represents the computation of $d^o_1$, which is required in computing $c^0_2$. Coefficients circled in red are currently being computed, while coefficients marked in blue withe arrows are the coefficients required for that computation. Notice how the lower-right triangle of coefficients is never used.}
  \label{fig:aim_coeffs_d}
\end{figure}

These observations motivate us to see the \ac{AIM} algorithm as an ``evolution'' of the initial coefficient sets $c^i_0$ and $d^i_0$ by rewriting Eqs.~\eqref{eq:cin_def} and \eqref{eq:din_def} as
%
\begin{align}
  c^i_{n+1} = & (i+1)c^{i+1}_n + d^i_n + \sum_{k=0}^{i}c^k_0c^{i-k}_n, \label{eq:cin_iterative} \\
  d^i_{n+1} = & (i+1)d^{i+1}_n + \sum_{k=0}^{i}d^k_0c^{i-k}_n, \label{eq:din_iterative}
  .
\end{align}
%
We can now devise an algorithm that performs $n$ iterations of the \ac{AIM}:
%
\begin{enumerate}
  \item Construct two arrays of size $n$ where the $i$-th element is $c^i_0$ (or $d^i_0$) where $i$ ranges from zero to $n$. We shall call these \texttt{icda} (initial $c$ data array) and \texttt{idda} (initial $d$ data array).

  \item Construct two arrays of size $n$ to contain the current column of $c$ (or $d$) indexes. We shall call these \texttt{ccda} (current $c$ data array) and \texttt{cdda} (current $d$ data array)

  \item Construct two arrays of size $n$ to contain the previous column of $c$ (or $d$) indexes. We shall call these \texttt{pcda} (previous $c$ data array) and \texttt{pdda} (previous $d$ data array).

  \item Initialize \texttt{ccda} with data from \texttt{icda} and \texttt{cdda} with data from \texttt{idda}.

  \item Perform $n$ \ac{AIM} steps using the evolution Eqs.~\eqref{eq:cin_iterative} and \eqref{eq:din_iterative}. That is, repeat the following $n$ times:
        \begin{enumerate}
          \item Copy the content from \texttt{ccda} into \texttt{pcda}
          \item Copy the content from \texttt{cdda} into \texttt{pdda}
          \item Rewrite each element of \texttt{ccda} and \texttt{cdda} using Eqs. \eqref{eq:cin_iterative} and \eqref{eq:din_iterative}, respectively.
        \end{enumerate}

  \item Compute the quantization condition, Eq.~\eqref{eq:improved_delta}, using the first indexes of each array. Explicitly, perform \texttt{cdda[1]*pcda[1] - pdda[1]*ccda[1]}\footnote{We assume 1-base array indexing, the same scheme adopted by the \texttt{Julia}.}.

  \item If the coefficients are analytic, determine the roots of the resulting expression, otherwise use steps 1-6 to build a function that returns $\delta$ numerically with a given parameter set and use a numerical root finding method to find the roots of this function.
\end{enumerate}
%
The algorithm steps are depicted in Fig.~\ref{fig:arrays_steps} for the $n=2$ example. Each array is depicted as a sequence of blue (for storing $c^i_n$ coefficient) and red (for storing $d^i_n$ coefficients) squares, wherein each square is an array element. There are three columns of arrays, each representing, respectively, initial, current and previous data at various points in the algorithm. Each row indicates the algorithmic step that it represents to the left of the data arrays and which \ac{AIM} step ($n$ value) a set of steps corresponds to. On step 5 (c), colored arrows indicate the data dependency of each index in the current arrays (similarly to what is depicted in Figs.~\ref{fig:aim_coeffs_c} and \ref{fig:aim_coeffs_d}). Hatches in array indexes represent data that is not evolved/computed.
%
\begin{figure}[!ht]
  \centering
  \fontsize{9}{10}\selectfont
  \includesvg[width=\linewidth]{img/aim_qnm/arrays.svg}
  \caption{Representation of \ac{AIM} steps for an $n=2$ sized example. Arrays are represented by a sequence of blue (for storing $c^i_n$ coefficient) and red (for storing $d^i_n$ coefficients) squares, wherein each square is an array element. Each column represents, respectively, initial, current and previous data. Colored arrows indicate index dependencies. Hatches indicate ignored data.}
  \label{fig:arrays_steps}
\end{figure}

The implementation in \texttt{QuasinormalModes.jl} closely follows the description provided thus far, using additional coefficient arrays to enable multi-threading in the \ac{AIM} core loop in a thread-safe manner. The online documentation for the package includes extensive tutorials, usage examples, and package documentation, which are available at Ref.~\cite{QuasinormalModesDocs}. The code for the package is hosted on GitHub and can be found in Ref.~\cite{QuasinormalModesRepo}. The package is registered on the \texttt{Julia} package index and can be easily installed by following the instructions provided in the \texttt{README} of the GitHub repository.

We have developed a script to evaluate the performance and convergence characteristics of the QuasinormalModes.jl package. The script conducts a fundamental $s=l=0$ mode computation of a Schwarzschild black hole while sweeping the number of \ac{AIM} iterations from $n=1$ to $n=100$. The error of the computation, the time taken to compute the result, and the number of iterations performed are measured 20 times and stored in distinct text files to enhance the statistical significance of the time measurements. The error is defined as the discrepancy between the \ac{AIM}-computed value and values acquired in Ref.\cite{BertiQNMData}, which were obtained via Leaver's continued fraction method. We present the error convergence outcomes in Fig.\ref{fig:package_error} using a logarithmic scale in the $y$ axis for both the real (black dots) and imaginary (red crosses) components of the mode as a function of the number of \ac{AIM} iterations performed. Convergence towards the reference values can be observed. We depict the average time taken for 20 repetitions as a function of the number of iterations in Fig.~\ref{fig:package_perf} using a log-log plot. It is possible to observe that the time taken increases with a power law trend as the number of \ac{AIM} iterations increases.

\begin{figure}[!ht]
  \centering
  \includegraphics[width=\linewidth]{img/aim_qnm/err.pdf}
  \caption{Error of the fundamental $s=l=0$ Schwarzschild \ac{QNM} vs. the number of \ac{AIM} iterations. Reference values obtained in Ref.~\cite{BertiQNMData}. The y-axis of the plot is in logarithmic scale.}
  \label{fig:package_error}
\end{figure}

\begin{figure}[!ht]
  \centering
  \includegraphics[width=\linewidth]{img/aim_qnm/perf.pdf}
  \caption{Time taken (average of 20 repetitions) vs. the number of \ac{AIM} iterations taken to compute the fundamental Schwarzschild \ac{QNM}. The plot is in a log-log scale.}
  \label{fig:package_perf}
\end{figure}

\section{Revisiting the quasinormal modes of the Schwarzschild black}
\label{ch:qnm_aim:sec:revisiting}
Utilizing \texttt{QuasinormalModes.jl}, we have investigated in Ref.~\cite{Mamani2022} the quasinormal frequencies of asymptotically flat Schwarzschild black holes with spin values of $2$, $1$, $3/2$, $2$, and $5/2$, using both the AIM and the pseudo-spectral method. Our goal was to compare and contrast the numerical results obtained with available literature data (when possible). We computed higher overtones quasinormal frequencies for all perturbation fields considered, and found purely imaginary frequencies for the spin $1/2$ and $3/2$ fields, which agree with previous analytical results in the literature. Moreover, we observed that the purely imaginary frequencies for the spin $1/2$ perturbation field are identical to those for the spin $3/2$ perturbation field. In addition, we determined the quasinormal frequencies for the spin $5/2$ perturbation field for the first time, and similarly found purely imaginary frequencies in this case.

\subsection{Perturbation equations}

The spacetime metric for a spherically symmetric Schwarzschild black hole, is given by \cite{Schwarzschild:1916uq}
%
\begin{equation}\label{eq:Metric}
  ds^2=-f(r)\,dt^2+\frac{1}{f(r)}dr^2+r^2d\theta^2+
  r^2\sin^2{\theta}\,d\varphi^2,
\end{equation}
%
where the horizon function is given by
%
\begin{equation}\label{EqHoriFuncHayward}
  f(r)=1-\frac{2M}{r},
\end{equation}
%
where $M$ is the mass of the black hole, and $r$ is the radial coordinate which, in principle, belongs to the interval $r\in [0,\, \infty)$.

The coordinates employed in the metric described by Eq.~\eqref{eq:Metric} are commonly known as Schwarzschild coordinates. This metric features an event horizon located at $r=2M$ and a physical singularity at $r=0$, both of which are well-established characteristics. In the asymptotic region, namely as $r$ approaches infinity, the metric reduces to a flat metric. For the purposes of examining quasinormal modes in this black hole spacetime, the relevant region of the spacetime is the range of the radial coordinate $r$ between $2M$ and infinity.

Our analysis of linear perturbations in the Schwarzschild black hole spacetime described by \eqref{eq:Metric} followed the standard procedure. After selecting a specific perturbation field, we reduced the corresponding partial differential equations to a unique Schrödinger-like ordinary differential equation using a series of transformations. In this approach, the perturbation functions were decomposed into Fourier modes of the form $e^{i\omega,t}=e^{i(\omega_{Re}-i,\omega_{Im})}=e^{\omega_{Im},t}\cos{\left(\omega_{Re},t\right)}$. This eliminated the time derivatives from the differential equations, while the angular dependence was handled through expansion in spherical harmonics, as is customary

It is important to be reminded at this point that ordinary QNMs are characterized by frequencies with both non-zero real and imaginary components and represent oscillatory solutions that are exponentially suppressed by the imaginary component of the frequency. In contrast, modes with purely imaginary frequencies, where the real part of the frequency is zero, correspond to purely damping solutions since the respective perturbation functions decay as $ e^{i\omega,t}=e^{,\omega_{Im},t}$.

\subsubsection{Spin $0$, $1$ and $2$ perturbations}

The computation of perturbations of integer spin, including scalar, vector, and gravitational perturbations in the Schwarzschild black hole spacetime, has been a long-standing problem with a significant body of literature dedicated to it. This literature is of great interest for our purposes, which involve comparisons and calibrations. In fact, it has been shown that the equations of motion can be expressed in a concise form known as Schrödinger-like differential equations (see, for instance, \cite{review3}). Thus, for massless scalar ($s=0$), electromagnetic ($s=1$), and vector-type gravitational perturbations ($s=2$), the Schrödinger-like equations take the following form:
%
\begin{equation}\label{Eq:IntegerSpin}
  \frac{d^2 \psi_{{s}}(r)}{d r_*^2}
  +\left[\omega^2-V_{{s}}(r)\right]\psi_{{s}}(r)=0,
\end{equation}
%
where the potential function $V_{s}(r)$ is given by
%
\begin{equation}\label{Eq:IntegerSpinPot}
  V_{s}(r)=f(r)\left[\frac{\ell\left(\ell+1\right)}{r^2}
    +\left(1-s^2\right)\frac{2M}{r^3}\right],
\end{equation}
%
and the tortoise coordinate $r_*$ is defined in terms of the areal coordinate $r$ by $dr_*=dr/f(r)$. Thus far, the problem of calculating quasinormal frequencies was reduced to solving an eigenvalue problem, which can be solved by either expanding the function $\psi$ in a base composed of special functions or solving the second-order differential equation directly.

It is important to note, that the potential function in Eq.~\eqref{Eq:IntegerSpinPot} is zero at the horizon because $f(r_h)=0$. Thus, the Schr\"odinger-like equation reduces to a single harmonic oscillator problem, whose solutions are given by
%
\begin{equation} \label{eq:sol1}
  \psi_{s}(r)=c_1\, e^{-i\omega r_*}+c_2\, e^{i\omega r_*}, \quad r\to r_h.
\end{equation}
%
The first of these solutions is interpreted as an ingoing wave, i.e., a wave that travels inward and eventually falls into the black hole event horizon. The second solution is interpreted as an outgoing wave, i.e., a wave that travels outward with respect to the black hole and can escape to spatial infinity. Considering that perturbation theory is implemented using classical assumptions, nothing is expected to come out from the black hole interior, thus, in the following analysis we impose ingoing solutions as boundary conditions at the horizon and discarding outgoing solutions altogether. This is accomplished by setting  $c_2=0$.

At spatial infinity, however, one has $f(r)\to 1$ and the effective potential of Eq.~\eqref{Eq:IntegerSpinPot} also vanishes. Thus, in such a limit the general solutions to the wave equation \eqref{Eq:IntegerSpin} have the same form as the function given in Eq.~\ref{eq:sol1}, i.e.,
%
\begin{equation}
  \psi_{s}(r_*)=c_3\, e^{-i\omega r_*}+c_4\, e^{i\omega r_*}, \quad r\to \infty.
\end{equation}
%
The first solution can be interpreted as waves that originate from outside the universe and can be avoided by setting $c_3=0$. Conversely, the second solution can be interpreted as waves that leave the universe, serving as the boundary condition at spatial infinity. It is important to note that neither the angular momentum $\ell$ nor the spin have an explicit influence on the boundary conditions.

It is interesting to note that close to the event horizon, the tortoise coordinate becomes
%
\begin{equation}
  r_*=\int \frac{dr}{f'(r_h)(r-r_h)}\approx \frac{\ln{(r-r_h)}}{ f'(r_h)},\quad r\to r_h
\end{equation}
where $r_h=2M$.
Thus, when written in terms of the radial coordinate, the boundary condition at the horizon becomes ($f'(r_h)=1/r_h$)
%
\begin{equation}
  \psi_s(r)\sim e^{-i\omega \frac{\ln{(r-r_h)}}{f'(r_h)}}\sim \left(r-r_h\right)^{-i\frac{\omega}{f'(r_h)}}.
\end{equation}
%
In turn, the tortoise coordinate at the spatial infinity becomes
%
\begin{equation}
  r_*=\int \frac{dr}{f(r)}\approx r+r_h\,\ln{r},\quad r\to \infty
\end{equation}
%
while the asymptotic solution at spatial infinity becomes
%
\begin{equation}
  \psi_s(r)\sim e^{i\omega(r+ r_h \ln{r})}\sim r^{i\, r_h\omega}e^{i\omega r}.
\end{equation}

We shall now work in terms of new coordinates defined by $u\equiv2M/r$. This is equivalent to choosing $u=1/r$ and then normalizing the mass $M$ to $2M=1$. The relation between the tortoise coordinate $r_*$ and the new coordinate $u$  becomes $du/dr_*=-u^2f(u)$. We shall also constrain our analysis to the outer region of the black hole, such that $r_h\leq r<\infty$. Hence, in terms of the new coordinate, this region is bounded to the interval $u\in [0,\, 1]$, and the potential becomes
%
\begin{equation}
  V_{s}(u)=f(u)\,u^2\left[\ell\left(1+\ell\right)+\left(1-s^2\right)\,u\right],
\end{equation}
%
where $f(u) = 1- \,u$.

To apply the pseudo-spectral method, it is necessary to express quantities defined on the background using horizon-penetrating Eddington-Finkelstein coordinates (as discussed in Ref.~\cite{qnmspectral}). However, we have developed a shortcut for writing the equations directly from the Schrödinger-like equation by implementing certain transformations. These transformations result in a differential equation that is equivalent to the one obtained from the Eddington-Finkelstein metric coordinates. In order to express the perturbation equations in terms of Eddington-Finkelstein coordinates, we utilize the transformation $\psi_{s}\to \Phi_s$, which is given by
%
\begin{equation}\label{Eq:TransN1}
  \psi_{s}= \frac{\Phi_{s}(u)}{u}e^{-i\omega\,r_*(u)}.
\end{equation}
%
Thus, Eq.~\eqref{Eq:IntegerSpin} becomes
%
%\begin{widetext}
\begin{equation}\label{Eq:IntegerSpin2}
  \begin{split}
    &\left[s^2u^2-\ell\left(\ell+1\right)u-2\,i\,\omega\right]\Phi_{s}(u)\\
    &-u\left(u^2-2\,i\,\omega\right)\Phi_{s}'(u)+\left(1-u\right)\,u^3\,\Phi_{s}''(u)=0,\end{split}
\end{equation}
%\end{widetext}
%
where we have set $M=1/2$ so that $r_h=1$. The asymptotic solutions close to the horizon may be calculated using the ansatz $\Phi_{s}(u)=(1-u)^{\alpha}$. By substituting this ansatz into Eq.~\eqref{Eq:IntegerSpin2} we get two solutions,
%
\begin{equation}\label{Eq:AsympHorion}
  \alpha=0,\qquad\qquad \alpha=2\,i\,\omega.
\end{equation}
%
The solution for $\alpha =0$ is interpreted physically as the ingoing waves at the horizon, while the other is interpreted as a wave coming out from the black hole interior, and thus, is neglected in the following analysis. In the same way, we consider the ansatz $\Phi_{s}=u^{\beta}$ to get the asymptotic solution close to the spatial infinity. Plugging this ansatz in Eq.~\eqref{Eq:IntegerSpin2} we get
%
\begin{equation}\label{Eq:AsymInfinite}
  \Phi_{s}(u)=c_5\,e^{2\,i\,\omega/u}u^{-2\,i\,\omega}+c_6\,u.
\end{equation}
%
We are interested in the divergent solution thus setting $c_6=0$. We then implement the final transformation, which takes into consideration the boundary conditions,
%
\begin{equation}\label{Eq:FinalTrans}
  \Phi_{s}(u)=e^{2\,i\,\omega/u}u^{-2\,i\,\omega}\phi_{s}(u),
\end{equation}
%
where $\phi_{s}(u)$ is a regular function in the interval $u\in[0,\,1]$ by definition. Finally, the equation of motion describing spin $0$, $1$, and $2$ perturbations is given by
%
\begin{equation}\label{Eq:IntegerSpin3}
  \begin{split}
    &\!\!\Big[\ell\left(\ell+1\right) u -s^2u^2-4\,i\lambda-16\,u\left(1+u\right)\lambda^2\Big]\phi_{s}(u)+\\
    & \!\!\Big[u^3+4i\,u\left(1-2u^2\right)\lambda\Big]\phi_{s}'(u)-(1-u)u^3\phi_{s}''(u)=0,
  \end{split}
\end{equation}
%
where we have used $\lambda=\omega M=\omega/2$. The final differential equation is then a quadratic eigenvalue problem in $\lambda$. It is also worth mentioning that in the limit of zero spin $s\to 0$, Eq.~\eqref{Eq:IntegerSpin3} reduces to Eq.~(4.8) of Ref.~\cite{qnmspectral}. These results just prove that the alternative way for getting the equations for the integer spin perturbations presented here is consistent with other approaches in the literature.

\subsection{Spin $1/2$ perturbations}

The differential equations for half-integer spin perturbations are quite distinct from Eq.~\eqref{Eq:IntegerSpin3}. The equation for the spin $1/2$ Dirac field as a perturbation on the Schwarzschild background was derived in Ref.~\cite{Cho:2003qe} by using the Newman-Penrose formalism. The analysis was generalized for arbitrary half-integer spin in Ref.~\cite{Shu:2005fw}. The resulting equation of motion for the perturbations may be written in the Schr\"odinger-like form of Eq.~\eqref{Eq:IntegerSpin}, where the potential for the massless spin 1/2 field is given by
%
\begin{equation}\label{eq:pot-s12}
  V_{\scriptscriptstyle{1/2}}= \frac{\left(1+\ell\right)\sqrt{f(r)}}{r^{2}} \left[\left(1+\ell\right)\sqrt{f(r)}+\frac{3M}{r}-1\right].
\end{equation}

It is worth mentioning that we have found a typo in the definition of $\Delta$ in \cite{Cho:2003qe}, which must read $\Delta=r(r-2M)$.

We then implement the same transformations applied in the integer spin cases. First, we change the radial coordinate to $u=2M/r$, which is defined in the interval $u\in[0,1]$. Then by setting $2M=1$ the effective potential in Eq.~\eqref{eq:pot-s12} becomes
%
\begin{equation}
  \!\!V_{\scriptscriptstyle{1/2}}=\left(1+\ell\right)u^2\sqrt{f(u)} \left[\left(1+\ell\right)\sqrt{f(u)}+\frac{3u}{2} -1\right].
\end{equation}

It is interesting pointing out that the asymptotic solutions do not depend on the spin of the field, and therefore the asymptotic solutions for this problem are the same as those obtained in Eqs.~\eqref{Eq:AsympHorion} and \eqref{Eq:AsymInfinite}. Similar transformations as those given in Eqs.~\eqref{Eq:TransN1} and~\eqref{Eq:FinalTrans} can then also be applied. Thus, the differential equation to be solved is given by
%
\begin{equation}\label{eq:onda-s12}
  \begin{split}
    &R(u)\phi_{\scriptscriptstyle{1/2}}(u) + Q(u)\phi_{\scriptscriptstyle{1/2}}'(u) + P(u) \,\phi_{\scriptscriptstyle{1/2}}''(u)=0,
  \end{split}
\end{equation}
%
in which the coefficients $R(u)$, $Q(u)$, and $P(u)$ are given by
%
\begin{eqnarray}
  R(u)=\,&& u^3+u(1+\ell)\left(1+\ell-\sqrt{1-u}\, \right) \nonumber \\
  && +\frac{u^2}{2}\left[(1+\ell)\left(3\sqrt{1-u}-4\right)-2\ell^2\right]\nonumber \\
  &&-4\,i\,(1-u)\lambda-16u(1-u^2)\lambda^2,\\
  Q(u) = && u^3(1-u)+4\,i\,u\, \lambda\left(1-u-2u^2+2u^3\right), \\
  P(u)  =&& -u^3(1-u)^2,
\end{eqnarray}
%
respectively, and with $\lambda$ standing for $\lambda=M\omega=\omega/2$.

The square root terms present in Eq.~\eqref{eq:onda-s12} may difficult the convergence of the numerical methods. To avoid them, we perform an additional change of variables given by $\chi^2=1-u$. Note that the new coordinate also belongs to the interval $\chi \in [0,\,1]$. The differential equation of Eq.~\eqref{eq:onda-s12} thus becomes
%
\begin{equation}\label{eq:onda-s12a}
  \begin{split}
    &R(\chi)\phi_{\scriptscriptstyle{1/2}}(\chi) + Q(\chi)\phi_{\scriptscriptstyle{1/2}}'(\chi) + P(\chi) \,\phi_{\scriptscriptstyle{1/2}}''(\chi)=0,
  \end{split}
\end{equation}
%
in which the coefficients $R(\chi)$, $Q(\chi)$, and $P(\chi)$ are given by
%
\begin{eqnarray}
  R(\chi)=
  \,&&2(1-\chi^2)\left[\left(\ell+1\right) \left(1+ 2\ell\,\chi-3\chi^2\right) + 2\ell\, \chi +2\chi^3\right] \nonumber \\
  && - 8\,i\,\chi \,\lambda-32\,\chi\left(2-3\chi^2+\chi^4\right)\lambda^2,\\
  Q(\chi) = && (\chi^2-1\big)\left[\big(1-\chi^2\big)^2-8\,i\big(1-4\chi^2+2\chi^4\big)\lambda\right],\\
  P(\chi)  =&& -\chi\big(1-\chi^2\big)^3.
\end{eqnarray}

\subsubsection{Spin $3/2$ perturbations}

As is the case for spin $1/2$ perturbations, the perturbation equation for spin the $3/2$ field is quite different from that of integer spin fields. The relevant equation is obtained by making use of Ref.~\cite{Shu:2005fw}, specifically Eq.~(37) of that reference, and by setting $s=3/2$ on such an equation we then get the effective potential of the Schr\"odinger-like equation Eq.~\eqref{Eq:IntegerSpin},
%
\begin{equation} \begin{split}
    V_{\scriptscriptstyle{3/2}} &=\frac{(1+\ell)(2+\ell)(3+\ell)\sqrt{f(r)}}{\left[2M+r(1+\ell)(3+\ell)\right]^2} \bigg(\frac{2M^2}{r^2}\\ &+(1+\ell)(3+\ell)\left[(2+\ell)\sqrt{f(r)}+\frac{3M}{r}-1\right]\bigg).
  \end{split}
\end{equation}
%
As before, we change the radial coordinate to $u=1/r$ and rescale the mass as $2M=1$. Thus, the potential becomes
%
\begin{equation}\begin{split}
    V_{\scriptscriptstyle{3/2}}&=\frac{u^2(1+\ell)(2+\ell)(3+\ell)\sqrt{1-u}}{2\big[u+(1+\ell)(3+\ell)\big]^2} \bigg(\frac{u^2}{2}\\ &+(1+\ell)(3+\ell)\left[(2+\ell)\sqrt{1-u}+\frac{3u}{2}-1\right]\bigg).
  \end{split}
\end{equation}

Following the procedure implemented in the transition of Eq.~\eqref{Eq:TransN1} to Eq.~\eqref{Eq:AsymInfinite}, the differential equation Eq.~\eqref{eq:onda-s12} becomes
%
\begin{equation}\label{eq:onda-s32a}
  \begin{split}
    &R(\chi)\phi_{\scriptscriptstyle{3/2}}(\chi) + Q(\chi)\phi_{\scriptscriptstyle{3/2}}'(\chi) + P(\chi) \,\phi_{\scriptscriptstyle{3/2}}''(\chi)=0,
  \end{split}
\end{equation}
%
where the coefficients $R(\chi)$, $Q(\chi)$, and $P(\chi)$ are given by
%
\begin{eqnarray}
  R(\chi)=
  \,&& 2\big(1-\chi^2\big)\Big[6+11\ell+6\ell^2+\ell^3 +2\chi^5+  \nonumber\\
    &&4\chi^4(2+\ell) +2\chi^3\left(3+4\ell+\ell^2\right)- \nonumber\\
    &&\!\!\chi^2\left(2-7\ell-6\ell^2-\ell^3\right)  + 2\chi(2+\ell)^2\left(2+4\ell+\ell^2\right)\Big] \nonumber \\
  &&\! -16\chi(2+\chi+\ell)^2\lambda\Big[i + 4 \left(2-\chi^2\right)\left(1-\chi^2\right)\lambda\Big],\nonumber\\
  Q(\chi) = && -\left(1-\chi^2\right)\left(2+\chi+\ell\right)^2\nonumber\\
  && \times \left[\left(1-\chi^2\right)^2-8\,i\,(1-4\chi^2+2\chi^4)\lambda\right],\nonumber\\
  P(\chi)  =&&
  -\chi\left(1-\chi^2\right)^3\left(2+\chi+\ell\right)^2,\nonumber
\end{eqnarray}
%
where we have used the new coordinate $\chi^2=1-u$ to avoid square roots. Once again, we obtain a quadratic eigenvalue problem, and the function $\phi_{\scriptscriptstyle{3/2}}(\chi)$ is regular in the interval $\chi\in [0,\,1]$.

\subsection{Spin $5/2$ perturbations}

The investigation of higher spin fields is believed to be a promising avenue for gaining insight into fundamental physics, including the development of new unifying theories for fundamental interactions or new phenomenology that goes beyond the standard model. The primary motivation for exploring perturbations of the spin $5/2$ field is the Rarita-Schwinger theory. Drawing inspiration from this theory, the authors of Ref.\cite{Shklyar:2009cx} computed physical observables for the spin $5/2$ field. In our work, we utilize the generic equation derived in Ref.\cite{Shu:2005fw}, specifically Eq.~(37), to determine the quasinormal frequencies of this perturbation field on the Schwarzschild black hole. However, due to the size of the resulting differential equation for this perturbation field, we refer the reader to Appendix A of Ref.\cite{Mamani2022} for details.

\subsection{Results}

This section presents our numerical findings for the quasinormal frequencies of various spin perturbation fields, as shown in Tables \ref{Tab:Spin0}-\ref{Tab:Spin5/2}. Each table includes four columns: the first two display data acquired through the pseudo-spectral method using different numbers of interpolating polynomials, the third column shows results obtained from the AIM method using \texttt{QuasinormalModes.jl}, and the fourth and fifth columns are reproductions of literature results, when available. As observed in all data tables, pseudo-spectral method I (computed with $60$ polynomials) and pseudo-spectral method II (computed with $40$ polynomials) produce practically identical results to those generated by \texttt{QuasinormalModes.jl}, with a precision of six decimal places. The numerical methods utilized in this study yield more precise outcomes than those attained previously using the WKB approximation and enable us to compute additional frequencies that have not been reported in previous literature.

We also found purely imaginary frequencies for the spin $1/2$, $3/2$ and $5/2$ fields, reported in Tables \ref{Tab:PurelyImSpin1/2}-\ref{Tab:PurelyImSpin5/2}, respectively. Such frequencies arise when investigating the quasinormal modes in the limit of large $\ell$. Notice that these results are also in agreement with the analytic solutions obtained in the literature but for spin $5/2$ fields, the discrepancy between both methods increases as the imaginary frequency gets more negative. We do not have a concrete explanation for this fact.

\begin{table}[ht]
  \centering
  \resizebox{\linewidth}{!}{%
    \begin{tabular}{l|c|c|c|c|c|c}
      \hline
      $l$ & $n$ & \tlt{Pseudo-spectral}{I (60 Polynomials)} & \tlt{Pseudo-spectral}{II (40 polynomials)} & \tlt{AIM}{100 Iterations}  & Ref.~\cite{Shu:2005fw} & Ref.~\cite{Konoplya:2004ip} \\ \hline\hline
      0   & $0$ & $\pm 0.110455 -0.104896 i$                & $\pm 0.110455 -0.104896 i$                 & $\pm 0.110455 -0.104896 i$ & $0.1046-0.1152 i$      & $\pm 0.1105-0.1008i$        \\ \hline
      1   & $0$ & $\pm 0.292936 -0.097660 i$                & $\pm 0.292936 -0.097660 i$                 & $\pm 0.292936 -0.097660 i$ & $0.2911-0.0980 i$      & $\pm 0.2929-0.0978i$        \\
          & $1$ & $\pm 0.264449 -0.306257 i$                & $\pm 0.264449 -0.306257 i$                 & $\pm 0.264449 -0.306257 i$ & ---                    & $\pm 0.2645-0.3065i$        \\ \hline
      2   & $0$ & $\pm 0.483644 -0.096759 i$                & $\pm 0.483644 -0.096759 i$                 & $\pm 0.483644 -0.096759 i$ & $0.4832-0.0968 i$      & $\pm 0.4836-0.0968i$        \\
          & $1$ & $\pm 0.463851 -0.295604 i$                & $\pm 0.463851 -0.295604 i$                 & $\pm 0.463851 -0.295604 i$ & $0.4632-0.2958 i$      & $\pm 0.4638-0.2956i$        \\
          & $2$ & $\pm 0.430544 -0.508558 i$                & $\pm 0.430544 -0.508558 i$                 & $\pm 0.430544 -0.508558 i$ & ---                    & $\pm 0.4304-0.5087i$        \\ \hline
      3   & $0$ & $\pm 0.675366 -0.096500 i$                & $\pm 0.675366 -0.096500 i$                 & $\pm 0.675366 -0.096500 i$ & $0.6752-0.0965 i$      & ---                         \\
          & $1$ & $\pm 0.660671 -0.292285 i$                & $\pm 0.660671 -0.292285 i$                 & $\pm 0.660671 -0.292285 i$ & $0.6604-0.2923 i$      & ---                         \\
          & $2$ & $\pm 0.633626 -0.496008 i$                & $\pm 0.633626 -0.496008 i$                 & $\pm 0.633626 -0.496008 i$ & $0.6348-0.4941 i$      & ---                         \\
          & $3$ & $\pm 0.598773 -0.711221 i$                & $\pm 0.598773 -0.711221 i$                 & $\pm 0.598773 -0.711221 i$ & ---                    & ---                         \\ \hline
      4   & $0$ & $\pm 0.867416 -0.096392 i$                & $\pm 0.867416 -0.096392 i$                 & $\pm 0.867416 -0.096392 i$ & $0.8673-0.0964 i$      & ---                         \\
          & $1$ & $\pm 0.855808 -0.290876 i$                & $\pm 0.855808 -0.290876 i$                 & $\pm 0.855808 -0.290876 i$ & $0.8557-0.2909 i$      & ---                         \\
          & $2$ & $\pm 0.833692 -0.490325 i$                & $\pm 0.833692 -0.490325 i$                 & $\pm 0.833692 -0.490325 i$ & $0.8345-0.4895 i$      & ---                         \\
          & $3$ & $\pm 0.803288 -0.697482 i$                & $\pm 0.803288 -0.697482 i$                 & $\pm 0.803288 -0.697482 i$ & $0.8064-0.6926 i$      & ---                         \\
          & $4$ & $\pm 0.767733 -0.914019 i$                & $\pm 0.767733 -0.914019 i$                 & $\pm 0.767733 -0.914019 i$ & ---                    & ---                         \\
      \hline\hline
    \end{tabular}%
  }
  \caption{
    Quasinormal frequencies of the spin $0$ perturbations normalized by the mass $(M\omega)$ compared against the results of Refs.~\cite{Shu:2005fw, Konoplya:2004ip}.
  }
  \label{Tab:Spin0}
\end{table}

\begin{table}[ht]
  \centering
  \resizebox{\linewidth}{!}{%
    \begin{tabular}{l |c|c|c|c|c|c}
      \hline
      $l$ & $n$ & \tlt{Pseudo-spectral}{I (60 Polynomials)} & \tlt{Pseudo-spectral}{II (40 polynomials)} & \tlt{AIM}{100 Iterations}  & Ref.~\cite{Shu:2005fw} & Ref.~\cite{Konoplya:2004ip} \\ \hline\hline
      1   & $0$ & $\pm 0.248263-0.092488 i$                 & $\pm 0.248263 -0.092488 i$                 & $\pm 0.248263 -0.092488 i$ & $0.2459-0.0931i$       & $\pm 0.2482-0.0926i$        \\
          & $1$ & $\pm 0.214515-0.293668 i$                 & $\pm 0.214515 -0.293667 i$                 & $\pm 0.214515 -0.293668 i$ & ---                    & $\pm 0.2143-0.2941i$        \\ \hline
      2   & $0$ & $\pm 0.457596-0.095004 i$                 & $\pm 0.457595 -0.095004 i$                 & $\pm 0.457596 -0.095004 i$ & $0.4571-0.0951i$       & $\pm 0.4576-0.0950i$        \\
          & $1$ & $\pm 0.436542-0.290710 i$                 & $\pm 0.436542 -0.290710 i$                 & $\pm 0.436542 -0.290710 i$ & $0.4358-0.2910i$       & $\pm 0.4365-0.2907i$        \\
          & $2$ & $\pm 0.401187-0.501587 i$                 & $\pm 0.401187 -0.501587 i$                 & $\pm 0.401187 -0.501587 i$ & ---                    & $\pm 0.4009-0.5017i$        \\ \hline
      3   & $0$ & $\pm 0.656899-0.095616 i$                 & $\pm 0.656899 -0.095616 i$                 & $\pm 0.656899 -0.095616 i$ & $0.6567-0.0956i$       & $\pm 0.6569-0.0956i$        \\
          & $1$ & $\pm 0.641737-0.289728 i$                 & $\pm 0.641737 -0.289728 i$                 & $\pm 0.641737 -0.289728 i$ & $0.6415-0.2898i$       & $\pm 0.6417-0.2897i$        \\
          & $2$ & $\pm 0.613832-0.492066 i$                 & $\pm 0.613832 -0.492066 i$                 & $\pm 0.613832 -0.492066 i$ & $0.6151-0.4901i$       & $\pm 0.6138-0.4921i$        \\
          & $3$ & $\pm 0.577919-0.706331 i$                 & $\pm 0.577919 -0.706331 i$                 & $\pm 0.577919 -0.706330 i$ & ---                    & $\pm 0.5775-0.7065i$        \\ \hline
      4   & $0$ & $\pm 0.853095-0.095860 i$                 & $\pm 0.853095 -0.095860 i$                 & $\pm 0.853095 -0.095810 i$ & $0.8530-0.0959i$       & ---                         \\
          & $1$ & $\pm 0.841267-0.289315 i$                 & $\pm 0.841267 -0.289315 i$                 & $\pm 0.841267 -0.289315 i$ & $0.8411-0.2893i$       & ---                         \\
          & $2$ & $\pm 0.818728-0.487838 i$                 & $\pm 0.818728 -0.487838 i$                 & $\pm 0.818728 -0.487838 i$ & $0.8196-0.4870i$       & ---                         \\
          & $3$ & $\pm 0.787748-0.694242 i$                 & $\pm 0.787748 -0.694242 i$                 & $\pm 0.787748 -0.694242 i$ & $0.7909-0.6892i$       & ---                         \\
          & $4$ & $\pm 0.751549-0.910242 i$                 & $\pm 0.751549 -0.910242 i$                 & $\pm 0.751549 -0.910242 i$ & ---                    & ---                         \\
      \hline\hline
    \end{tabular}%
  }
  \caption{
    Quasinormal frequencies of the spin $1$ perturbations normalized by the mass $(M\omega)$ compared against the results of Refs.~\cite{Shu:2005fw, Konoplya:2004ip}.
  }
  \label{Tab:Spin1}
\end{table}

\begin{table}[ht]
  \centering
  \resizebox{\linewidth}{!}{%
    \begin{tabular}{l |c|c|c|c|c|c}
      \hline
      $l$ & $n$ & \tlt{Pseudo-spectral}{I (60 Polynomials)} & \tlt{Pseudo-spectral}{II (40 polynomials)} & \tlt{AIM}{100 Iterations}  & Ref.~\cite{Shu:2005fw} & Ref.~\cite{Konoplya:2004ip} \\ \hline\hline
      2   & $0$ & $\pm 0.373672-0.088962i$                  & $\pm 0.373672 -0.088962 i$                 & $\pm 0.373672 -0.088962 i$ & $0.3730-0.0891i$       & $\pm 0.3736-0.0890i$        \\
          & $1$ & $\pm 0.346711-0.273915i$                  & $\pm 0.346711 -0.273915 i$                 & $\pm 0.346711 -0.273915 i$ & $0.3452-0.2746i$       & $\pm 0.3463-0.2735i$        \\
          & $2$ & $\pm 0.301053-0.478277i$                  & $\pm 0.301053 -0.478277 i$                 & $\pm 0.301053 -0.478277 i$ & ---                    & $\pm 0.2985-0.4776i$        \\ \hline
      3   & $0$ & $\pm 0.599443-0.092703i$                  & $\pm 0.599443 -0.092703 i$                 & $\pm 0.599443 -0.092703 i$ & $0.5993-0.0927i$       & $\pm 0.5994-0.0927i$        \\
          & $1$ & $\pm 0.582644-0.281298i$                  & $\pm 0.582644 -0.281298 i$                 & $\pm 0.582644 -0.281298 i$ & $0.5824-0.2814i$       & $\pm 0.5826-0.2813i$        \\
          & $2$ & $\pm 0.551685-0.479093i$                  & $\pm 0.551685 -0.479093 i$                 & $\pm 0.551685 -0.479027 i$ & $0.5532-0.4767i$       & $\pm 0.5516-0.4790i$        \\
          & $3$ & $\pm 0.511962-0.690337i$                  & $\pm 0.511962 -0.690337 i$                 & $\pm 0.511962 -0.690337 i$ & ---                    & $\pm 0.5111-0.6905i$        \\ \hline
      4   & $0$ & $\pm 0.809178-0.094164i$                  & $\pm 0.809178 -0.094164 i$                 & $\pm 0.809178 -0.094164 i$ & $0.8091-0.0942i$       & $\pm 0.8092-0.0942i$        \\
          & $1$ & $\pm 0.796632-0.284334i$                  & $\pm 0.796632 -0.284334 i$                 & $\pm 0.796632 -0.284334 i$ & $0.7965-0.2844i$       & $\pm 0.7966-0.2843i$        \\
          & $2$ & $\pm 0.772710-0.479908i$                  & $\pm 0.772710 -0.479908 i$                 & $\pm 0.772710 -0.479908 i$ & $0.7736-0.4790i$       & $\pm 0.7727-0.4799i$        \\
          & $3$ & $\pm 0.739837-0.683924i$                  & $\pm 0.739837 -0.683924 i$                 & $\pm 0.739837 -0.683924 i$ & $0.7433-0.6783i$       & $\pm 0.7397-0.6839i$        \\
          & $4$ & $\pm 0.701516-0.898239i$                  & $\pm 0.701516 -0.898239 i$                 & $\pm 0.701516 -0.898239 i$ & ---                    & $\pm 0.7006-0.8985i$        \\
      \hline\hline
    \end{tabular}%
  }
  \caption{
    Quasinormal frequencies of spin $2$ perturbations normalized by the mass $(M\omega)$ compared against the results of Refs.~\cite{Shu:2005fw, Konoplya:2004ip}.
  }
  \label{Tab:Spin2}
\end{table}

\begin{table}[ht]
  \centering
  \resizebox{\linewidth}{!}{%
    \begin{tabular}{l |c|c|c|c|c|c}
      \hline
      $l$ & $n$ & \tlt{Pseudo-spectral}{I (60 Polynomials)} & \tlt{Pseudo-spectral}{II (40 polynomials)} & \tlt{AIM}{100 Iterations}  & Ref.~\cite{Shu:2005fw} & Ref.~\cite{Cho:2003qe} \\ \hline\hline
      0   & $0$ & $\pm 0.182963 -0.096982 i$                & $\pm 0.182963 -0.096982 i$                 & $\pm 0.182963 -0.096824 i$ & ---                    & ---                    \\ \hline
      1   & $0$ & $\pm 0.380037 -0.096405 i$                & $\pm 0.380037 -0.096405 i$                 & $\pm 0.380037 -0.096405 i$ & $0.3786-0.0965 i$      & $0.379 -0.097i$        \\
          & $1$ & $\pm 0.355833 -0.297497 i$                & $\pm 0.355833 -0.297497 i$                 & $\pm 0.355833 -0.297497 i$ & ---                    & ---                    \\ \hline
      2   & $0$ & $\pm 0.574094 -0.096305 i$                & $\pm 0.574094 -0.096305 i$                 & $\pm 0.574094 -0.096305 i$ & $0.5737-0.0963 i$      & $0.574 -0.096i$        \\
          & $1$ & $\pm 0.557015 -0.292715 i$                & $\pm 0.557015 -0.292715 i$                 & $\pm 0.557015 -0.292715 i$ & $0.5562-0.2930 i$      & $0.556 -0.293i$        \\
          & $2$ & $\pm 0.526607 -0.499695 i$                & $\pm 0.526607 -0.499695 i$                 & $\pm 0.526607 -0.499695 i$ & ---                    & ---                    \\ \hline
      3   & $0$ & $\pm 0.767355 -0.096270 i$                & $\pm 0.767355 -0.096270 i$                 & $\pm 0.767355 -0.096270 i$ & $0.7672-0.0963 i$      & $0.767 -0.096i$        \\
          & $1$ & $\pm 0.754300 -0.290968 i$                & $\pm 0.754300 -0.290968 i$                 & $\pm 0.754300 -0.290968 i$ & $0.7540-0.2910 i$      & $0.754 -0.291i$        \\
          & $2$ & $\pm 0.729770 -0.491910 i$                & $\pm 0.729770 -0.491910 i$                 & $\pm 0.729770 -0.491910 i$ & $0.7304-0.4909 i$      & $0.730 -0.491i$        \\
          & $3$ & $\pm 0.696913 -0.702293 i$                & $\pm 0.696913 -0.702293 i$                 & $\pm 0.696913 -0.702293 i$ & ---                    & ---                    \\ \hline
      4   & $0$ & $\pm 0.960293 -0.096254 i$                & $\pm 0.960293 -0.096254 i$                 & $\pm 0.960293 -0.096254 i$ & $0.9602-0.0963 i$      & $0.960 -0.096i$        \\
          & $1$ & $\pm 0.949759 -0.290148 i$                & $\pm 0.949759 -0.290148 i$                 & $\pm 0.949759 -0.290148 i$ & $0.9496-0.2902 i$      & $0.950 -0.290i$        \\
          & $2$ & $\pm 0.929494 -0.488116 i$                & $\pm 0.929494 -0.488116 i$                 & $\pm 0.929494 -0.488116 i$ & $0.9300-0.4876 i$      & $0.930 -0.488i$        \\
          & $3$ & $\pm 0.901129 -0.692520 i$                & $\pm 0.901129 -0.692520 i$                 & $\pm 0.901129 -0.692520 i$ & $0.9036-0.6892 i$      & $0.904 -0.689i$        \\
          & $4$ & $\pm 0.867043 -0.905047 i$                & $\pm 0.867008 -0.905066 i$                 & $\pm 0.867043 -0.905047 i$ & ---                    & ---                    \\
      \hline\hline
    \end{tabular}%
  }
  \caption{
    Quasinormal frequencies of the spin $1/2$ perturbations normalized by the mass $(M\omega)$ compared against the results of Refs.~\cite{Cho:2003qe, Shu:2005fw}.
  }
  \label{Tab:Spin1/2}
\end{table}

\begin{table}[ht]
  \centering
  \resizebox{\linewidth}{!}{%
    \begin{tabular}{l |c|c|c|c|c|c}
      \hline
      $l$ & $n$ & \tlt{Pseudo-spectral}{I (60 Polynomials)} & \tlt{Pseudo-spectral}{II (40 polynomials)} & \tlt{AIM}{100 Iterations}  & Ref.~\cite{Shu:2005fw} & Ref.~\cite{Chen:2016qii} \\ \hline\hline
      0   & $0$ & $\pm 0.311292 -0.090087 i$                & $\pm 0.311292 -0.090087 i$                 & $\pm 0.311292 -0.090087 i$ & ---                    & $0.3112 -0.0902 i$       \\ \hline
      1   & $0$ & $\pm 0.530048 -0.093751 i$                & $\pm 0.530048 -0.093751 i$                 & $\pm 0.530048 -0.093751 i$ & ---                    & $0.5300 -0.0937 i$       \\
          & $1$ & $\pm 0.511392 -0.285423 i$                & $\pm 0.511392 -0.285423 i$                 & $\pm 0.511392 -0.285423 i$ & ---                    & $0.5113 -0.2854 i$       \\ \hline
      2   & $0$ & $\pm 0.734750 -0.094878 i$                & $\pm 0.734750 -0.094878 i$                 & $\pm 0.734750 -0.094878 i$ & $\pm 0.7346 -0.0949 i$ & $0.7347 -0.0948 i$       \\
          & $1$ & $\pm 0.721047 -0.286906 i$                & $\pm 0.721047 -0.286906 i$                 & $\pm 0.721047 -0.286906 i$ & $\pm 0.7206 -0.2870 i$ & $0.7210 -0.2869 i$       \\
          & $2$ & $\pm 0.695287 -0.485524 i$                & $\pm 0.695287 -0.485524 i$                 & $\pm 0.695287 -0.485524 i$ & ---                    & $0.6952 -0.4855 i$       \\ \hline
      3   & $0$ & $\pm 0.934364 -0.095376 i$                & $\pm 0.934364 -0.095376 i$                 & $\pm 0.934364 -0.095376 i$ & $\pm 0.9343 -0.0954 i$ & $0.9343 -0.0953 i$       \\
          & $1$ & $\pm 0.923502 -0.287560 i$                & $\pm 0.923502 -0.287560 i$                 & $\pm 0.923502 -0.287560 i$ & $\pm 0.9233 -0.2876 i$ & $0.9235 -0.2875 i$       \\
          & $2$ & $\pm 0.902599 -0.483957 i$                & $\pm 0.902599 -0.483957 i$                 & $\pm 0.902599 -0.483957 i$ & $\pm 0.9031 -0.4835 i$ & $0.9025 -0.4839 i$       \\
          & $3$ & $\pm 0.873342 -0.687024 i$                & $\pm 0.873343 -0.687024 i$                 & $\pm 0.873342 -0.687024 i$ & ---                    & $0.8732 -0.6870 i$       \\ \hline
      4   & $0$ & $\pm 1.131530 -0.095640 i$                & $\pm 1.131530 -0.095640 i$                 & $\pm 1.131530 -0.095640 i$ & $\pm 1.1315 -0.0956 i$ & $1.1315 -0.0956 i$       \\
          & $1$ & $\pm 1.122523 -0.287908 i$                & $\pm 1.122523 -0.287908 i$                 & $\pm 1.122523 -0.287908 i$ & $\pm 1.1224 -0.2879 i$ & $1.1225 -0.2879 i$       \\
          & $2$ & $\pm 1.104976 -0.483096 i$                & $\pm 1.104976 -0.483096 i$                 & $\pm 1.104976 -0.483096 i$ & $\pm 1.1053 -0.4828 i$ & $1.1049 -0.4830 i$       \\
          & $3$ & $\pm 1.079852 -0.683000 i$                & $\pm 1.079852 -0.683000 i$                 & $\pm 1.079852 -0.683000 i$ & $\pm 1.0817 -0.6812 i$ & $1.0798 -0.6829 i$       \\
          & $4$ & $\pm 1.048599 -0.889113 i$                & $\pm 1.048596 -0.889115 i$                 & $\pm 1.048599 -0.889113 i$ & ---                    & $1.0484 -0.8890 i$       \\
      \hline\hline
    \end{tabular}%
  }
  \caption{
    Quasinormal frequencies of spin $3/2$ perturbations normalized by the mass $(M\omega)$ compared against the results of Refs.~\cite{Chen:2016qii, Shu:2005fw}.
  }
  \label{Tab:Spin3/2}
\end{table}

\begin{table}[ht]
  \centering
  \resizebox{\linewidth}{!}{%
    \begin{tabular}{l |c|c|c|c}
      \hline
      $l$ & $n$ & \tlt{Pseudo-spectral}{I (60 Polynomials)} & \tlt{Pseudo-spectral}{II (40 polynomials)} & \tlt{AIM}{100 Iterations} \\ \hline\hline
      0   & $0$ & $\pm 0.462727-0.092578i$                  & $\pm 0.462727-0.092578i$                   & $0.462727 - 0.092577 i$   \\ \hline
      1   & $0$ & $\pm 0.687103-0.094566i$                  & $\pm 0.687103-0.094566i$                   & $0.687103 - 0.094566 i$   \\
          & $1$ & $\pm 0.670542-0.285767i$                  & $\pm 0.670542-0.285767i$                   & $0.670542 - 0.285767 i$   \\ \hline
      2   & $0$ & $\pm 0.897345-0.095309i$                  & $\pm 0.897345-0.095309i$                   & $0.897345 - 0.095309 i$   \\
          & $1$ & $\pm 0.884980-0.287266i$                  & $\pm 0.884980-0.287266i$                   & $0.884980 - 0.287266 i$   \\
          & $2$ & $\pm 0.861109-0.483113i$                  & $\pm 0.861109-0.483113i$                   & $0.861109 - 0.483113 i$   \\ \hline
      3   & $0$ & $\pm 1.101190-0.095648i$                  & $\pm 1.101190-0.095648i$                   & $1.101190 - 0.095648 i$   \\
          & $1$ & $\pm 1.091300-0.287886i$                  & $\pm 1.091300-0.287886i$                   & $1.091300 - 0.287886 i$   \\
          & $2$ & $\pm 1.071999-0.482895i$                  & $\pm 1.071999-0.482895i$                   & $1.071999 - 0.482895 i$   \\
          & $3$ & $\pm 1.044272-0.682307i$                  & $\pm 1.044272-0.682307i$                   & $1.044272 - 0.682307 i$   \\ \hline
      4   & $0$ & $\pm 1.301587-0.095829i$                  & $\pm 1.301587-0.095829i$                   & $1.301587 - 0.095829 i$   \\
          & $1$ & $\pm 1.293328-0.288184i$                  & $\pm 1.293328-0.288184i$                   & $1.293328 - 0.288184 i$   \\
          & $2$ & $\pm 1.277107-0.482604i$                  & $\pm 1.277107-0.482604i$                   & $1.277107 - 0.482604 i$   \\
          & $3$ & $\pm 1.253526-0.680366i$                  & $\pm 1.253526-0.680366i$                   & $1.253526 - 0.680366 i$   \\
          & $4$ & $\pm 1.223513-0.882554i$                  & $\pm 1.223512-0.882553i$                   & $1.223513 - 0.882554 i$   \\
      \hline\hline
    \end{tabular}%
  }
  \caption{
    Quasinormal frequencies of spin $5/2$ perturbations normalized by the mass $(M\omega)$.
  }
  \label{Tab:Spin5/2}
\end{table}


\begin{table}[t]
  \centering
  \resizebox{\linewidth}{!}{%
    \begin{tabular}{c|c|c}
      \hline
      \tlt{Pseudo-spectral}{I (60 Polynomials)} & \tlt{Pseudo-spectral}{II (40 polynomials)} & \tlt{AIM}{100 Iterations} \\ \hline\hline
      $-0.250000i$                              & $-0.250000i$                               & $-0.250000i$              \\ \hline
      $-0.500000i$                              & $-0.500000i$                               & $-0.500000i$              \\ \hline
      $-0.750000i$                              & $-0.750000i$                               & $-0.750000i$              \\ \hline
      $-1.000000i$                              & $-1.000000i$                               & $-1.000031i$              \\ \hline
      $-1.2499998i$                             & $-1.250000i$                               & $-1.246550i$              \\
      \hline\hline
    \end{tabular}%
  }
  \caption{
    Purely imaginary frequencies for spin $1/2$ perturbations normalized by the mass $(M\omega)$. The numerical values of such frequencies are exactly the same as for the purely imaginary frequencies arising in the QNM of spin $3/2$ perturbations.
  }
  \label{Tab:PurelyImSpin1/2}
\end{table}

\begin{table}[ht]
  \centering
  \resizebox{\linewidth}{!}{%
    \begin{tabular}{c|c|c}
      \hline
      \tlt{Pseudo-spectral}{I (60 Polynomials)} & \tlt{Pseudo-spectral}{II (40 polynomials)} & \tlt{AIM}{100 Iterations} \\ \hline\hline
      $-0.250000i$                              & $-0.250000i$                               & $-0.250000i$              \\ \hline
      $-0.500000i$                              & $-0.500000i$                               & $-0.500000i$              \\ \hline
      $-0.750000i$                              & $-0.750000i$                               & $-0.750000i$              \\ \hline
      $-1.000000i$                              & $-1.000000i$                               & $-1.000031i$              \\ \hline
      $-1.2499998i$                             & $-1.250000i$                               & $-1.246550i$              \\
      \hline\hline
    \end{tabular}%
  }
  \caption{
    Purely imaginary frequencies for spin $3/2$ perturbations normalized by the mass $(M\omega)$. The numerical values of such frequencies are exactly the same as for the purely imaginary frequencies arising in the QNM of spin $1/2$ perturbations.
  }
  \label{Tab:PurelyImSpin3/2}
\end{table}

\begin{table}[ht]
  \centering
  \resizebox{\linewidth}{!}{%
    \begin{tabular}{c|c|c}
      \hline
      \tlt{Pseudo-spectral}{I (60 Polynomials)} & \tlt{Pseudo-spectral}{II (40 polynomials)} & \tlt{AIM}{100 Iterations} \\ \hline\hline
      $-0.125000i$                              & $-0.125000i$                               & $-0.125000i$              \\ \hline
      $-0.375602i$                              & $-0.375602i$                               & $-0.378659i$              \\ \hline
      $-0.626877i$                              & $-0.626877i$                               & $-0.623931i$              \\ \hline
      $-0.878946i$                              & $-0.878948i$                               & $-0.907374i$              \\
      \hline\hline
    \end{tabular}%
  }
  \caption{
    Purely imaginary frequencies of spin $5/2$ perturbations normalized by the mass $(M\omega)$.
  }
  \label{Tab:PurelyImSpin5/2}
\end{table}

\myclearpage
\par

\chapter{Conclusions and perspectives}
\label{ch:conclusion}
In this thesis we have investigated classical phenomena in multiple models of black hole binaries. Specifically, we have analyzed energy extraction, quasinormal modes, and wave scattering around black hole binaries, utilizing numerical, analytical, and semi-analytical techniques. This study has provided the first-ever demonstration of how the phenomena under investigation are altered by the presence of a binary, as compared to their single black hole counterparts.

Regarding the PP, we have demonstrated in several comprehensive ways how energy extraction becomes increasingly effective in the presence of charged and rotating binaries. Additionally, we have devised an innovative technique that extends the analysis of energy extraction in a single black hole to arbitrary spacetimes, including numeric ones. We have provided a proof-of-concept example featuring a superimposed Kerr binary, which confirms that the energy extraction efficiency is enhanced by the presence of a secondary object.

Furthermore, a numerical package with broad applicability for computing quasinormal frequencies of black hole spacetimes has been developed. This package employs the recently developed Asymptotic Iteration Method and can easily handle multiple systems, that is, those whose perturbation evolutions can be written as second order ODEs. We have utilized this package to revisit the perturbation problem in the Schwarzschild spacetime, comparing the frequencies computed via the new code with frequencies obtained through traditional pseudospectral methods and values presented in the literature via the WKB approximation or Leaver's continued fraction method. The findings indicate that the results obtained using the AIM code are in excellent agreement with available results, thereby enhancing confidence in the method's accuracy and potential to address novel problems. Finally, we have computed new quasinormal frequencies for spin $5/2$ perturbations for the first time, highlighting that these perturbations also possess purely imaginary frequencies.

Lastly, we have elaborated on the creation of a new \texttt{EinsteinToolkit Thorn}, called \texttt{FieldPerturbations}, for the numerical evolution of a scalar field superimposed on a dynamically evolving metric. We have presented preliminary findings concerning the scattering of a massless field over the GW150914 binary collision. Specifically, our results have demonstrated that the field profile displays a damped oscillatory phase following a highly nonlinear scattering, which bears similarities to the behavior observed when perturbing a single black hole. However, distinct differences prohibit direct fitting of a damped sine or cosine to the signal. Given that the run is in its initial stage, and the code remains subject to rigorous testing and development, we must conclude that it is currently impossible to confirm conclusively whether this effect is exclusively attributable to the nonlinear nature of the interaction between the black holes in the binary or to numerical artifacts due to various sources.

We contend that our study has yielded valuable insights and ideas for extending established phenomena to binary systems. We maintain that it is crucial to develop tools and methods capable of extending these analyses to binary black hole spacetime models that lack analytic solutions, given that numerical methods are necessary to capture the full effect of nonlinear interaction in General Relativity. Moreover, we believe that our work represents a crucial step towards this goal and serves to inspire further research in these phenomena.

We propose three potential directions for future research. Firstly, we aim to incorporate our proposed approach for studying the Penrose process in dynamic spacetimes into fully dynamical simulations. We suggest utilizing the \texttt{EinsteinToolkit} infrastructure to compute orbits as a background metric is evolved, or even using tables of a numerically evolved metric as input for the code. Secondly, we contend that it may be possible to separate the perturbation partial differential equations (PDEs) that arise from analytical models of binary spacetimes by employing series approximations to the solutions. This approach would enable the derivation of a system of coupled ordinary differential equations (ODEs) that could, in principle, be solved via the AIM or other methods to yield arbitrarily accurate approximations of quasinormal frequencies in these contexts. Finally, we suggest further exploring the numerical simulation of the scalar field on top of the GW150914 binary, searching for potential energy growth via superradiant scattering, analyzing the profile of the field to identify quasinormal frequencies, and investigating other spin perturbations. Additionally, we propose considering the backreaction of the spacetime metric to enhance the realism of the simulations.

\myclearpage
\par

%\appendix

%%%%%%%%%%%%%%%%%%%%%%%%%%%%%
%%% BIBLIOGRAPHY
\bibliographystyle{apsrev4-1}
\bibliography{src/ref_aim.bib,src/ref_books.bib,src/ref_intro.bib,src/ref_owned.bib,src/ref_penrose.bib,src/ref_wave_scattering.bib}

\end{document}
