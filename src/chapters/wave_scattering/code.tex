The coordinate system used in our grid setup was the \texttt{Thornburg04} coordinates, implemented through the \texttt{Llama}~\cite{PhysRevD.83.044045} infrastructure. This coordinate system comprises spherical wedges, each covering 90 degrees in all three dimensions, and centered on the $+x$, $-x$, $+y$, $-y$, $+z$, and $-z$ axes. These wedges provide complete coverage of a region between $r_\text{min}$ and $r_\text{max}$, with smooth inner and outer boundaries, and a cubical patch enclosing $r < r_\text{min}$ where standard \texttt{Carpet}~\cite{Schnetter:2003rb} box-in-box mesh refinement can be applied.

To set up the gravitational initial data, we used \texttt{TwoPunctures}~\cite{Ansorg:2004ds} with parameters that emulate the GW150914 merger. Gravitational evolution was performed using \texttt{ML\_BSSN}~\cite{Brown:2008sb,Kranc:web,McLachlan:web} with a $1+\log$ slicing condition and a $\Gamma$-driver shift. We employed 7 levels of mesh refinement boxes via \texttt{Carpet}, centered around each of the punctures and tracking their motion via \texttt{PunctureTracker}. We developed the \texttt{KleinGordon Thorn}, which we describe in this section, to evolve a massless scalar field with initial data provided by a multipolar Gaussian function, centered around the origin with $R_0 = 15.0$, width $\sigma = 1.0$, and the only non-zero multipolar coefficient being $c_{00} = 1 / A_{00}$. This spherically symmetric initial data allowed us to activate reflection symmetry around the $+z$ plane in the code, which saved computational time, memory, and storage space.

In the gravitational evolution code, spatial derivatives were provided by \texttt{SummationByParts}~\cite{Diener:2005tn}, which provides finite difference stencils satisfying the summation-by-parts property with embedded Kreiss-Oliger type artificial dissipation. In \texttt{KleinGordon}, we utilized handcrafted central finite difference stencils and added artificial Kreiss-Oliger dissipation to the evolved Klein-Gordon fields via \texttt{Dissipation}. The gravitational evolution code projected its derivative operators into the global coordinate system utilized by \texttt{Llama} via \texttt{GlobalDerivative}, but in \texttt{KleinGordon}, this projection was also handcrafted via the Jacobian and Jacobian derivative components provided by \texttt{Llama}. Code generation for these projections can be found in the \texttt{Notebooks/equations.nb} notebook. Both evolutions utilized 8th order finite difference stencils, and radiating boundary conditions were employed in the outer boundaries via \texttt{NewRad}.

The time integration scheme utilized in this study was method of lines, provided by the \texttt{MoL Thorn}. A 4th order Runge-Kutta method with fixed time step was employed for the time integration process. It is important to note that \texttt{Carpet} employs different time steps for various refinement levels, thus the time step specified in the parameter files pertains to the coarsest level. The final integration time was selected in conjunction with the outer boundary radius to minimize interference from reflections at the outer boundaries, inter-patch boundaries, and inter-refinement level boundaries on the scalar field signal being measured.

At regular time intervals, data was extracted at $7$ different radii using \texttt{Multipole}, as well as 2D and 3D field data evolved by \texttt{KleinGordon}. Additionally, statistics such as puncture locations were recorded, which enabled the construction of visualizations of the simulated data. These results will be presented shortly. The parameter file used for this evolution can be located in the specified GitHub repository under \texttt{KleinGordon/par/GW150914/GW150914\_Scalar\_field.par}.