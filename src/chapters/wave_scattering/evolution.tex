To configure the evolution of Eqs.~\eqref{eq:wave_scattering_a3c} and \eqref{eq:wave_scattering_a3d} on top of the GW150914 event simulation within the \texttt{EinsteinToolkit}, the demonstration parameter file that reproduces this event, which can be found in Ref.~\cite{GW150914Demo}, was used as a basis and modified in order to accommodate the \texttt{KleinGordon} thorn. To set up the gravitational initial data, the \texttt{TwoPunctures Thorn}~\cite{Ansorg:2004ds} was utilized and spacetime evolution was performed using \texttt{ML\_BSSN}~\cite{Brown:2008sb,Kranc:web,McLachlan:web}.

As explained in Sec.~\ref{ch:wave_scattering:sec:multipatch}m the coordinate system used in our grid setup was the \texttt{Thornburg04} coordinates, implemented through the \texttt{Llama} infrastructure covering the region between $r_\text{min}$ and $r_\text{max}$, with smooth inner and outer boundaries, with an added a cubical patch enclosing $r < r_\text{min}$ where standard \texttt{Carpet}~\cite{Schnetter:2003rb} box-in-box mesh refinement can be applied.

We employed 7 levels of mesh refinement boxes via \texttt{Carpet}, centered around each of the BH centers. The motion of the BHs during the evolution was tracked via the \texttt{PunctureTracker Thorn}. Once the BHs were moved by a certain configurable threshold, \texttt{Carpet} ``regrided'' the simulation domain, that is, it updated the positions of the refinement boxes so that they keep following the BHs.

We have utilized the newly developed \texttt{KleinGordon Thorn}, to evolve a massless scalar field with initial data provided by a multipolar Gaussian function, centered around the origin with $R_0 = 15.0$, which is large enough to engulf both BHs in their initial configuration. Additionally, we set the Gaussian width $\sigma$ to $1.0$, and the only non-zero multipolar coefficient being $c_{00} = 1 / A_{00}$. This spherically symmetric initial data allowed us to activate reflection symmetry around the $+z$ plane in the code, which saved computational time, memory, and storage space.

As was explained in Sec.~\ref{ch:wave_scattering:sec:multipatch}, \texttt{KleinGordon} handcrafted 8th order central finite difference derivative operators were utilized and projected onto global coordinates utilizing the Jacobian and Jacobian derivative components provided by \texttt{Llama}. Artificial Kreiss-Oliger dissipation was added to the evolved Klein-Gordon fields via the \texttt{Dissipation Thorn}. Radiating boundary conditions were employed in the outer boundaries via the \texttt{NewRad Thorn}.

The time integration scheme utilized was the method of lines~\cite{LeVeque_Randall_J_2007-09-06}, provided by the \texttt{MoL Thorn}. A 4th order Runge-Kutta method with fixed time step was employed for the time integration process. It is important to note that \texttt{Carpet} employs different time steps for various refinement levels, thus the time step specified in the parameter files pertains to the coarsest level. The final integration time was selected in conjunction with the outer boundary radius to minimize interference from reflections at the outer boundaries, inter-patch boundaries, and inter-refinement level boundaries on the scalar field signal being measured.

At regular time intervals, data was extracted at $7$ different radii using \texttt{Multipole}, as well as 2D and 3D field data evolved by \texttt{KleinGordon}. Additionally, statistics such as puncture locations were recorded, which enabled the construction of visualizations of the simulated data. These results will be presented on the next section. The parameter file used for this evolution can be located in the specified GitHub repository under \texttt{KleinGordon/par/GW150914/GW150914\_Scalar\_field.par} in Ref~\cite{FieldPerturbationsRepo}.