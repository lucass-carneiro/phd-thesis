In this chapter, we will focus on the numerical simulation of a scalar wave scattering off a black hole binary that mirrors the first event detected by LIGO, know as GW150914. Due to the challenges imposed by the strong gravity regime and the highly dynamic nature of astrophysical binary systems already discussed in Chapter \ref{ch:introduction}, to accurately simulate wave scattering around black hole binaries it is necessary to solve Einstein's field equations numerically. In recent years, significant progress has been made in the numerical simulation of black hole binaries thanks to state-of-the-art strongly hyperbolic reformulations of the field equations, such as the BSSN formulation~\cite{PhysRevD.52.5428,PhysRevD.59.024007} and advancements in the understanding of stability criteria in partial differential equation discretization schemes employing techniques such as summation by parts~\cite{Diener2007}. These techniques have enabled us to simulate the dynamics of black hole binaries and their gravitational wave emission with unprecedented accuracy, stability and reliability.

Numerical evolution through self-coding can be a daunting task. This is where the employment of pre-existing tools, such as the \texttt{EinsteinToolkit}~\cite{EinsteinToolkit:2022_11}, becomes beneficial. The \texttt{EinsteinToolkit} is, according to its home page, ``a community-driven software platform of core computational tools to advance and support research in relativistic astrophysics and gravitational physics.''. The \texttt{EinsteinToolkit} is composed of a variety of software modules, referred to as \texttt{Thorns}, that are integrated into the \texttt{Cactus}~\cite{Cactuscode:web,Cactusprize:web,Goodale:2002a} computational framework. Each \texttt{Thorn} has the ability to perform a range of tasks, including infrastructure tasks such as file input/output, adaptive mesh refinement (AMR), as well as physics-related tasks such as the evolution of binary systems, the extraction of gravitational waves, and the multipole decomposition of a signal.

In the following sections, I will introduce the equations utilized for the evolution of a scalar wave atop a numerically metric without back-reaction, where the spacetime evolution disregards the scalar field's contribution to its stress-energy-momentum tensor. Subsequently, I will provide a detailed description of the \texttt{EinsteinToolkit Thorn KleinGordon} responsible for this evolution, and present the results that have been obtained thus far. The scalar field evolution code was developed by the author and its available, together with useful resources, in the \texttt{GitHub} repository~\cite{FieldPerturbationsRepo}. It is hoped that these results will offer valuable insights into the dynamics of the relaxation of astrophysical binary systems when perturbed.