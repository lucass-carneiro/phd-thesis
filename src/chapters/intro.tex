\section{The relevance of binary systems}

The astrophysical black hole (BH) is a vacuum solution of Einstein's field equations, and its mathematical description depends on two parameters: its mass and angular momentum~\cite{1986bhmp.book.....T}. Hence, BHs are one of the simplest objects in astrophysics from a mathematical standpoint.

Despite their mathematical simplicity, BHs exhibit rich physical interactions with their surroundings. These interactions can be classical, such as the formation of accretion disks around them~\cite{Abramowicz2013} or their interaction with other matter fields~\cite{Ficarra2023}, which generate effects like superradiance~\cite{PhysRevD.87.043513}. Alternatively, they can be semiclassical, such as Hawking radiation~\cite{Wald2001}. Recently, some of these phenomena, specifically superradiance and Hawking radiation, have been studied in the laboratory context using analogue models of gravity~\cite{Barcel2011} as reported in Refs.~\cite{Torres2017} and \cite{Kolobov2021}, respectively.

One of the most well-known predictions of the Theory of General Relativity (GR) is the existence of spacetime oscillations that propagate throughout the universe, known as gravitational waves. While this may seem to be an orthogonal topic to that of astrophysical black holes, the two are closely related. In 1975, the binary system PSR B1913$+$16 provided the first (indirect) evidence of the existence of gravitational waves \cite{1975ApJ...195L..51H}. The observational data was consistent with theoretical analysis for neutron star binaries that emit gravitational waves.

Direct observations of gravitational waves, which had been attempted since the 1960s, only became a reality in 2015 with measurements from the Advanced Laser Interferometer Gravitational-wave Observatory (Advanced LIGO) \cite{grav1,grav2}, ushering in a new era in astronomy and cosmology. These observations also involve binary systems, in which two astronomical objects are close enough for their mutual gravitational attraction to cause them to orbit around a common center of mass. As they get closer and closer, a catastrophic collision event occurs, releasing huge amounts of energy and angular momentum in the form of gravitational radiation. Therefore, we can view gravitational waves as an interesting effect arising from the interaction of black holes with each other.

So far, no analytic description of a binary system in GR is known. This is due to the extreme complexity of Einstein's equations. Even if such an exact solution were to be found, it would likely be too large, complicated and impractical to use. Given that in a collision event the nonlinear character of the equations becomes important the problem must be treated numerically, with the techniques of the field of Numerical Relativity (NR).

One of the objectives of this thesis is to investigate interaction effects in General Relativity that are well understood in spacetimes containing a single astrophysical object (such as a star or a black hole) and extend them to binary systems.

\section{Modeling binary systems}

Our work employs different types of models to describe binary systems. Each of these models has different characteristics and nomenclature.  We will now define precisely the terms that will be used throughout the text to refer to different binary black hole models.

\begin{definition}[Static/Dynamic BBH model]
  A \textbf{static} BBH model represents two black holes that \textbf{do not move} with respect to the observers that will participate in the PP. This is in contrast to a \textbf{dynamic} BBH model.
\end{definition}

\begin{definition}[Exact/Approximate BBH model]
  An \textbf{exact} BBH model is one that is an exact vacuum solution of Einstein's field equations. An \textbf{approximate} BBH model is one that is not an exact vacuum solution of said equations, and the deviation from vacuum is not considered to be an exotic type of matter but a measure of its ``non-exactness''.
\end{definition}

\begin{definition}[Analitic/Numeric BBH model]
  An \textbf{analytic} BBH model is one whose entire spacetime metric is analytically known at all points and times. In contrast, a \textbf{numeric} BBH model is obtained only at a certain coordinate time hypersurface (typically $t=0$) by solving the Arnowit-Deser-Misner (ADM) constraint equations for a vacuum configuration numerically. Note that even tough such numerical solutions contain errors, and thus do not strictly solve the vacuum field equations (giving rise to constraint violations), we do not consider these to be approximate, since it is possible (at least in theory but not in practice due to physical limitation of machines and computational resources) to obtain these solutions with arbitrarily small constraint violations given that they are convergent. For more information on solving the ADM constraint equations and obtaining initial data in Numerical Relativity, see for instance Chapter 3 of Ref~\cite{Alcubierre2012-xp}
\end{definition}

\section{The Penrose process}

One instance of a black hole BH interaction with its surrounding that will be extensively discussed is the Penrose Process (PP). The first incarnation of the PP arose as a consequence of the Kerr metric. The Kerr metric is the best-known mathematical description of rotating black holes given by the theory of General Relativity~\cite{Visser:2007fj,Bambi:2011mj,Teukolsky:2014vca,berti}.

Unlike static black holes, a Kerr black hole is characterized by the existence of a very peculiar region around its event horizon, known as the \emph{ergosphere} or \emph{ergoregion}. Particles that reach the ergosphere can still avoid the event horizon and, hence, are not doomed to end at the spacetime singularity. Nevertheless, any observer lying inside the ergosphere is unavoidably dragged along by the rotational motion of the black hole.

Particles inside the ergosphere may have negative energies (according to a static observer at infinity). Relying on this property, Penrose and Floyd devised a mechanism that allows one to extract energy from a rotating black hole~\cite{PENROSE1971}. The idea consists of sending a particle from infinity towards the black hole and assuming that, once inside the ergosphere, it decays into two other particles. If one of the fragments is counter-rotating with the black hole and has negative energy (which implies that the split happens inside the ergosphere), it will be captured by the black hole, meaning that the other fragment will escape to infinity. Due to the conservation of the four-momentum, the escaping fragment will have more energy and more angular momentum than the incident particle. Rotational energy and angular momentum are, thus, effectively extracted from the black hole.

The main motivating factor for this investigation was that despite being a well-known process, the PP had not been studied in the context of black hole binaries. Moreover, many recent research endeavors attempt to establish a relation of the PP with observable astrophysical phenomena.

The collisional version of the process, for instance, considers multiple particles that collide and scatter in the ergoregion, allowing for arbitrarily high center-of-mass energies. This process can potentially act as a mechanism to eliminate dark matter particles near a supermassive black hole~\cite{Schnittman:2018ccg}.

The magnetic PP~\cite{Wagh1985,Tursunov:2019oiq}, on the other hand, considers charged particles and black holes surrounded by magnetic fields (originated, for instance, by plasma accretion disks around the black hole). The electromagnetic interaction allows for highly efficient energy extraction schemes, such as the one introduced in Ref.\cite{Tursunov:2020juz} to model the emission of ultra-high energy cosmic rays from supermassive black holes. Furthermore, recent numerical simulations of plasma and jets around Kerr black holes indicate the important role that negative energy particles and the PP play on the total energy flux coming from the black hole's jets\cite{Parfrey:2018dnc}.

Recognizing these already discussed difficulties of modeling astrophysical binary systems, we have utilized BBH models that are static, exact and analytic in this investigation.

This discussion can be found in Chapter \ref{ch:penrose_binaries}, where we discuss energy extraction via the Penrose mechanism of different binary black hole models. The contents of Sec.~\ref{ch:penrose_binaries:sec:mp_penrose} and \ref{ch:penrose_binaries:sec:cmmr_penrose}, regarding non-coalescing binaries, have been published in Ref.~\cite{PhysRevD.104.124025}. The contents of Sec.~\ref{ch:penrose_binaries:sec:arbitrary_penrose} extend the discussion to non-static binaries and present novel preliminary results that we intend to publish soon.

\section{Black Hole Perturbations and Quasinormal Modes - Wave Scattering}

When a closed physical system (like a guitar string) is perturbed, it relaxes by emitting certain natural frequencies know as \emph{normal modes}. If, however, the system is open (and therefore energy is being somehow dissipated away), its emitted natural frequencies will decay with time (like for instance sounding a bell in a church). These decaying modes are called \emph{quasinormal modes} (QNMs).

Quasinormal frequencies can be used to obtain information about the system that produces them and black holes, like church bells, are also subject to these phenomena: Perturbed black holes relax by emitting waves in characteristic frequencies that decay with time, thanks to the dissipative nature of the event horizon. See Refs.~\cite{review1, review2, review3, review4} for an in-depth review of the quasinormal mode problem in the context of general relativity and black holes.

Determining these characteristic frequencies quickly and accurately for a large range of models is important for many practical reasons. It has been shown that the gravitational wave signal emitted at the final stage of the coalescence of two compact objects is well described by quasinormal modes~\cite{buonanno,seidel} (note that recent developments during the elaboration of this thesis have called this fact into question. See~\ref{PhysRevLett.130.081401} for details). This means that if one has access to a database of quasinormal modes and of gravitational wave signals from astrophysical collision events, it is possible to characterize the remnant object using its quasinormal frequencies. Since there are many models that aim to describe remnants, being able to compute the quasinormal frequencies for such models reliably is paramount for confirming or discarding them.

In the context of binary systems, in Chapter~\ref{ch:wave_scattering} we will focus on the numerical simulation of a scalar wave scattering off a black hole binary that mirrors the first event detected by LIGO, know as GW150914. In our simulations, the spacetime metric will not ``back-react'' to the presence of the field. That means that the scalar wave shall propagate on top of the evolving black hole binary without interfering in the stress-energy tensor of the spacetime. An analysis of the scattered signal will be made in an attempt to identify quasinormal ringing of the binary system.

In recent years, significant progress has been made in the numerical simulation of black hole binaries thanks to state-of-the-art strongly hyperbolic reformulations of the field equations, such as the BSSN formulation~\cite{PhysRevD.52.5428,PhysRevD.59.024007} and advancements in the understanding of stability criteria in partial differential equation discretization schemes employing techniques such as summation by parts~\cite{Diener2007}. These techniques have enabled us to simulate the dynamics of black hole binaries and their gravitational wave emission with unprecedented accuracy, stability and reliability.

Numerical evolution through self-coding can be a daunting task. This is where the employment of pre-existing tools, such as the \texttt{EinsteinToolkit}~\cite{EinsteinToolkit:2022_11}, becomes beneficial. The \texttt{EinsteinToolkit} is, according to its home page, ``a community-driven software platform of core computational tools to advance and support research in relativistic astrophysics and gravitational physics.''. The \texttt{EinsteinToolkit} is composed of a variety of software modules, referred to as \texttt{Thorns}, that are integrated into the \texttt{Cactus}~\cite{Goodale:2002a} computational framework. Each \texttt{Thorn} has the ability to perform a range of tasks, including infrastructure tasks such as file input/output, adaptive mesh refinement (AMR), as well as physics-related tasks such as the evolution of binary systems, the extraction of gravitational waves, and the multipole decomposition of a signal.

The outcomes reported in this chapter are preliminary and exploratory in nature. They do not possess the necessary level of maturity for publication. However, we have deemed it appropriate to incorporate them in this thesis as they demonstrate crucial skills that were acquired during the course of this research. Additionally, they can serve as a foundation for more advanced analyses in the future.

It should be noted that while we were drafting this chapter, another PhD thesis authored by G. Ficarra was made publicly available and can be found in Ref.~\cite{Ficarra2023}. In his thesis, Ficarra has carried out similar work with significantly more detail and precision, as his findings are not preliminary and his thesis is wholly dedicated to that topic. It is not our intention to compete with his work. Rather, we aim to highlight the contributions we have made and the skills gained in doing so. If one is interested in sophisticated analysis of binary systems interacting with scalar fields, including instances where the gravitational potential is influenced by the scalar cloud, we recommend reading his thesis.

\section{Black Hole Perturbations and Quasinormal Modes - \texttt{QuasinormalModes.jl}}

Frequently, the task of computing quasinormal modes of a given system can be reduced to determining the discrete set of eigenvalues of a second-order differential equation subject to appropriate boundary conditions and asymptotic behavior~\cite{review3}. Several techniques can be employed to address such problems, but Leaver's continued fraction method~\cite{Leaver1985} stands out as the most widely used and successful method for a broad range of spacetimes.

Nevertheless, Leaver's method may not always yield satisfactory results, such as in the case of determining the quasinormal modes of an extreme Reisnner-N"ordstrom black hole, where additional treatments are required~\cite{PhysRevD.93.064062}. Therefore, Leaver's method cannot be regarded as universally applicable and may require manual adjustments for specific cases.

Similarly, the pseudospectral method, another popular approach to solving eigenvalue problems in ordinary differential equations, may encounter difficulties. This method constructs matrices from the original ODE, and if the sought eigenvalue is not a polynomial function, manual adaptations must be performed, rendering the method non-general and specific to certain problems. While this is not prohibitively restrictive to most astrophysics problems, it can be an important limitation in other areas

There is also a lack of generalists, free (both in the financial and license-wise sense) open source tools that compute quasinormal modes generally. More precisely, there are tools which are free and open source, but run on top of a proprietary paid and expensive software framework such as the ones developed in Refs.~\cite{qnmspectral,spectralbp}, which are both excellent packages that can be obtained and modified freely but, unfortunately, require the user to own a license of the proprietary \texttt{Wolfram Mathematica} CAS. There are also packages that are free and run on top of \texttt{Mathematica} but are not aimed at being general eigenvalue solvers at all, such as the one in Ref.~\cite{bhpt_quasinormalmodes}, that can only compute modes of Schwarzschild and Kerr black holes. Finally, the \texttt{Python} package in Ref.~\cite{bhpt_qnm} is open source and free but can only compute Kerr quasinormal modes.

Fortunately, a new numerical technique was recently developed for tackling these problems, known as the \emph{Asymptotic Iteration Method} (AIM). The groundwork of the technique was laid out in Ref.~\cite{aim_original} and in Ref.~\cite{aim_improved} the method was refined and adapted to GR. Given its generality, the method is directly applicable to cases where other methods require greater care.

We have implemented and published \texttt{QuasinormalModes.jl} (see the accompanying paper in Ref.~\cite{Sanches2022}), a \texttt{Julia}~\cite{Bezanson2017} software package for finding quasinormal modes using the AIM. Not only that, the package can be used to compute the discrete eigenvalues of \emph{any} second order homogeneous ODE (such as the energy eigenstates of the time independent Schrödinger equation or spin weighted spheroidal harmonics, defined in Ref.~\cite{PhysRevD.73.024013}) provided that these eigenvalues actually exist.

\texttt{QuasinormalModes.jl} fills the existing gap for free, open source tools that are able to compute discrete eigenvalues (and in particular, quasinormal modes) efficiently for a broad class of models and problems. The package features a flexible and user-friendly API where the user simply needs to provide the coefficients of the problem ODE after incorporating boundary and asymptotic conditions on it.

The user can also choose to use machine or arbitrary precision arithmetic for the underlying floating point operations involved and whether to do computations sequentially or in parallel using threads. The API also tries not to force any particular workflow on the users so that they can incorporate and adapt the existing functionality on their research pipelines without unwanted intrusions.

Often user-friendliness, flexibility and performance are treated as mutually exclusive, particularly in scientific applications. By using \texttt{Julia} as an implementation language, the package can have all these features simultaneously.

The package was used in Ref.~\cite{Mamani2022} where perturbations of various spins of the Schwarzschild metric were considered. Novel frequencies were obtained and results that were previously available were compared against literature values, while also cross-checking results for the same models obtained via the more traditional pseudo-spectral method.

In chapter~\ref{ch:qnm_aim}, we describe the AIM in detail, it's implementation in \texttt{QuasinormalModes.jl} and the perturbation equations and results obtained in Ref.~\cite{Mamani2022} using the package.