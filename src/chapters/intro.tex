One of the most well know predictions of the Theory of General Relativity (GR) is the existence of spacetime oscillations that propagate throughout the universe, the so called, gravitational waves. The first (indirect) evidence of the existence of such waves occurred when Hulse and Taylor observed, in 1975, the binary system PSR B1913$+$16~\cite{1975ApJ...195L..51H}. Observational data proved to be compatible with theoretical analysis for neutron star binaries that emit gravitational waves. On the other hand, direct observations, despite being attempted since the 1960's, only became a reality recently in the year 2015, with measurements from the Advanced Laser Interferometer Gravitational-wave Observatory (Advanced LIGO)~\cite{grav1,grav2} giving birth to a new era in astronomy and cosmology. These two observations have a common factor: they involve \emph{binary systems}, that is, two astronomical objects that are close enough for their mutual gravitational attraction to cause them to orbit around a common center of mass as they get closer and closer until a catastrophic collision event occurs, releasing huge amounts of energy and angular momentum in the form of gravitational radiation.

So far, no analytic description of a binary system in GR is know. This is due to the extreme complexity of Einstein's equations (a system of 10 coupled non-linear partial differential equations) whose exact solution is know only for very specific systems with a high degree of symmetry. Even if such an exact solution were to be found, it would likely be too large, complicated and and impractical to use. Given that in a collision event the nonlinear character of the equations becomes important the problem must be treated numerically, with the techniques of the field of Numerical Relativity (NR).

The main goal of our work is to investigate how classical and semi-classical effects in GR that are well know to spacetimes containing a single astrophysical object (a star or a black hole), namely the \emph{quasinormal ringdown}, \emph{the Penrose process} and \emph{Hawking radiation} extend to binary systems. To that end, we will work on two fronts: The first is a \emph{semi-analytic} approach in which we will study exact (and analytic) solutions of Einstein's equations that approximate binary black hole systems in static equilibrium. Such solutions allow us to employ numerical and semi-analytic techniques that are well know to our research group. The second is a \emph{numeric} approach in which we (with the help of international collaborators) will use numeric approximations of binary systems that are no longer in static equilibrium for our investigations.

In astrophysical contexts any excess of electric charge in a black hole tends to be quickly neutralized~\cite{gibbons1975}. For this reason, along with the existence and unicity theorems~\cite{Chrusciel2012,PhysRevLett.114.151102}, the most accepted description for an astrophysical black hole is given by the Kerr metric. Despite this fact, Einsten's equations when coupled to Maxwell's equations (a system know as the Einstein-Maxwell equations) admits solutions that represent black holes that are charged electrically and/or magnetically. The non-null electric charge allows for the existence of exact-many body solutions to the Einstein-Maxwell system. On the semi-analytic front we will make use of such charged spacetime that was obtained independently by S. D. Majumdar~\cite{MAJUMDAR1947} and A. Papapetrou~\cite{PAPAPETROU1947}, also know as the Majumdar-Papapetrou (MP) solution. This solution represents a binary system composed of two extremally charged Reissner-N\"ordstrom black holes~\cite{HARTLE1972}. Despite containing two charged static black holes, the MP solution can be considered a good approximation for a frontal collision of two non-charged black holes if their approximation velocities are much smaller than the speed of light~\cite{BINI2019}.

It's natural to ask if the MP solution can be generalized to represent systems of multiple charged Kerr-Newman black holes, which would add another degree of ``realism'' to the analytic treatment. It turns out that such generalization exists and were discovered by W. Israel, G. A. Wilson~\cite{ISRAEL1972} and A. Perj\'es~\cite{PERJES1971} and became know as the Israel-Wilson-Perj\'es (IWP) class. Unfortunately, these solutions cannot represent black holes -- they instead must always be naked Kerr-Newman singularities. In order to remain more astrophysically relevant, we will employ another class of solutions that is able to describe static Kerr binaries (and not only naked singularities) that was found very recently by Cabrera-Munguia, Manko and Ruiz (CMMR)~\cite{CABRERA2018,MANKO2019,MANKO2020}. In this solution, the Kerr binaries are kept static thanks to a massless ``strut'' that is represented by a conical singularity that keeps the black holes from colliding. Despite the ``nonphysical'' strut, such solution was used to compute the shadow of a binary black hole system and showed a good agreement with the shadow computed using a fully numerical binary system, as was shown by Cunha \emph{et. al.} in \cite{Cunha_2018}. The same group showed in \cite{Cunha_2018_2} that the massless strut does not influence the motion of photons composing the shadow.

