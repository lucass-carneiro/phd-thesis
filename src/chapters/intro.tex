One of the most well know predictions of the Theory of General Relativity (GR) is the existence of spacetime oscillations that propagate throughout the universe, the so called, gravitational waves. The first (indirect) evidence of the existence of such waves occurred when Hulse and Taylor observed, in 1975, the binary system PSR B1913$+$16~\cite{1975ApJ...195L..51H}. Observational data proved to be compatible with theoretical analysis for neutron star binaries that emit gravitational waves. On the other hand, direct observations, despite being attempted since the 1960s, only became a reality recently in the year 2015, with measurements from the Advanced Laser Interferometer Gravitational-wave Observatory (Advanced LIGO)~\cite{grav1,grav2} giving birth to a new era in astronomy and cosmology. These two observations have a common factor: they involve \emph{binary systems}, that is, two astronomical objects that are close enough for their mutual gravitational attraction to cause them to orbit around a common center of mass as they get closer and closer until a catastrophic collision event occurs, releasing huge amounts of energy and angular momentum in the form of gravitational radiation.

So far, no analytic description of a binary system in GR is known. This is due to the extreme complexity of Einstein's equations (a system of 10 coupled non-linear partial differential equations) whose exact solution is known only for very specific systems with a high degree of symmetry. Even if such an exact solution were to be found, it would likely be too large, complicated and impractical to use. Given that in a collision event the nonlinear character of the equations becomes important the problem must be treated numerically, with the techniques of the field of Numerical Relativity (NR).

The main goal of our work is to study certain effects in GR that are well-know to spacetimes containing a single astrophysical object (a star or a black hole), namely the \emph{quasinormal ringdown} and \emph{the Penrose process} (PP), and extend them to binary systems. To that end, we will work on two fronts: The first is a \emph{semi-analytic} approach in which we will study exact (and analytic) solutions of Einstein's field equations that approximate binary black hole systems in static equilibrium. Such solutions allow us to employ well established numerical and semi-analytic techniques in our studies. The second is a \emph{numeric} approach in which we will use non-exact approximations of binary systems that are no longer in static equilibrium for our investigations.

This thesis is organized as follows: In Chapter~\ref{ch:penrose_binaries}, we discuss energy extraction via the Penrose mechanism of different binary black hole models. The contents of Sec.~\ref{ch:penrose_binaries:sec:mp_penrose} and \ref{ch:penrose_binaries:sec:cmmr_penrose}, regarding non-coalescing binaries, have been published in Ref.~\cite{PhysRevD.104.124025}. The contents of Sec.~\ref{ch:penrose_binaries:sec:arbitrary_penrose} extend the discussion to non-static binaries and present novel preliminary results that we intend to publish soon.

In Chapter~\ref{ch:qnm_aim}, we look at computing quasinormal modes of black holes via the asymptotic iteration method, a recently proposed technique for computing the eigenvalues of second order ODEs. The contents of this chapter resulted in the creation of a software package for performing these computations with a companion paper found in Ref.\cite{Sanches2022}. We also present novel results obtained with the package for perturbations of various spins that were published in Ref.~\cite{Mamani2022}.

In Chapter~\ref{ch:wave_scattering} we introduce and discuss a numerical code for simulating the scattering of a scalar field by an arbitrary background spacetime and present the preliminary results of the simulation of scattering this field in the background of a simulation of the GW150914 event.

Finally, we conclude the text in Chapter~\ref{ch:conclusion} by reviewing the main conclusions of each individual chapter while also providing perspectives on the continuation of each individual work.

% Black holes can be considered the simplest objects in nature: it is the vacuum solution of the Einstein Eqs. However, their interations with the surrounding is still not fully understood. Studying its properties is essential to understand its nature. From the study of BH perturbation theory many insights can the learned, some of which were proven to be detectable, like the QNMs from binary black hole merger, which was detected in the recents gravitational wave observation [cite]. (talvez falar da banheira do mau mau, mas acho q GW tem mais marketing e deixa a banheira do maumau quando vc fala q vai falar de penrose)

% A pertubed BH is described as bla bla bla (aqui vc gastas umas 2 paginas falando como descreve um BH perturbado pode usar equação)

% One very interesting phenomenon that emerges from the perturbation of a single BH is the penrose effect. (aqui vc explica em palavas o q é o processo de penrose e cita a banheira do maumau)

% (aqui acho q vc pode juntar espalhamento com QNMs)

% These phenomena are well known for \emph{single} astrophysical objects. With the achievement of direct observation of gravitational waves made by the LIGO collaboration in 2015, the existence of binary black holes were confirmed. By the date, almost a hundred (vc pode checar apra ver o numero certo) of these systems were detected, which highlights the importance of understanting better binary spacetimes. The main goal of this thesis is to extend the bla bla bla (continua com o resto q vc escreveu)