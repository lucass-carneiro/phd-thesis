\section{Semi analytic approximation of the MP quasinormal modes}

In this section we will compute the quasinormal modes of the MP spacetime by separating and solving the scalar wave equation. For a general separation parameter $a$ it's impossible to separate the wave equation. We will show, however, that it's possible to separate the wave equation when the separation of the binary is large or small.

\subsection{MP metric centered around one black hole}

In order to facilitate the imposition of the quasinormal boundary conditions later on, we will rewrite the MP metric using spherical coordinates that are centered around the black hole of mass $M_1$ as described in~\cite{SMERAK2016}. This amounts to transforming Weyl's cylindrical coordinates using

\begin{equation}
  \rho = (r - M_1)\sin\theta, \quad z = (r - M_1)\cos\theta + a
  \label{ch:penrose_binaries/eq:weyl_to_spherical_transform}
\end{equation}

%
which recasts the metric in Eq.~\eqref{ch:penrose_binaries/ch:penrose_binaries/eq:majumdar_papapetrou_line_element} as

\begin{equation}
  \ud s^2 = -U(r,\theta)^{-2} \ud t^2 +  U(r,\theta)^2\ud r^2 + U(r,\theta)^2\left( 1 - \frac{M1}{r} \right)^2r^2\left[ \ud\theta^2 + \sin^2\theta\ud\phi^2 \right],
  \label{ch:penrose_binaries/eq:majumdar_papapetrou_line_element_spherical}
\end{equation}
%
where

\begin{equation}
  U(r,\theta) = \left(1 - \frac{M_1}{r} \right)^{-1} + \frac{M_2}{\sqrt{(r - M_1 + 2a\cos\theta)^2 + 4a^2\sin^2\theta}}.
  \label{ch:penrose_binaries/eq:mp_metric_potential_spherical}
\end{equation}

Notice that in the limit $M_2 = 0$ or $a\rightarrow\infty$, this form of the metric yields the extreme Reissner-Nordstr\"om line element. For reasons that shall become clear later on, we will define two auxiliary functions,

\begin{equation}
  F(r) \equiv \left(1 - \frac{M_1}{r} \right)
  \label{ch:penrose_binaries/eq:part_f_of_u}
\end{equation}
%
and
\begin{equation}
  G(r,\theta) \equiv \frac{M_2}{\sqrt{(r - M_1 + 2a\cos\theta)^2 + 4a^2\sin^2\theta}}.
  \label{ch:penrose_binaries/eq:part_g_of_u}
\end{equation}
%
With these functions we can rewrite Eq.~\eqref{ch:penrose_binaries/eq:mp_metric_potential_spherical} as

\begin{equation}
  G(r,\theta) \equiv \frac{1}{F(r)} + G(r,\theta).
  \label{ch:penrose_binaries/eq:rewrite_of_u}
\end{equation}

The wave equation for a scalar field $\Psi$ with mass $\mu$ and electric charge $e$ in a curved spacetime endowed with the electromagnetic potential $\tens{A}{}{\nu}$ is given by

\begin{equation}
  \left[ \umtrtens{\mu}{\nu}(\tens{\nabla}{}{\nu} - ie\tens{A}{}{\nu})(\tens{\nabla}{}{\mu} - ie\tens{A}{}{\mu}) - \mu^2 \right]\Psi = 0.
  \label{ch:penrose_binaries/eq:wave_equation}
\end{equation}

By using the electromagnetic potential of Eq.~\eqref{ch:penrose_binaries/eq:electromagnetic_potential_mp} together with metric of Eq.~\eqref{ch:penrose_binaries/eq:majumdar_papapetrou_line_element_spherical} and choosing a solution \emph{ansatz} in the form

\begin{equation}
  \Psi(t,r,\theta,\phi) = \frac{R(r)}{r}S(\theta)e^{-i(\omega t - m\phi)},
  \label{ch:penrose_binaries/eq:rewrite_of_u}
\end{equation}
%
in Eq.~\eqref{ch:penrose_binaries/eq:wave_equation} and dividing the result by $F(r)^2 \psi (t,r,\theta ,\phi )$ we obtain

\begin{multline}
  \frac{\cot (\theta ) S^\prime(\theta )+S^{\prime\prime}(\theta )}{S(\theta )} + \frac{2 r^2 F(r) F^{\prime}(r) R^{\prime}(r)+r^2 F(r)^2 R^{\prime\prime}(r)}{R(r)} + r^2 (e-\mu ) (e+\mu )-m^2 \csc ^2(\theta ) \\
  + F(r)^{-2}r \left(2 F(r)^3 \left(r G(r,\theta ) ((e+\omega ) G(r,\theta ) (2 (e+\omega ) G(r,\theta )-3 e)+(e-\mu ) (e+\mu ))-F'(r)\right) \right. \\
  +r F(r)^4 G(r,\theta )^2 ((e+\omega ) G(r,\theta ) ((e+\omega ) G(r,\theta )-2 e)+(e-\mu ) (e+\mu ))\\
  +6 r (e+\omega ) F(r)^2 G(r,\theta ) ((e+\omega ) G(r,\theta )-e)\\
  \left. +2 r (e+\omega ) F(r) (2 (e+\omega ) G(r,\theta )-e)+r (e+\omega )^2\right) = 0.
  \label{ch:penrose_binaries/eq:unseparated_wave_eq}
\end{multline}

The resulting equation, despite being large and cumbersome, could be separable if not for the last term, that mixes $r$ and $\theta$. Let's assume that it's possible to perform an approximation of the form

\begin{equation}
  G(r,\theta) \approx H(r).
  \label{ch:penrose_binaries/eq:approx_for_g}
\end{equation}
%
If this is true, Eq.~\eqref{ch:penrose_binaries/eq:unseparated_wave_eq} could be easily separated using the separation constant $l(l+1)$ into an angular equation,

\begin{equation}
  S''(\theta ) + \cot (\theta ) S'(\theta )+ [l(l+1) - m^2 \csc ^2(\theta )]S(\theta) = 0,
  \label{ch:penrose_binaries/eq:sep_ang_eq}
\end{equation}
%
which is simply Legandre's equation in trigonometric form and a radial equation,

\begin{multline}
  \frac{2 r^2 F(r) F^{\prime}(r) R^{\prime}(r)+r^2 F(r)^2 R^{\prime\prime}(r)}{R(r)} + r^2 (e-\mu ) (e+\mu ) \\
  + F(r)^{-2}r \left(2 F(r)^3 \left(r H(r) ((e+\omega ) H(r) (2 (e+\omega ) H(r)-3 e)+(e-\mu ) (e+\mu ))-F'(r)\right) \right. \\
  +r F(r)^4 H(r)^2 ((e+\omega ) H(r) ((e+\omega ) H(r)-2 e)+(e-\mu ) (e+\mu ))\\
  +6 r (e+\omega ) F(r)^2 H(r) ((e+\omega ) H(r)-e)\\
  \left. +2 r (e+\omega ) F(r) (2 (e+\omega ) H(r)-e)+r (e+\omega )^2\right) = l(l+1).
  \label{ch:penrose_binaries/eq:sep_rad_eq}
\end{multline}
%
which can be readily solved using numerically if the proper boundary conditions are applied.

A similar separation by approximation procedure for the wave equation in the MP metric was introduced in Ref.~\cite{LAURA2019}. There the authors used the MP metric with two equal mass constituents in prolate spheroidal coordinates and expanded the whole metric potential $U$ for the regime when the separation of the black holes is much larger than their mass. The main disadvantage of using this scheme becomes apparent when trying to apply horizon boundary conditions to the separated wave equations as the position of the event horizons must be described by two coordinates (because the event horizons are points in these coordinates) and thus by imposing, e.g., ingoing boundary conditions at the singularity of the ``radial'' equation one is actually imposing a boundary condition over the whole coordinate axis that joins the two black holes. By using spherical coordinates centered around one of the black holes, this problem is somewhat mitigated as we can make sure that the boundary conditions applied to the separated equations are actually boundary conditions applied to the event horizon of at least one of the black holes. Furthermore we also relax the assumption of an equal mass binary. In our approach the use of spherical coordinates centered around one of the holes recasts the metric potential $U$ in a form that favors viewing the second hole as a perturbation to the first. By expanding only the part of the metric potential that mixes coordinates we have the added bonus of being able to separate the wave equation not only when the binary separation is large but also when the holes are very close together, a feat that is impossible using the method prescribed in Ref.~\cite{LAURA2019}.

\subsection{Small separation approximation}

By dividing both the numerator and denominator of Eq.~\eqref{ch:penrose_binaries/eq:part_g_of_u} by $M_1$ and Taylor expanding the result in powers of $a/M1$ around $0$ we have

\begin{multline}
  G(r,\theta) = \frac{M_2}{\sqrt{(M_1-r)^2}} + \frac{2 (a/M_1)  M_1 M_2 \cos (\theta ) \sqrt{(M_1-r)^2}}{(M_1-r)^3} \\
  + \frac{(a/M_1) ^2 M_1^2 M_2 (3 \cos (2 \theta )+1)}{\left((M_1-r)^2\right)^{3/2}} + O(a/M_1)^3.
  \label{ch:penrose_binaries/eq:g_taylor_exp_small}
\end{multline}
%
If we truncate the series to first order in $a/M_1$, we can approximate $G(r,\theta)$ as a function $H(r)$ as was considered in Eq.~\eqref{ch:penrose_binaries/eq:approx_for_g}, that is,

\begin{equation}
  H(r) = \frac{M_2}{r - M_1}.
  \label{ch:penrose_binaries/eq:h_for_small}
\end{equation}
%
where we've considered that $r \geq M_1$, since we are interested in solving the ODE only outside of the black holes. Plugging Eq.~\eqref{ch:penrose_binaries/eq:h_for_small} in Eq.~\eqref{ch:penrose_binaries/eq:sep_rad_eq} we get

\begin{multline}
  \left(\frac{M_1}{r}-1\right)^4 R^{\prime\prime}(r) - \frac{2 M_1 (M_1-r)^3}{r^5} R^{\prime}(r) \\
  + R(r) r^{-6}\left\{ M_1^2 r^2 \left[(e-\mu ) (e+\mu ) (M_2+r)^2-l (l+1)+6\right] \right. \\
  + r^2 \left[e^2 M_2^2 (M_2+r)^2+2 e M_2 \omega  (M_2+r)^3+r^2 \left(\mu ^2 \left(-(M_2+r)^2\right)-l (l+1)\right)+\omega ^2 (M_2+r)^4\right] \\
  \left. + 2 M_1 r^2 \left((M_2+r)^2 \left(e^2 M_2+e \omega  (M_2+r)+\mu ^2 r\right)+\left(l^2+l-1\right) r\right) -6 M_1^3 r + 2 M_1^4 \right\} \\
  = 0,
  \label{ch:penrose_binaries/eq:ode_close}
\end{multline}
%
which can be solved numerically to yield the QNMs of the binary system when it's constituents are close together.

\subsection{Large separation approximation}

Once again dividing both the numerator and denominator of Eq.~\eqref{ch:penrose_binaries/eq:part_g_of_u} by $M_1$ and Taylor expanding the result in powers of $a/M_1$ around $\infty$ (which is the same as Taylor expanding $M_1/a$ around $0$) we have

\begin{equation}
  G(r,\theta) = \frac{M_2}{2 (a/M_1)  M_1} + \frac{M_2 \cos (\theta ) (M_1-r)}{4 (a/M_1) ^2 M_1^2}+O\left(\frac{1}{(a/M_1) }\right)^3.
  \label{ch:penrose_binaries/eq:g_taylor_exp_large}
\end{equation}
%
Truncating the series to first order in $M_1/a$, we can approximate $G(r,\theta)$ as a function of $r$ and thus
\begin{equation}
  H(r) = \frac{M2}{2 a}.
  \label{ch:penrose_binaries/eq:h_for_large}
\end{equation}
%
Plugging Eq.~\eqref{ch:penrose_binaries/eq:h_for_large} in Eq.~\eqref{ch:penrose_binaries/eq:sep_rad_eq} we get

\begin{multline}
  \left(\frac{\text{M1}}{r}-1\right)^4 R^{\prime\prime}(r)-\frac{2 \text{M1} (\text{M1}-r)^3}{r^5} R^\prime(r)\\
  R(r) \left\{ (e+\omega )^2+\left(1-\frac{\text{M1}}{r}\right) \left[ \frac{2 \text{M2} (e+\omega )^2}{a}-2 e (e+\omega ) \right. \right. \\
  -\frac{\text{M2}^2 (\text{M1}-r)^3 (2 a (\mu -e)+\text{M2} (e+\omega )) (\text{M2} (e+\omega )-2 a (e+\mu ))}{16 a^4 r^3} \\
  \left(1-\frac{\text{M1}}{r}\right) \left(\frac{3 \text{M2} (e+\omega ) (\text{M2} (e+\omega )-2 a e)}{2 a^2}+e^2-\mu ^2-\frac{l (l+1)}{r^2}\right) \\
  \left. \left. \frac{2}{r} \left(\frac{\text{M1}}{r}-1\right)^2 \left(\frac{\text{M2} r \left(2 a^2 (e-\mu ) (e+\mu )-3 a e \text{M2} (e+\omega )+\text{M2}^2 (e+\omega )^2\right)}{4 a^3}-\frac{\text{M1}}{r^2}\right) \right] \right\} \\
  = 0
  \label{ch:penrose_binaries/eq:ode_far}
\end{multline}
%
which can be solved numerically to yield the QNMs of the binary system when it's constituents are far apart.

\subsection{Imposing QNM boundary conditions}

Having obtained two approximate QNM equations, Eq.~\eqref{ch:penrose_binaries/eq:ode_close} and Eq.~\eqref{ch:penrose_binaries/eq:ode_far}, we can proceed to impose the QNM boundary conditions to each equation. That means imposing that waves cannot leave neither of the black holes and also cannot come from infinity. Making sure that waves are only ingoing precisely on the event horizons is a difficult task in the MP metric. The coordinate systems that we've used places our focus on one of the black holes and allows us to describe this holes horizon as a spherical surface. That means that to impose ingoing boundary conditions in the black hole of mass $M_1$ translates to imposing a boundary condition when $r=M_1$ for any value of $\theta$. This is not the case with the black hole of mass $M_2$, as it is still a point-like singularity. To correctly impose the boundary conditions, we would need to also consider a boundary condition on a certain value of $\theta$. Because or separation of the wave equation is already an approximation, we will impose boundary conditions on the spherical horizon and at infinity regardless of the fact that the second black holes is a point-like structure. We believe this to be physically justifiable in the regime of large separations: Because the black holes are far apart it is as if the second black hole is at the border of the physical domain, close to infinity. By saying that waves are only leaving in this region we are implicitly saying that the second black hole is absorbing the waves.

If we consider a scalar field with no charge or mass, that is, $e = \mu = 0$, Eq.~\eqref{ch:penrose_binaries/eq:ode_close} simplifies to

\begin{multline}
  \left(\frac{M_1}{r}-1\right)^4 R^{\prime\prime}(r) - \frac{2 M_1 (M_1-r)^3}{r^5}R^\prime(r) \\
  + R(r) \left( \frac{\omega ^2 (M_1 M_2-r (2 a+M_2))^4}{16 a^4 r^4}+\frac{(M_1-r)^2 \left(-l (l+1) r^2+2 M_1^2-2 M_1 r\right)}{r^6} \right) \\
  = 0
  \label{ch:penrose_binaries/eq:ode_far_no_charge_no_mass}
\end{multline}

\section{Wave scattering in generic space-times}

In order to investigate the QNMs (and other aspects) of more astrophysically relevant space-times, we intend to employ numerical methods. More specifically we intend to evolve the scalar wave equation using a numeric background spacetime that is itself evolving in time. In order to do that we will follow closely the procedure outlined in ~\cite{ASSUMPCAO2018}

\subsection{ADM decomposition of the scalar wave equation}
\subsection{Proposal for investigating wave scattering in numeric space-times}