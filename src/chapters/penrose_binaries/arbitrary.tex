In the previous sections, all of our considerations depended upon the existence of a conserved negative and ``global'' (as seen by a static observer at infinity) energy. This occurs if the spacetime under consideration is stationary. In this section we will demonstrate how one can still observe the Penrose Process even when there is no global and conserved energy available, looking only at locally defined quantities. This technique allows one to study the Penrose mechanism even when the spacetime metric is defined numerically, such as is the case in numerical simulations of binary black hole collisions.

\subsection{3+1 split of the geodesic equation}

To understand our proposed technique, it is first fundamental to understand how General Relativity can be reformulated by explicitly separating its spatial and temporal components. This decomposition know as a 3+1 split is very commonly used in Numerical Relativity and was motivated by the first attempts of posing GR as a Cauchy problem and numeric spacetime metrics are often given in terms of it's 3+1 components. In this section, we will assume familiarity of the reader with this concept which can be readily reviewed in Refs.~\cite{Alcubierre2012-xp, 9780521514071, 9781108928250}. Given that the Penrose process requires us to investigate the trajectories of particles in a background spacetime, we must now solve the geodesic equation taking into account that the spacetime metric (and it's derivatives) will be provided via 3+1 split components. The need to solve the geodesic equation in this context overlaps with works that are interested in simulating an image of a black hole, that is, what a camera would capture if a picture of a BH was to be taken. In this type of simulation a technique called \emph{backwards ray tracing} is employed, which consists in choosing a position and orientation of a model camera and integrating the trajectory of the photons that hit the camera's ``film'' backwards in time. If the photons fall in the black hole, that pixel of the image will be black or colored otherwise.

Although our purposes differ, the mathematical tools used in backwards ray tracing are fundamental in our work. The complete and detailed derivation of the 3+1 split of the geodesic equation can be found in Ref.~\cite{Vincent_2012}. We shall recover only the nomenclatures and concepts necessary for the further development of our proposal. To begin with, we consider the background spacetime of interest to be described by a metric tensor $\mtrtens{\mu}{\nu}$ and to be globally hyperbolic, and thus admits a one parameter space-like foliation of constant coordinate time $t$ hypersurfaces that we shall denote by $\Sigma_t$. We will also assume that the space-time has coordinates\footnote{We use the convention that Greek indices run over all 4 coordinates while Latin indices run over only the spatial coordinates.} that are compatible with the foliation, that is, $x^0=t$ and $x^i$ span $\Sigma_t$. Let us denote the unit time-like (future directed) normal vector of $\Sigma_t$ by $n^\mu$. This vector coincides with the 4-velocity of an observer whose worldlines are orthogonal to $\Sigma_t$, which we call the \emph{Eulerian Observer} $\mathcal{O}_E$. We denote by $\gamma_{\mu\nu}$ the spatial metric induced in $\Sigma_t$, $D_i$ it's associated covariant derivative, $K_{ij}$ the extrinsic curvature tensor, $N$ the lapse function and $\beta^\mu$ the shift vector. The 3+1 split metric is thus
%
\begin{equation}
  \ud s^2 = -N^2 \ud t^2 + \gamma_{ij}(\ud x^i + \beta^i \ud t)(\ud x^j + \beta^j \ud t).
  \label{eq:arbitrary_penrose_decomposed_metric}
\end{equation}

Let us now consider a particle $\mathcal{P}$ of 4-momentum $p^\mu$. Let us assume that the particle moves in either a time-like or null geodesic (and thus without the influence of any force but gravity), which implies that
%
\begin{equation}
  p_\mu p^\mu = \left\{
  \begin{array}{lr}
    -m^2, & \text{if the particle is massive}  \\
    0,    & \text{if the particle is a photon}
  \end{array}
  \right. .
  \label{eq:arbitrary_penrose_p_norm}
\end{equation}
%
The 4-momentum can be decomposed as
%
\begin{equation}
  p^\mu = E(n^\mu + V^\mu)
  \label{arbitrary_penrose_p_decomp}
\end{equation}
%
in which $E$ represents the particle's energy as measured $\mathcal{O}_E$ (which means that $E = - n_\mu p^\mu$) and $V^\mu$ represents the 3-velocity of the particle, also as measured by $\mathcal{O}_E$. The 3-momentum $P^\mu$ of $\mathcal{P}$ as observed by $\mathcal{O}_E$ is thus
%
\begin{equation}
  P^\mu \equiv \tens{\gamma}{\mu}{\nu} p^\nu = E V^\mu
  \label{arbitrary_penrose_3_momentum}
\end{equation}
%
and the normalization of $p^\mu$, together with $n_\mu n^\mu = -1$ and Eq.~\eqref{arbitrary_penrose_p_decomp} imposes
%
\begin{equation}
  V_\mu V^\mu = V_i V^i \left\{
  \begin{array}{lr}
    = 1, & \text{if the particle is massive}  \\
    < 1, & \text{if the particle is a photon}
  \end{array}
  \right. .
  \label{eq:arbitrary_penrose_V_norm}
\end{equation}

Parametrizing the particle's position vector $X^i$ by the coordinate time (that is $x^i = X^i(t)$) the geodesic equation of $\mathcal{P}$ is decomposed in a set of 7 equations, namely
%
\begin{align}
  \der{X^i}{t} & = N V^i - \beta^i \label{eq:arbitrary_penrose_geodesic_eq_X}                                                                                                                                                  \\
  \der{V^i}{t} & = N V^j\left[ V^i \left( \partial_j \ln N - K_{jk} V^k \right) + 2 \tens{K}{i}{j} - {}^3\Gamma^{i}_{jk}V^k\right] - \gamma^{ij}\partial_j N - V^j\partial_j\beta^i \label{eq:arbitrary_penrose_geodesic_eq_V} \\
  \der{E}{t}   & = E (N K_{jk} V^j V^k - V^j \partial_j N) \label{eq:arbitrary_penrose_geodesic_eq_E}
\end{align}
%
that can be solved once initial positions, velocities and energy are supplied.

\subsection{Penrose process}