In this chapter, we will explore energy extraction from black hole binaries via the Penrose Process (PP) in black hole binaries.

The first incarnation of the PM arose as a consequence of the Kerr metric. The Kerr metric is the best know mathematical description of rotating black holes given by the theory of General Relativity~\cite{Visser:2007fj,Bambi:2011mj,Teukolsky:2014vca,berti}. Unlike static black holes, a Kerr black hole is characterized by the existence of a very peculiar region around its event horizon, known as the \emph{ergosphere} or \emph{ergoregion}. Particles that reach the ergosphere can still avoid the event horizon and, hence, are not doomed to end at the spacetime singularity. Nevertheless, any observer lying inside the ergosphere is unavoidably dragged along by the rotational motion of the black hole.

Particles inside the ergosphere may have negative energies (according to a static observer at infinity). Relying on this property, Penrose and Floyd devised a mechanism that allows one to extract energy from a rotating black hole~\cite{PENROSE1971}. The idea consists in sending a particle from infinity towards the black hole and assumes that, once inside the ergosphere, it decays into two other particles. If one of the fragments is counter-rotating with the black hole and has negative energy (which implies that the split happens inside the ergosphere), it will be captured by the black hole, meaning that the other fragment will escape to infinity. Due to the conservation of the four-momentum, the escaping fragment will have more energy and more angular momentum than the incident particle. Rotational energy and angular momentum are, thus, effectively extracted from the black hole.

Our main motivating factor for this investigation was that despite being a well-known process, the PP had not been studied in the context of black hole binaries. We have also found that many recent research endeavors attempt to establish a relation of the PP with observable astrophysical phenomena. The collisional version of the process, for instance, considers multiple particles that collide and scatter in the ergoregion, allowing  arbitrarily high center-of-mass energies. This process can potentially act as a mechanism to eliminate dark matter particles near a supermassive black hole~\cite{Schnittman:2018ccg}. The magnetic PP~\cite{Wagh1985,Tursunov:2019oiq}, on the other hand, considers charged particles and black holes surrounded by magnetic fields (originated, for instance, by plasma accretion disks around the black hole). The electromagnetic interaction allows for highly efficient energy extraction schemes, such as the one introduced in Ref.~\cite{Tursunov:2020juz} to model the emission of ultra-high energy cosmic rays from supermassive black holes. Furthermore, recent numerical simulations of plasma and jets around Kerr black holes indicate the important role that negative energy particles and the PP play on the total energy flux coming from the black hole's jets~\cite{Parfrey:2018dnc}.

A few comments on the models chosen for representing the binary systems are in order. Firstly, the nomenclature that we use to refer to a BBH is as follows:

\begin{definition}[Static/Dynamic BBH model]
    A \textbf{static} BBH model represents two black holes that \textbf{do not move} with respect to the observers that will participate in the PP. This is in contrast to a \textbf{dynamic} BBH model.
\end{definition}

\begin{definition}[Exact/Approximate BBH model]
    An \textbf{exact} BBH model is one that is an exact vacuum solution of Einstein's field equations. An \textbf{approximate} BBH model is one that is not an exact vacuum solution of said equations, and the deviation from vacuum is not considered to be an exotic type of matter but a measure of its ``non-exactness''.
\end{definition}

\begin{definition}[Analitic/Numeric BBH model]
    An \textbf{analytic} BBH model is one whose entire spacetime metric is analytically known at all points and times. In contrast, a \textbf{numeric} BBH model is obtained only at a certain coordinate time hypersurface (typically $t=0$) by solving the Arnowit-Deser-Misner (ADM) constraint equations for a vacuum configuration numerically. Note that even tough such numerical solutions contain errors, and thus do not strictly solve the vacuum field equations (giving rise to constraint violations), we do not consider these to be approximate, since it is possible (at least in theory but not in practice due to physical limitation of machines and computational resources) to obtain these solutions with arbitrarily small constraint violations given that they are convergent. For more information on solving the ADM constraint equations and obtaining initial data in Numerical Relativity, see for instance Chapter 3 of Ref~\cite{Alcubierre2012-xp}
\end{definition}

Secondly, any study that aims to model astrophysical binary systems must acknowledge the fact that there is no known exact and analytical solution of Einstein's equations describing such a system. Even if exact numerical or approximate analytical solutions are employed, the resulting spacetime metrics are non-stationary and thus limit the applicability of the standard concept of an ergosphere. Recognizing these difficulties, we have utilized BBH models that are static, exact and analytic.

Our first task was to investigate the possibility of energy extraction from the Majumdar-Papapetrou (MP) metric~\cite{MAJUMDAR1947,PAPAPETROU1947}, which is an exact solution of Einstein's equations that describes a static binary of extremally charged black holes. Despite its mathematical simplicity, the MP solution has recently been used as a surrogate model for black hole binaries in order to understand how single black hole phenomena transpose to a binary system. For instance, Ref.~\cite{ASSUMPCAO2018} has employed the MP metric to understand the connection between quasinormal modes and light rings in the context of black hole binaries. Refs.~\cite{Shipley:2016omi,Shipley:2019kfq}, on the other hand, have computed the shadows cast by an MP binary to better understand chaotic scattering in a binary black hole system. The resulting shadow shares many similarities and qualitative features with the shadows computed in Ref.~\cite{Bohn:2014xxa} using a numerically simulated binary black hole background. Ref.~\cite{BINI2019} has also applied the MP metric to analyze particle scattering around a black hole binary and asserted its effectiveness in approximating a head on collision in the limit of large separations and small approach speeds. We followed an analog approach in order to gain physical intuition and qualitative insights about energy extraction from black hole binaries by the Penrose mechanism. In particular, we extended the concept of a particle dependent \emph{generalized ergosphere}~\cite{RUFFINI1971}, which enables the extraction of electromagnetic energy from Reissner-N\"ordstrom (RN) black holes, to the MP solution and study how the energy extraction efficiency is affected by the presence of a companion black hole.

Taking into account the fact that, in astrophysical contexts, any excess of electric charge in a black hole tends to be quickly neutralized~\cite{gibbons1975}, we also considered rotating systems in our work. More specifically, in order to illustrate how the main results for the MP spacetime can be extrapolated to a binary system composed of Kerr black holes, we employ a static exact and analytic solution of Einstein's field equations, discovered independently by Cabrera-Munguia, Manko and Ruiz (hereby referred to as the CMMR metric)~\cite{cabrera_metric,manko_ruiz_metric, manko_ruiz_thermo}. The CMMR metric describes two generic Kerr black holes that do not coalesce thanks to the presence of a ``strut'' that holds them apart at a fixed distance. In particular, we sketched the ergosphere of the CMMR spacetime for a selected set of parameters and gave an example of a PP around a binary of rotating black holes.

The combined results presented in Sections \ref{ch:penrose_binaries:sec:mp_penrose} and \ref{ch:penrose_binaries:sec:cmmr_penrose} have been compiled in a paper and were published in Ref.~\cite{PhysRevD.104.124025}. As an extension to this work, we have begun the development of a framework that should allow one to observe the PP for an arbitrary spacetime and shall present these preliminary results in Sec. \ref{ch:penrose_binaries:sec:arbitrary_penrose}. We illustrate the viability of our proposition by utilizing a non-exact solution to Einstein's field equations obtained by superimposing two Kerr metric solutions in Kerr-Schild coordinates (thus called the SKS - Superimposed Kerr-Schild Kerr solution), that models an orbiting binary of rotating black holes close to merger.