In this chapter, we will explore energy extraction from black hole binaries via the Penrose Process (PP) in black hole binaries.

Our first task was to investigate the possibility of energy extraction from the Majumdar-Papapetrou (MP) metric~\cite{MAJUMDAR1947,PAPAPETROU1947}, which is an exact solution of Einstein's equations that describes a static binary of extremally charged black holes. Despite its mathematical simplicity, the MP solution has recently been used as a surrogate model for black hole binaries in order to understand how single black hole phenomena transpose to a binary system. For instance, Ref.~\cite{ASSUMPCAO2018} has employed the MP metric to understand the connection between quasinormal modes and light rings in the context of black hole binaries. Refs.~\cite{Shipley:2016omi,Shipley:2019kfq}, on the other hand, have computed the shadows cast by an MP binary to better understand chaotic scattering in a binary black hole system. The resulting shadow shares many similarities and qualitative features with the shadows computed in Ref.~\cite{Bohn:2014xxa} using a numerically simulated binary black hole background. Ref.~\cite{BINI2019} has also applied the MP metric to analyze particle scattering around a black hole binary and asserted its effectiveness in approximating a head on collision in the limit of large separations and small approach speeds. We followed an analog approach in order to gain physical intuition and qualitative insights about energy extraction from black hole binaries by the Penrose mechanism. In particular, we extended the concept of a particle dependent \emph{generalized ergosphere}~\cite{RUFFINI1971}, which enables the extraction of electromagnetic energy from Reissner-N\"ordstrom (RN) black holes, to the MP solution and study how the energy extraction efficiency is affected by the presence of a companion black hole.

Taking into account the fact that, in astrophysical contexts, any excess of electric charge in a black hole tends to be quickly neutralized~\cite{gibbons1975}, we also considered rotating systems in our work. More specifically, in order to illustrate how the main results for the MP spacetime can be extrapolated to a binary system composed of Kerr black holes, we employ a static exact and analytic solution of Einstein's field equations, discovered independently by Cabrera-Munguia, Manko and Ruiz (hereby referred to as the CMMR metric)~\cite{cabrera_metric,manko_ruiz_metric, manko_ruiz_thermo}. The CMMR metric describes two generic Kerr black holes that do not coalesce thanks to the presence of a ``strut'' that holds them apart at a fixed distance. In particular, we sketched the ergosphere of the CMMR spacetime for a selected set of parameters and gave an example of a PP around a binary of rotating black holes.

The combined results presented in Sections \ref{ch:penrose_binaries:sec:mp_penrose} and \ref{ch:penrose_binaries:sec:cmmr_penrose} have been compiled in a paper and were published in Ref.~\cite{PhysRevD.104.124025}. As an extension to this work, we have begun the development of a framework that should allow one to observe the PP for an arbitrary spacetime and shall present these preliminary results in Sec. \ref{ch:penrose_binaries:sec:arbitrary_penrose}. We illustrate the viability of our proposition by utilizing a non-exact solution to Einstein's field equations obtained by superimposing two Kerr metric solutions in Kerr-Schild coordinates (thus called the SKS - Superimposed Kerr-Schild Kerr solution), that models an orbiting binary of rotating black holes close to merger.