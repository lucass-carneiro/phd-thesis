In this section, we shall briefly review energy extraction of rotating and charged black holes via the PP to familiarize the reader with the underlying physical concepts, methods and techniques involved in the mechanism. This is important since these tools will be necessary later on while we present novel results. Our review will follow closely Ref.~\cite{carroll}. We start by reminding that the Kerr metric, in Boyer-Lindquist coordinates $(t, r, \theta, \phi)$ is given by

\begin{multline}
  \ud s^2 = -\left(1 - \frac{2 M r}{\rho^2}\right)\ud t^2 - \frac{2 M a r \sin^2\theta}{\rho^2}\left( \ud t \ud \phi + \ud \phi \ud t \right)\\
  + \frac{\rho^2}{\Delta} \ud r^2 + \rho^2 \ud\theta^2 + \frac{\sin^2\theta}{\rho^2}\left[ (r^2 + a^2)^2 - a^2\Delta\sin^2\theta \right]\ud\phi^2,
  \label{eq:kerr_penrose_review_kerr_metric}
\end{multline}
%
where
%
\begin{equation}
  \Delta(r) = r^2 - 2Mr + a^2
  \label{eq:kerr_penrose_review_kerr_delta}
\end{equation}
%
and
%
\begin{equation}
  \rho^2(r, \theta) = r^2 + a^2\cos^2\theta.
  \label{eq:kerr_penrose_review_kerr_rho}
\end{equation}

The constants $M$, $J$ and $a = J/M$ represent, respectively the black hole's mass, angular momentum and spin parameter. The metric possesses two event horizons, located at
%
\begin{equation}
  r_{H\pm} = M \pm \sqrt{M^2 - a^2}
  \label{eq:kerr_penrose_review_kerr_horizons}
\end{equation}
%
and since its components are independent of both the coordinate time $t$ and the axial angular variable $\phi$, there are global Killing vector fields $K = \partial_t$ and $R = \partial_\phi$ that generate these symmetries. Since the metric is stationary, the region where the time-like global Killing vector field changes its sign and thus static observers become prohibited does not coincide with the event horizons. In fact, one can easily see that
%
\begin{equation}
  K^\mu K_\mu = -\frac{1}{\rho^2}\left({\Delta - a^2 \sin^2\theta}\right) = -\frac{a^2 + r(r-2M)-a^2\sin^2\theta}{(r^2+a^2\cos^2\theta)^2}.
  \label{eq:kerr_penrose_review_kerr_killing_horizon_equation}
\end{equation}
%
and thus $K^\mu K_\mu = 0$ implies that the killing horizons are located at
%
\begin{equation}
  r_{K\pm} = M \pm \sqrt{M^2 - a^2\cos^2\theta}.
  \label{eq:kerr_penrose_review_kerr_killing_horizon_solution}
\end{equation}
%
which means that $r_{K+}$ is outside $r_{H+}$, coinciding with it only at the poles ($\theta=0$ or $\theta=\pi$). Notice also that using $r_{H+}$ in Eq.~\eqref{eq:kerr_penrose_review_kerr_killing_horizon_equation} yields $\Delta=0$ and thus
%
\begin{equation}
  K^\mu K_\mu = \frac{a^2}{\rho^2}\sin^2\theta \geq 0.
  \label{eq:kerr_penrose_review_kerr_killing_horizon_solution_2}
\end{equation}
%
The region that lies in-between $r_{K+}$ and $r_{H+}$ is called the \emph{ergosphere} or \emph{ergoregion}. The fact that this region is outside the event horizons and that $K^\mu K_\mu>0$ inside it is paramount to the Penrose mechanism, as we shall see further on. A mathematically accurate representation of important structures in the Kerr spacetime plotted in Kerr-Schild Cartesian coordinates for $M=1$ and $a=0.9$ can be found in Fig. \ref{fig:kerr_penrose_review_kerr_surfaces}. Starting from the exterior and moving towards the center, one first encounters the ergosphere, shaded in light green and bounded by $r_{K+}$. Following that, we have the outer event horizon, shaded in light blue and bounded by $r_{H+}$, and the inner event horizon, shaded in dark blue and bounded by $r_{H-}$. Finally, we have the inner ergosphere in dark green and bounded by $r_{K-}$ followed by the ring singularity at $r=0$. This figure is based on the interactive visualization generated by Ref.~\cite{KerrSurfaceViz}.

\begin{figure}[!ht]
  \centering
  \includesvg[width=\linewidth]{img/penrose_binaries/kerr_surfaces.svg}
  \caption{Mathematically accurate representation of a Kerr black hole of mass $M=1$ and spin $a=0.9$ plotted in Cartesian Kerr-Schild coordinates. The light green region represents the ergosphere of the black hole. This figure is based on the interactive visualization generated by Ref.~\cite{KerrSurfaceViz}.}
  \label{fig:kerr_penrose_review_kerr_surfaces}
\end{figure}

Let us now consider a particle of 4-momentum $p^\mu$ moving along a time-like geodesic in the Kerr spacetime parametrized by its proper time $\tau$. The energy of the particle along its trajectory, as measured by a static observer infinitely far away from the black hole, is given by
%
\begin{equation}
  E = -K_\mu p^\mu.
  \label{eq:kerr_penrose_review_kerr_enrgy_per_unit_mass_full}
\end{equation}
%
Outside the ergosphere, $K^\mu$ is time-like and $K_\mu p^\mu<0$. Since we would like the energy to be positive infinitely far away from the BH, we must introduce a leading minus sign in Eq.~\eqref{eq:kerr_penrose_review_kerr_enrgy_per_unit_mass_full}. On the other hand, inside the ergosphere $K^\mu$ is space-like and $K_\mu p^\mu >0$ which implies that in this region $E < 0$. It is important to remark that, despite the energy being negative relatively to a static observer at infinity, it still remains positive according to a local observer. Furthermore, it can be shown that these negative energy orbits must be confined within the ergosphere and must always cross the outer horizon~\cite{Grib:2013hxa,Contopoulos1984}, so there is no risk of such negative energy particles ``leaking out'' to infinity.

Let us now consider a scenario where a particle marked as $(0)$ moves through the Kerr spacetime from infinity with 4-momentum $p^{(0)\mu}$. Suppose this particle disintegrates within the ergosphere at a specific point called the break-up point $b$. This break-up results in the formation of two other particles, one marked as $(1)$, with 4-momentum $p^{(1)\mu}$, that gets absorbed by the black hole, and the other labeled as $(2)$, with 4-momentum $p^{(2)\mu}$, which escapes the gravitational pull of the system and returns to infinity. The law of conservation of 4-momentum applied at point $b$ implies that
%
\begin{equation}
  p^{(0)\mu} = p^{(1)\mu} + p^{(2)\mu}.
  \label{eq:kerr_penrose_review_kerr_4_momentum_conservation}
\end{equation}
%
Contracting Eq.~\eqref{eq:kerr_penrose_review_kerr_4_momentum_conservation} with $K^\mu$ we get
\begin{equation}
  E^{(0)} = E^{(1)} + E^{(2)}.
  \label{eq:kerr_penrose_review_kerr_energy_conservation}
\end{equation}
%
If we engineer the trajectory of particle $(1)$ such that $E^{(1)} < 0$, we get
\begin{equation}
  E^{(2)} = E^{(0)} + E^{(1)} > E^{(0)},
  \label{eq:kerr_penrose_review_kerr_energy_increase}
\end{equation}
%
which means that the energy of the returning particle is greater than the energy of the original particle. This mechanism is precisely the one proposed by Penrose and Floyd, and is therefore known as the \emph{Penrose Process} (PP) or \emph{Penrose Mechanism}. This break-up process is represented in Fig.~\ref{fig:kerr_penrose_review_kerr_breakup}.

\begin{figure}[!ht]
  \centering
  \includesvg[width=\linewidth]{img/penrose_binaries/kerr_breakup.svg}
  \caption{Schematic representation of a PP. A particle of energy $E^{(0)}$ (red) comes in from infinity and decays at point $b$ inside the ergosphere in a negative energy $E^{(1)}$ trajectory (black) and a positive energy $E^{(2)}>E^{(0)}$ trajectory (blue) that returns to infinity. The black hole's rotational axis is pointing outside the page, towards the reader.}
  \label{fig:kerr_penrose_review_kerr_breakup}
\end{figure}

Where does the excess energy of particle $(2)$ comes from? To answer that, let us analyze particle $(1)$ as it crosses the outer horizon. In the Kerr metric, the event horizons are Killing horizons of the Killing vector formed by a linear combination of the two spacetime symmetries, namely
%
\begin{equation}
  \chi^\mu = K^\mu + \Omega R^\mu
  \label{eq:kerr_penrose_review_kerr_killing_horizon}
\end{equation}
%
where $\Omega$ is the horizon's angular velocity. Considering that particle $(1)$ crosses the event horizon while remaining time-like and $\chi^\mu$ is null at the outer horizon, we must have that
%
\begin{equation}
  p^{(1)\mu}\chi_\mu = -E^{(1)} + \Omega L^{(1)} < 0
  \label{eq:kerr_penrose_review_kerr_energy_momentum_inequality}
\end{equation}
%
where $L^{(1)}$ represents the particle's angular momentum. In order to satisfy Eq.~\eqref{eq:kerr_penrose_review_kerr_energy_momentum_inequality} given that $E^{(1)} < 0$ and $\Omega > 0$ we must have that
%
\begin{equation}
  L^{(1)} < \frac{E^{(1)}}{\Omega},
  \label{eq:kerr_penrose_review_kerr_energy_momentum_inequality_L}
\end{equation}
%
which implies that $L^{(1)} < 0$ and thus the negative energy particle must be rotating in the opposite direction of the black hole. This means that when the negative energy particle gets absorbed, the BH loses a small amount of its own angular momentum, transferring it to the particle that is to be ejected back to infinity.

At this stage, one may contemplate whether it is feasible to extract all the matter from the black hole using repeated Penrose Processes (PPs). Christodoulou, in Ref.~\cite{CHRISTODOULOU1970}, has identified a restriction on energy extraction. This limitation is expressed as an irreducible mass, which represents the minimum mass that a black hole can have, and beyond which no further energy can be extracted. To see that, let us consider that when particle $(1)$ interacts with the BH, energy is conserved and thus the loss in mass and angular momentum of the black hole corresponds respectively to $E^{(1)}$ and $L^{(1)}$, that is,
%
\begin{align}
  \ud M & = E^{(1)} \label{eq:kerr_penrose_review_kerr_mass_change}, \\
  \ud J & = L^{(1)} \label{eq:kerr_penrose_review_kerr_L_change}.
\end{align}
%
By virtue of Eq.~\eqref{eq:kerr_penrose_review_kerr_energy_momentum_inequality_L}, we get that
%
\begin{equation}
  \ud J < \frac{\ud M}{\Omega}.
  \label{eq:kerr_penrose_review_kerr_energy_momentum_inequality_L_diff}
\end{equation}
%
The maximum energy extraction occurs when Eq.~\eqref{eq:kerr_penrose_review_kerr_energy_momentum_inequality_L_diff} is saturated and $\ud J = \frac{\ud M}{\Omega}$. Given that
%
\begin{equation}
  \Omega = \frac{a}{r_{H+}^2 + a^2}
  \label{eq:kerr_penrose_review_kerr_angular_momentum_def}
\end{equation}
%
we can solve
%
\begin{equation}
  \frac{\ud M}{\ud J} = \frac{a}{r_{H+}^2 + a^2}
  \label{eq:kerr_penrose_review_kerr_mass_ODE}
\end{equation}
%
for $M(J)$ with $M(0) = M_0$ to find
%
\begin{equation}
  M_0^2 = \frac{1}{2}\left( M^2 + \sqrt{M^4 - J^2} \right),
  \label{eq:kerr_penrose_review_kerr_irreducible_mass}
\end{equation}
%
which is known as the \emph{irreducible mass}. To see why that is the case, one can take the differential of Eq.~\eqref{eq:kerr_penrose_review_kerr_irreducible_mass} to find
%
\begin{equation}
  \ud M_0 = \frac{a}{4 M_0 \sqrt{M^2 - a^2}}\left( \Omega^{-1} \ud M - \ud J \right).
  \label{eq:kerr_penrose_review_kerr_irreducible_mass_diff}
\end{equation}
%
Thanks to Eq.~\eqref{eq:kerr_penrose_review_kerr_energy_momentum_inequality_L_diff}, we can infer from Eq.~\eqref{eq:kerr_penrose_review_kerr_irreducible_mass_diff} that $\ud M_0 > 0$ always, thus $M_0$ cannot be reduced (hence it's name) and gives the lower energy bound that the black hole can achieve. The formula for the maximum energy that can be extracted is thus
%
\begin{equation}
  M - M_0 = M - \frac{1}{\sqrt{2}}\left( M^2 + \sqrt{M^4 - J^2} \right)^{1/2}.
  \label{eq:kerr_penrose_review_kerr_max_energy_formula}
\end{equation}
%
By considering an extreme BH (with $a=1$), Eq.~\eqref{eq:kerr_penrose_review_kerr_max_energy_formula} states that approximately $29\%$ of the BH's energy can be extracted via the PP. One can get to the same conclusion by starting of with Hawking's area theorem, which states that the event horizon area cannot decrease (See Ref.~\cite{carroll} page 270).