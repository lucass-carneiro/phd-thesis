In this thesis we have investigated classical phenomena in multiple models of black hole binaries. Specifically, we have analyzed energy extraction, quasinormal modes, and wave scattering around black hole binaries, utilizing numerical, analytical, and semi-analytical techniques. This study has provided the first-ever demonstration of how the phenomena under investigation are altered by the presence of a binary, as compared to their single black hole counterparts.

With regard to the Penrose process, we have demonstrated in several comprehensive ways how energy extraction becomes increasingly effective in the presence of charged and rotating binaries. Additionally, we have devised an innovative technique that extends the analysis of energy extraction in a single black hole to arbitrary spacetimes, including numeric ones. We have provided a proof-of-concept example featuring a superimposed Kerr binary, which confirms that the energy extraction efficiency is enhanced by the presence of a secondary object.

Furthermore, a numerical package with broad applicability for computing quasinormal frequencies of black hole spacetimes has been developed. This package employs the recently developed Asymptotic Iteration Method and can easily handle multiple systems, that is, those whose perturbation evolutions can be written as second order ODEs. We have utilized this package to revisit the perturbation problem in the Schwarzschild spacetime, comparing the frequencies computed via the new code with frequencies obtained through traditional pseudospectral methods and values presented in the literature via the WKB approximation or Leaver's continued fraction method. The findings indicate that the results obtained using the AIM code are in excellent agreement with available results, thereby enhancing confidence in the method's accuracy and potential to address novel problems. Finally, we have computed new quasinormal frequencies for spin $5/2$ perturbations for the first time, highlighting that these perturbations also possess purely imaginary frequencies.

Lastly, we have elaborated on the creation of a new \texttt{EinsteinToolkit Thorn}, called \texttt{FieldPerturbations}, for the numerical evolution of a scalar field superimposed on a dynamically evolving metric. We have presented preliminary findings concerning the scattering of a massless field over the GW150914 binary collision. Specifically, our results have demonstrated that the field profile displays a damped oscillatory phase following a highly nonlinear scattering, which bears similarities to the behavior observed when perturbing a single black hole. However, distinct differences prohibit direct fitting of a damped sine or cosine to the signal. Given that the run is in its initial stage, and the code remains subject to rigorous testing and development, we must conclude that it is currently impossible to confirm conclusively whether this effect is exclusively attributable to the nonlinear nature of the interaction between the black holes in the binary or to numerical artifacts due to various sources.

We contend that our study has yielded valuable insights and ideas for extending established phenomena to binary systems. We maintain that it is crucial to develop tools and methods capable of extending these analyses to binary black hole spacetime models that lack analytic solutions, given that numerical methods are necessary to capture the full effect of nonlinear interaction in General Relativity. Moreover, we believe that our work represents a crucial step towards this goal and serves to inspire further research in these phenomena.

We propose three potential directions for future research. Firstly, we aim to incorporate our proposed approach for studying the Penrose process in dynamic spacetimes into fully dynamical simulations. We suggest utilizing the \texttt{EinsteinToolkit} infrastructure to compute orbits as a background metric is evolved, or even using tables of a numerically evolved metric as input for the code. Secondly, we contend that it may be possible to separate the perturbation partial differential equations (PDEs) that arise from analytical models of binary spacetimes by employing series approximations to the solutions. This approach would enable the derivation of a system of coupled ordinary differential equations (ODEs) that could, in principle, be solved via the AIM or other methods to yield arbitrarily accurate approximations of quasinormal frequencies in these contexts. Finally, we suggest further exploring the numerical simulation of the scalar field on top of the GW150914 binary, searching for potential energy growth via superradiant scattering, analyzing the profile of the field to identify quasinormal frequencies, and investigating other spin perturbations. Additionally, we propose considering the backreaction of the spacetime metric to enhance the realism of the simulations.