The spacetime metric for a spherically symmetric Schwarzschild black hole, is given by \cite{Schwarzschild:1916uq}
%
\begin{equation}\label{eq:Metric}
  ds^2=-f(r)\,dt^2+\frac{1}{f(r)}dr^2+r^2d\theta^2+
  r^2\sin^2{\theta}\,d\varphi^2,
\end{equation}
%
where the horizon function is given by
%
\begin{equation}\label{EqHoriFuncHayward}
  f(r)=1-\frac{2M}{r},
\end{equation}
%
where $M$ is the mass of the black hole, and $r$ is the radial coordinate which, in principle, belongs to the interval $r\in [0,\, \infty)$.

The coordinates employed in the metric described by Eq.~\eqref{eq:Metric} are commonly known as Schwarzschild coordinates. This metric features an event horizon located at $r=2M$ and a physical singularity at $r=0$. In the asymptotic region, namely as $r$ approaches infinity, the metric reduces to a flat metric. For the purposes of examining quasinormal modes in this black hole spacetime, the relevant region of the spacetime is the range of the radial coordinate $r$ between $2M$ and infinity.

Our analysis of linear perturbations in the Schwarzschild black hole spacetime described by \eqref{eq:Metric} follows the standard procedure. After selecting a specific perturbation field, we reduce the corresponding partial differential equations to a unique Schrödinger-like ordinary differential equation using a series of transformations. In this approach, the perturbation functions were decomposed into Fourier modes of the form $e^{i\omega t}=e^{i(\omega_{Re}-i \omega_{Im})t}=e^{\omega_{Im} t}\cos{\left(\omega_{Re} t\right)}$. This eliminated the time derivatives from the differential equations, while the angular dependence was handled through expansion in spherical harmonics.

It is important to be reminded at this point that ordinary \acp{QNM} are characterized by frequencies with both non-zero real and imaginary components and represent oscillatory solutions that are exponentially suppressed by the imaginary component of the frequency. In contrast, modes with purely imaginary frequencies, where the real part of the frequency is zero, correspond to purely damping solutions since the respective perturbation functions decay as $ e^{i\omega t}=e^{\omega_{Im} t}$.

\subsection{Spin $0$, $1$ and $2$ perturbations}

The computation of perturbations of integer spin, including scalar, vector, and gravitational perturbations in the Schwarzschild black hole spacetime, is a well understood and solved problem with a significant body of literature dedicated to it. This literature is of great interest for our purposes, which involves comparisons and calibrations. In fact, it has been shown that the equations of motion can be expressed in a concise form known as Schrödinger-like differential equations (see, for instance, \cite{review3}). Thus, for massless scalar ($s=0$), electromagnetic ($s=1$), and vector-type gravitational perturbations ($s=2$), the Schrödinger-like equations take the following form:
%
\begin{equation}\label{Eq:IntegerSpin}
  \frac{d^2 \psi_{{s}}(r)}{d r_*^2}
  +\left[\omega^2-V_{{s}}(r)\right]\psi_{{s}}(r)=0,
\end{equation}
%
where the potential function $V_{s}(r)$ is given by
%
\begin{equation}\label{Eq:IntegerSpinPot}
  V_{s}(r)=f(r)\left[\frac{\ell\left(\ell+1\right)}{r^2}
    +\left(1-s^2\right)\frac{2M}{r^3}\right],
\end{equation}
%
and the tortoise coordinate $r_*$ is defined in terms of the areal coordinate $r$ by $dr_*=dr/f(r)$. The problem of calculating quasinormal frequencies is therefore reduced to an eigenvalue problem, which can be solved by either expanding the function $\psi$ in a base composed of special functions or solving the second-order differential equation directly once appropriate asymptotic boundary conditions have been specified.

It is important to note that the potential function in Eq.~\eqref{Eq:IntegerSpinPot} is zero at the horizon because $f(r_h)=0$. Thus, the Schr\"odinger-like equation reduces to a single harmonic oscillator problem, whose solutions are given by
%
\begin{equation}
  \psi_{s}(r)=c_1\, e^{-i\omega r_*}+c_2\, e^{i\omega r_*}, \quad r\to r_h.
  \label{eq:sol1}
\end{equation}
%
The first term in Eq.~\eqref{eq:sol1} is interpreted as an ingoing wave, i.e., a wave that travels inward and eventually falls into the black hole event horizon. The second term is interpreted as an outgoing wave, i.e., a wave that travels outward with respect to the black hole and can escape to spatial infinity. Considering that perturbation theory is implemented using classical assumptions, nothing is expected to come out from the black hole interior, thus, in the following analysis we impose ingoing solutions as boundary conditions at the horizon and discarding outgoing solutions altogether. This is accomplished by setting  $c_2=0$.

At spatial infinity, however, one has $f(r)\to 1$ and the effective potential of Eq.~\eqref{Eq:IntegerSpinPot} also vanishes. Thus, in such a limit the general solutions to the wave equation \eqref{Eq:IntegerSpin} have the same form as the function given in Eq.~\eqref{eq:sol1}, i.e.,
%
\begin{equation}
  \psi_{s}(r_*)=c_3\, e^{-i\omega r_*}+c_4\, e^{i\omega r_*}, \quad r\to \infty.
\end{equation}
%
The first solution can be interpreted as waves that originate sufficiently far away from the black hole and can be avoided by setting $c_3=0$. Conversely, the second solution can be interpreted as waves that leave the computational domain, serving as the boundary condition at spatial infinity. It is important to note that neither the angular momentum $\ell$ nor the perturbing field's spin have an explicit influence on the boundary conditions.

It is interesting to note that close to the event horizon, the tortoise coordinate becomes
%
\begin{equation}
  r_*=\int \frac{dr}{f'(r_h)(r-r_h)}\approx \frac{\ln{(r-r_h)}}{ f'(r_h)},\quad r\to r_h
\end{equation}
where $r_h=2M$.
Thus, when written in terms of the radial coordinate, the boundary condition at the horizon becomes ($f'(r_h)=1/r_h$)
%
\begin{equation}
  \psi_s(r)\sim e^{-i\omega \frac{\ln{(r-r_h)}}{f'(r_h)}}\sim \left(r-r_h\right)^{-i\frac{\omega}{f'(r_h)}}.
\end{equation}
%
In turn, the tortoise coordinate at spatial infinity becomes
%
\begin{equation}
  r_*=\int \frac{dr}{f(r)}\approx r+r_h\,\ln{r},\quad r\to \infty
\end{equation}
%
while the asymptotic solution at spatial infinity becomes
%
\begin{equation}
  \psi_s(r)\sim e^{i\omega(r+ r_h \ln{r})}\sim r^{i\, r_h\omega}e^{i\omega r}.
\end{equation}

We shall now work in terms of new coordinates defined by $u\equiv2M/r$. This is equivalent to choosing $u=1/r$ and then normalizing the mass $M$ to $2M=1$. The relation between the tortoise coordinate $r_*$ and the new coordinate $u$  becomes $du/dr_*=-u^2f(u)$. We shall also constrain our analysis to the outer region of the black hole, such that $r_h\leq r<\infty$. Hence, in terms of the new coordinate, this region is bounded to the interval $u\in [0,\, 1]$, and the potential becomes
%
\begin{equation}
  V_{s}(u)=f(u)\,u^2\left[\ell\left(1+\ell\right)+\left(1-s^2\right)\,u\right],
\end{equation}
%
where $f(u) = 1- \,u$.

In numerical codes, it is frequently necessary to represent background quantities using horizon-penetrating Eddington-Finkelstein coordinates, as elaborated in Ref.~\cite{qnmspectral}. This is true despite the fact that we exclusively address perturbations outside the event horizon, as the coordinate singularity at the horizons can impede convergence and complicate the numerical solution of perturbation ODEs. We have developed a shortcut for writing the equations directly from the Schrödinger-like equation by implementing certain transformations. These transformations result in a differential equation that is equivalent to the one obtained from the Eddington-Finkelstein metric coordinates. In order to express the perturbation equations in terms of Eddington-Finkelstein coordinates, we utilize the transformation $\psi_{s}\to \Phi_s$, which is given by
%
\begin{equation}\label{Eq:TransN1}
  \psi_{s}= \frac{\Phi_{s}(u)}{u}e^{-i\omega\,r_*(u)}.
\end{equation}
%
Thus, Eq.~\eqref{Eq:IntegerSpin} becomes
%
%\begin{widetext}
\begin{equation}\label{Eq:IntegerSpin2}
  \begin{split}
    &\left[s^2u^2-\ell\left(\ell+1\right)u-2\,i\,\omega\right]\Phi_{s}(u)\\
    &-u\left(u^2-2\,i\,\omega\right)\Phi_{s}'(u)+\left(1-u\right)\,u^3\,\Phi_{s}''(u)=0,\end{split}
\end{equation}
%\end{widetext}
%
where we have set $M=1/2$ so that $r_h=1$. The asymptotic solutions close to the horizon may be calculated using the ansatz $\Phi_{s}(u)=(1-u)^{\alpha}$. By substituting this ansatz into Eq.~\eqref{Eq:IntegerSpin2} we get two solutions,
%
\begin{equation}\label{Eq:AsympHorion}
  \alpha=0,\qquad\qquad \alpha=2\,i\,\omega.
\end{equation}
%
The solution for $\alpha =0$ is interpreted physically as the ingoing waves at the horizon, while the other is interpreted as a wave coming out from the black hole interior, and thus, is neglected in the following analysis. In the same way, we consider the ansatz $\Phi_{s}=u^{\beta}$ to get the asymptotic solution close to the spatial infinity. Plugging this ansatz in Eq.~\eqref{Eq:IntegerSpin2} we get
%
\begin{equation}\label{Eq:AsymInfinite}
  \Phi_{s}(u)=c_5\,e^{2\,i\,\omega/u}u^{-2\,i\,\omega}+c_6\,u.
\end{equation}
%
We are interested in the divergent solution thus setting $c_6=0$. We then implement the final transformation, which takes into consideration the boundary conditions,
%
\begin{equation}\label{Eq:FinalTrans}
  \Phi_{s}(u)=e^{2\,i\,\omega/u}u^{-2\,i\,\omega}\phi_{s}(u),
\end{equation}
%
where $\phi_{s}(u)$ is a regular function in the interval $u\in[0,\,1]$ by definition. Finally, the equation of motion describing spin $0$, $1$, and $2$ perturbations is given by
%
\begin{equation}\label{Eq:IntegerSpin3}
  \begin{split}
    &\!\!\Big[\ell\left(\ell+1\right) u -s^2u^2-4\,i\lambda-16\,u\left(1+u\right)\lambda^2\Big]\phi_{s}(u)+\\
    & \!\!\Big[u^3+4i\,u\left(1-2u^2\right)\lambda\Big]\phi_{s}'(u)-(1-u)u^3\phi_{s}''(u)=0,
  \end{split}
\end{equation}
%
where we have used $\lambda=\omega M=\omega/2$. The final differential equation is then a quadratic eigenvalue problem in $\lambda$. It is also worth mentioning that in the limit of zero spin $s\to 0$, Eq.~\eqref{Eq:IntegerSpin3} reduces to Eq.~(4.8) of Ref.~\cite{qnmspectral}. These results just prove that the alternative way for getting the equations for the integer spin perturbations presented here is consistent with other approaches in the literature.

\subsection{Spin $1/2$ perturbations}

The differential equations for half-integer spin perturbations are quite distinct from Eq.~\eqref{Eq:IntegerSpin3}. See Ref.~\cite{Chandrasekhar1998-le} and references therein for the mathematical foundations of black hole perturbation theory for higher spin fields. The equation for the spin $1/2$ Dirac field as a perturbation on the Schwarzschild background was derived in Ref.~\cite{Cho:2003qe} by using the Newman-Penrose formalism. The analysis was generalized for arbitrary half-integer spin in Ref.~\cite{Shu:2005fw}. The resulting equation of motion for the perturbations may be written in the Schr\"odinger-like form of Eq.~\eqref{Eq:IntegerSpin}, where the potential for the massless spin 1/2 field is given by
%
\begin{equation}\label{eq:pot-s12}
  V_{\scriptscriptstyle{1/2}}= \frac{\left(1+\ell\right)\sqrt{f(r)}}{r^{2}} \left[\left(1+\ell\right)\sqrt{f(r)}+\frac{3M}{r}-1\right].
\end{equation}

It is worth mentioning that we have found a typo in the definition of $\Delta$ in \cite{Cho:2003qe}, which must read $\Delta=r(r-2M)$.

We then implement the same transformations applied in the integer spin cases. First, we change the radial coordinate to $u=2M/r$, which is defined in the interval $u\in[0,1]$. Then by setting $2M=1$ the effective potential in Eq.~\eqref{eq:pot-s12} becomes
%
\begin{equation}
  \!\!V_{\scriptscriptstyle{1/2}}=\left(1+\ell\right)u^2\sqrt{f(u)} \left[\left(1+\ell\right)\sqrt{f(u)}+\frac{3u}{2} -1\right].
\end{equation}

It is interesting to point out that the asymptotic solutions do not depend on the spin of the field, and therefore the asymptotic solutions for this problem are the same as those obtained in Eqs.~\eqref{Eq:AsympHorion} and \eqref{Eq:AsymInfinite}. Similar transformations as those given in Eqs.~\eqref{Eq:TransN1} and~\eqref{Eq:FinalTrans} can then also be applied. Thus, the differential equation to be solved is given by
%
\begin{equation}\label{eq:onda-s12}
  \begin{split}
    &R(u)\phi_{\scriptscriptstyle{1/2}}(u) + Q(u)\phi_{\scriptscriptstyle{1/2}}'(u) + P(u) \,\phi_{\scriptscriptstyle{1/2}}''(u)=0,
  \end{split}
\end{equation}
%
in which the coefficients $R(u)$, $Q(u)$, and $P(u)$ are given by
%
\begin{eqnarray}
  R(u)=\,&& u^3+u(1+\ell)\left(1+\ell-\sqrt{1-u}\, \right) \nonumber \\
  && +\frac{u^2}{2}\left[(1+\ell)\left(3\sqrt{1-u}-4\right)-2\ell^2\right]\nonumber \\
  &&-4\,i\,(1-u)\lambda-16u(1-u^2)\lambda^2,\\
  Q(u) = && u^3(1-u)+4\,i\,u\, \lambda\left(1-u-2u^2+2u^3\right), \\
  P(u)  =&& -u^3(1-u)^2,
\end{eqnarray}
%
respectively, and with $\lambda$ standing for $\lambda=M\omega=\omega/2$.

The square root terms present in Eq.~\eqref{eq:onda-s12} may difficult the convergence of the numerical methods. To avoid them, we perform an additional change of variables given by $\chi^2=1-u$. Note that the new coordinate also belongs to the interval $\chi \in [0,\,1]$. The differential equation of Eq.~\eqref{eq:onda-s12} thus becomes
%
\begin{equation}\label{eq:onda-s12a}
  \begin{split}
    &R(\chi)\phi_{\scriptscriptstyle{1/2}}(\chi) + Q(\chi)\phi_{\scriptscriptstyle{1/2}}'(\chi) + P(\chi) \,\phi_{\scriptscriptstyle{1/2}}''(\chi)=0,
  \end{split}
\end{equation}
%
in which the coefficients $R(\chi)$, $Q(\chi)$, and $P(\chi)$ are given by
%
\begin{eqnarray}
  R(\chi)=
  \,&&2(1-\chi^2)\left[\left(\ell+1\right) \left(1+ 2\ell\,\chi-3\chi^2\right) + 2\ell\, \chi +2\chi^3\right] \nonumber \\
  && - 8\,i\,\chi \,\lambda-32\,\chi\left(2-3\chi^2+\chi^4\right)\lambda^2,\\
  Q(\chi) = && (\chi^2-1\big)\left[\big(1-\chi^2\big)^2-8\,i\big(1-4\chi^2+2\chi^4\big)\lambda\right],\\
  P(\chi)  =&& -\chi\big(1-\chi^2\big)^3.
\end{eqnarray}

\subsection{Spin $3/2$ perturbations}

As is the case for spin $1/2$ perturbations, the perturbation equation for spin the $3/2$ field is quite different from that of integer spin fields. The relevant equation is obtained by making use of Ref.~\cite{Shu:2005fw}, specifically Eq.~(37) of that reference, and by setting $s=3/2$ on such an equation we then get the effective potential of the Schr\"odinger-like equation Eq.~\eqref{Eq:IntegerSpin},
%
\begin{equation} \begin{split}
    V_{\scriptscriptstyle{3/2}} &=\frac{(1+\ell)(2+\ell)(3+\ell)\sqrt{f(r)}}{\left[2M+r(1+\ell)(3+\ell)\right]^2} \bigg(\frac{2M^2}{r^2}\\ &+(1+\ell)(3+\ell)\left[(2+\ell)\sqrt{f(r)}+\frac{3M}{r}-1\right]\bigg).
  \end{split}
\end{equation}
%
As before, we change the radial coordinate to $u=1/r$ and rescale the mass as $2M=1$. Thus, the potential becomes
%
\begin{equation}\begin{split}
    V_{\scriptscriptstyle{3/2}}&=\frac{u^2(1+\ell)(2+\ell)(3+\ell)\sqrt{1-u}}{2\big[u+(1+\ell)(3+\ell)\big]^2} \bigg(\frac{u^2}{2}\\ &+(1+\ell)(3+\ell)\left[(2+\ell)\sqrt{1-u}+\frac{3u}{2}-1\right]\bigg).
  \end{split}
\end{equation}

Following the procedure implemented in the transition of Eq.~\eqref{Eq:TransN1} to Eq.~\eqref{Eq:AsymInfinite}, the differential equation Eq.~\eqref{eq:onda-s12} becomes
%
\begin{equation}\label{eq:onda-s32a}
  \begin{split}
    &R(\chi)\phi_{\scriptscriptstyle{3/2}}(\chi) + Q(\chi)\phi_{\scriptscriptstyle{3/2}}'(\chi) + P(\chi) \,\phi_{\scriptscriptstyle{3/2}}''(\chi)=0,
  \end{split}
\end{equation}
%
where the coefficients $R(\chi)$, $Q(\chi)$, and $P(\chi)$ are given by
%
\begin{eqnarray}
  R(\chi)=
  \,&& 2\big(1-\chi^2\big)\Big[6+11\ell+6\ell^2+\ell^3 +2\chi^5+  \nonumber\\
    &&4\chi^4(2+\ell) +2\chi^3\left(3+4\ell+\ell^2\right)- \nonumber\\
    &&\!\!\chi^2\left(2-7\ell-6\ell^2-\ell^3\right)  + 2\chi(2+\ell)^2\left(2+4\ell+\ell^2\right)\Big] \nonumber \\
  &&\! -16\chi(2+\chi+\ell)^2\lambda\Big[i + 4 \left(2-\chi^2\right)\left(1-\chi^2\right)\lambda\Big],\nonumber\\
  Q(\chi) = && -\left(1-\chi^2\right)\left(2+\chi+\ell\right)^2\nonumber\\
  && \times \left[\left(1-\chi^2\right)^2-8\,i\,(1-4\chi^2+2\chi^4)\lambda\right],\nonumber\\
  P(\chi)  =&&
  -\chi\left(1-\chi^2\right)^3\left(2+\chi+\ell\right)^2,\nonumber
\end{eqnarray}
%
where we have used the new coordinate $\chi^2=1-u$ to avoid square roots. Once again, we obtain a quadratic eigenvalue problem, and the function $\phi_{\scriptscriptstyle{3/2}}(\chi)$ is regular in the interval $\chi\in [0,\,1]$.

\subsection{Spin $5/2$ perturbations}

Exploring higher spin fields is considered a promising approach to obtaining a better understanding of fundamental physics, including the exploration of new unifying theories for fundamental interactions or novel phenomenology that extends beyond the standard model, as demonstrated by the authors of Ref.~\cite{Shklyar:2009cx}. In their work, physical observables for the spin $5/2$ field were computed. In our work, we utilize the generic equation derived in Ref.\cite{Shu:2005fw}, specifically Eq.~(37), to determine the quasinormal frequencies of this perturbation field on the Schwarzschild black hole. However, due to the size of the resulting differential equation for this perturbation field, we refer the reader to Appendix A of Ref.\cite{Mamani2022} for details.