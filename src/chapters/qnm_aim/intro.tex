When a closed physical system (like a guitar string) is perturbed, it relaxes by emitting certain natural frequencies know as \emph{normal modes}. If, however, the system is open (and therefore energy is being somehow dissipated away), its emitted natural frequencies will decay with time (like for instance sounding a bell in a church). These decaying modes are called \emph{quasinormal modes} (QNMs). Such frequencies can be used to obtain information about the system that produces them and black holes, like church bells, are also subject to these phenomena: Perturbed black holes relax by emitting waves in characteristic frequencies that decay with time, thanks to the dissipative nature of the event horizon. See Refs.~\cite{review1, review2, review3, review4} for an in-depth review of the quasinormal mode problem in the context of general relativity and black holes. Determining these characteristic frequencies quickly and accurately for a large range of models is important for many practical reasons. It has been shown that the gravitational wave signal emitted at the final stage of the coalescence of two compact objects is well described by quasinormal modes~\cite{buonanno,seidel}. This means that if one has access to a database of quasinormal modes and of gravitational wave signals from astrophysical collision events, it is possible to characterize the remnant object using its quasinormal frequencies. Since there are many models that aim to describe remnants, being able to compute the quasinormal frequencies for such models reliably is paramount for confirming or discarding them. Very often, computing quasinormal modes reduces to finding the discrete eigenvalue set of a second order differential equation with appropriate boundary conditions and asymptotic behavior. In this chapter, we will explore a numerical technique recently developed for tackling this problem, known as the \emph{Asymptotic Iteration Method} (AIM). The groundwork of the technique was laid out in Ref.~\cite{aim_original} and in Ref.~\cite{aim_improved} the method was refined and adapted to GR.

Motivated by these developments, we have implemented \texttt{QuasinormalModes.jl} (see the acompanying paper in Ref.~\cite{Sanches2022}), a \texttt{Julia}~\cite{Bezanson2017} software package for finding quasinormal modes using the AIM. Not only that, the package can be used to compute the discrete eigenvalues of \emph{any} second order homogeneous ODE (such as the energy eigenstates of the time independent Schrödinger equation) provided that these eigenvalues actually exist. The package features a flexible and user-friendly API where the user simply needs to provide the coefficients of the problem ODE after incorporating boundary and asymptotic conditions on it. The user can also choose to use machine or arbitrary precision arithmetic for the underlying floating point operations involved and whether to do computations sequentially or in parallel using threads. The API also tries not to force any particular workflow on the users so that they can incorporate and adapt the existing functionality on their research pipelines without unwanted intrusions. Often user-friendliness, flexibility and performance are treated as mutually exclusive, particularly in scientific applications. By using \texttt{Julia} as an implementation language, the package can have all these features simultaneously. Another important motivation for using \texttt{Julia} and writing this package was the lack of generalist, free (both in the financial and license-wise sense) open source tools that serve the same purpose. More precisely, there are tools which are free and open source, but run on top of a proprietary paid and expensive software framework such as the ones developed in Refs.~\cite{qnmspectral,spectralbp}, which are both excellent packages that aim to perform the same task as \texttt{QuasinormalModes.jl} and can be obtained and modified freely but, unfortunately, require the user to own a license of the proprietary \texttt{Wolfram Mathematica} CAS. Furthermore, their implementations are limited to solve problems where the eigenvalues must appear in the ODE as polynomials of order $p$. While this is not prohibitively restrictive to most astrophysics problems, it can be an important limitation in other areas. There are also packages that are free and run on top of \texttt{Mathematica} but are not aimed at being general eigenvalue solvers at all, such as the one in Ref.~\cite{bhpt_quasinormalmodes}, that can only compute modes of Schwarzschild and Kerr black holes. Finally, the \texttt{Python} package in Ref.~\cite{bhpt_qnm} is open source and free but can only compute Kerr quasinormal modes. \texttt{QuasinormalModes.jl} fills the existing gap for free, open source tools that are able to compute discrete eigenvalues (and in particular, quasinormal modes) efficiently for a broad class of models and problems. The package was used in Ref.~\cite{Mamani2022} where non-scalar perturbations were considered. Novel frequencies were obtained and results were compared against literature values, when possible, while also cross-checking results for the same models obtained via the more traditional pseudo-spectral method.
