The code was originally created with the intent of producing Gravitational Lensing images (hence the name) but was later adapted to investigate the PP. The code consists of a \emph{kernel} using the ARKODE~\cite{ARKODE} infrastructure for integrating the geodesic equation and verifying the stopping criteria of Eqs.~\eqref{eq:arbitrary_penrose_background_sphere_cplision_condition} and \eqref{eq:arbitrary_penrose_swallowing_condition} for an arbitrary spacetime metric. The kernel expects only to receive a set of functions that compute the ADM quantities of the spacetime metric (lapse, shift, extrinsic curvature and a few derivatives of these objects). Each spacetime metric is then a \emph{plugin} (a dynamically loaded library) that provides such functions at runtime. This allows the integration problem be separated of the problem of determining the ADM quantities of a given spacetime metric. Users can focus on the latter, since the kernel solves the former for an arbitrary input metric. In addition to spacetime metrics being plugins, file readers and writers are also plugins in the same way. This allows the user to implement different input/output data formats for the trajectories according to their needs. Currently, only a simple \texttt{ASCII} file writer is available, but that is enough for our current purposes. The behavior of the program is driven at runtime by a command line interface, \texttt{YAML} configuration files and the available dynamic libraries implementing spacetime metrics and writers. A basic configuration file named \texttt{grlensing\_config.yaml} is expected to be present, detailing various ARKODE internal settings, such as error tolerances, max. number of iterations, etc. It also details the metric and file writer plugins to be loaded and made available at runtime as well as settings of the \texttt{dump-metric} command. Additional configuration files are required in certain modes (for instance, describing a certain spacetime parameters and initial conditions of a particle). Through the command line, the user selects the operation mode of the code. At the time of writing, the available modes of operation are (these can be viewed by invoking the program with \texttt{--help})
%
\begin{itemize}
  \item \texttt{list-plugins}: Lists all the plugins that are set to be loaded (and were found)
  \item \texttt{dump-metric}: Writes a spacetime metric in a cube of arbitrary size and arbitrary number of internal points. This option is used mostly for debugging the implementation of a spacetime metric plugin.
  \item \texttt{integrate-trajectory}: Integrates a single particle trajectory with the specified configurations.
  \item \texttt{penrose-breakup}: Integrates a particle breakup process by using two particle configurations and obtaining a third from conservation of 4-momentum.
\end{itemize}
%
The general workflow for finding explicit examples of the PP is as follows:
%
\begin{enumerate}
  \item Chose a break-up point. This will be the initial position of all three particle participating in the process. We know that in order to extract energy this break-up must occur inside the ergosphere. In general, when the spacetime under study does not possess a global time-like killing vector field, we consider the ergosphere to be the now observer dependent region where the metric component $\mtrtens{t}{t}$ changes sign.
  \item Chose initial velocities and energy for the ingoing particle in such a way that it is an escaping orbit when integrated forward and backward in time. Requiring that the orbit is escaping in both ``temporal directions'' makes it more likely that one of the particles produced after the break-up will escape to infinity.
  \item Choose initial velocities and energy for the particle that is absorbed by the black whole and will be the equivalent of the negative energy orbit. These parameters are chosen so that the particle is counter-rotating with the black whole that will absorb it, so that its angular momentum can actually be decreased upon absorption.
  \item The escape orbit parameters are computed automatically by the code via 4-momentum conservation at the break-up point.
\end{enumerate}
%
Several utility scripts are also provided in the program repository under the \texttt{resources} folder. These scripts serve multiple purposes, from plotting trajectories, energies, the ADM quantities of dumped spacetime metrics or even generating a skeleton of a metric plugin that can be filled by users. It also includes an assortment of papers and notes required in the development of the code.